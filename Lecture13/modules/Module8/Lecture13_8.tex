\begin{frame}
	\myheading{Module 13.8 : Deep Dream}
\end{frame}

%%%%%%%%%%%%%%%%%%%%%%%%%%%%%%%%%%%%%%%%%%%%%%%%%%%%%%%%%%%%%%%%%%%%%%%%%%%%%%%%%%%%%%%%%

\begin{frame}
	\begin{columns}
		\column{0.5\textwidth}	
		
		\begin{overlayarea}{\textwidth}{\textheight}
			\begin{tikzpicture}
				
				\onslide<1->{
					\node[inner sep=0pt] (B) at (-4.7,12.5){
					\input{modules/Module8/tikz_images/as3.tex}
					};
				}	
				
				\onslide<2->{\node[inner sep=0pt] (A) at (-5,8.5) {\includegraphics[scale=0.35]{images/sky.JPG}};
					
					\draw[thick] (-8.5,9) -- (-7.9,9);
					\draw[thick] (-8.5,9) -- (-8.5,11);
					\draw[->] (-8.5,11) -- (-8.15,11);
				}
				\onslide<3->{	
					\draw[blue] (-5.35,11.8) ellipse (0.05 and 0.09);		
					\draw[blue] (-5.35,12.25) ellipse (0.05 and 0.09);		
					\draw[blue] (-5.2,12.3) ellipse (0.05 and 0.09);				
					\draw[blue] (-5.4,12.5) ellipse (0.05 and 0.09);				
					\draw[blue] (-5.3,12.7) ellipse (0.05 and 0.09);				
					\draw[blue] (-5.1,13.1) ellipse (0.05 and 0.09);				
					\draw[blue] (-4.9,13.4) ellipse (0.05 and 0.09);				
					\draw[blue] (-4.85,13.65) ellipse (0.05 and 0.09);				
					\draw[blue] (-5.22,12.95) ellipse (0.05 and 0.09);	
					\draw[blue] (-5.16,12.65) ellipse (0.05 and 0.09);	
									
					\draw[blue] (-5.25,12) ellipse (0.05 and 0.09);		
					\draw[blue] (-5.05,12.25) ellipse (0.05 and 0.09);			
					\draw[blue] (-5,12.5) ellipse (0.05 and 0.09);			
					\draw[blue] (-5,12.8) ellipse (0.05 and 0.09);			
					\draw[blue] (-4.9,13) ellipse (0.05 and 0.09);			
					\draw[blue] (-4.8,12.6) ellipse (0.05 and 0.09);			
					%	\draw[blue] (-2.35,9.85) ellipse (0.05 and 0.09);			
				}
			\end{tikzpicture}
		\end{overlayarea}
		
		\column{0.5\textwidth}	
		\begin{overlayarea}{\textwidth}{\textheight}
			\begin{itemize}
				\justifying
				\onslide<1->{
					\item Suppose instead of starting with a blank (zero) image we start with an actual image.
					}	
				\onslide<3->{
					\item We focus on some layer and check the activations of the neurons 
					}	
				\onslide<4->{
					\item We want to change the image so that these neurons fire even more
					}	
								
			\end{itemize}
			
		\end{overlayarea}
	\end{columns}			
	
\end{frame}

%%%%%%%%%%%%%%%%%%%%%%%%%%%%%%%%%%%%%%%%%%%%%%%%%%%%%%%%%%%%%%%%%%%%%%%%%%%%%%%%%%%%%%%%%

\begin{frame}
\begin{columns}
	\column{0.5\textwidth}	
	
	\begin{overlayarea}{\textwidth}{\textheight}
		\begin{tikzpicture}
				
			\onslide<1->{
				\node[inner sep=0pt] (B) at (-4.7,12.5){
				\input{modules/Module8/tikz_images/as3.tex}
				};
			}	
				
			\onslide<1->{\node[inner sep=0pt] (A) at (-5,8.5) {\includegraphics[scale=0.35]{images/sky.JPG}};
						
				\draw[thick] (-8.5,9) -- (-7.9,9);
				\draw[thick] (-8.5,9) -- (-8.5,11);
				\draw[->] (-8.5,11) -- (-8.15,11);
			}
			\onslide<1->{	
				\draw[blue] (-5.35,11.8) ellipse (0.05 and 0.09);		
				\draw[blue] (-5.35,12.25) ellipse (0.05 and 0.09);		
				\draw[blue] (-5.2,12.3) ellipse (0.05 and 0.09);				
				\draw[blue] (-5.4,12.5) ellipse (0.05 and 0.09);				
				\draw[blue] (-5.3,12.7) ellipse (0.05 and 0.09);				
				\draw[blue] (-5.1,13.1) ellipse (0.05 and 0.09);				
				\draw[blue] (-4.9,13.4) ellipse (0.05 and 0.09);				
				\draw[blue] (-4.85,13.65) ellipse (0.05 and 0.09);				
				\draw[blue] (-5.22,12.95) ellipse (0.05 and 0.09);	
				\draw[blue] (-5.16,12.65) ellipse (0.05 and 0.09);	
						
				\draw[blue] (-5.25,12) ellipse (0.05 and 0.09);		
				\draw[blue] (-5.05,12.25) ellipse (0.05 and 0.09);			
				\draw[blue] (-5,12.5) ellipse (0.05 and 0.09);			
				\draw[blue] (-5,12.8) ellipse (0.05 and 0.09);			
				\draw[blue] (-4.9,13) ellipse (0.05 and 0.09);			
				\draw[blue] (-4.8,12.6) ellipse (0.05 and 0.09);			
				
			}
			\onslide<3->{
				\draw[draw=blue,fill=cyan!20] (-5,12.8) ellipse (0.05 and 0.09);	
				\node at (-4.3,13.8) {$h_{ij}$};		
				\draw[->] (-4.9,12.8) -- (-4.4,13.7);
			}
		\end{tikzpicture}
	\end{overlayarea}
	\column{0.5\textwidth}	
	\begin{overlayarea}{\textwidth}{\textheight}
		\begin{itemize}
			\justifying
			\onslide<1->{
				\item How would we achieve this? %(boost activations of neurons)
			}	
			\onslide<2->{
				\item Suppose we want to boost the activation $h_{ij}$ (some neuron in some layer)
			}	
			\onslide<3->{
				\item We can formulate this as the following optimization problem
			}	
			\onslide<4->{
				\begin{align*}
					\max_{I} \mathscr{L}(I)   \\
					\mathscr{L}(I) = h^2_{ij} 
				\end{align*}
			}
			\vspace{-0.3cm}
			\onslide<5->{
				\item Consider a pixel $i_{mn}$ in the image 
						\begin{align*}
				      	\frac{\partial \mathscr{L}(I)}{\partial i_{mn}} = \frac{\partial \mathscr{L}(I)}{\partial h_{ij}}\frac{\partial h_{ij}}{\partial i_{mn}} 
					\end{align*}
			}
											
			\end{itemize}
						
		\end{overlayarea}
	\end{columns}			
					
\end{frame}

%%%%%%%%%%%%%%%%%%%%%%%%%%%%%%%%%%%%%%%%%%%%%%%%%%%%%%%%%%%%%%%%%%%%%%%%%%%%%%%%%%%%%%%%%
							
\begin{frame}
	\begin{columns}
		\column{0.5\textwidth}	
		\begin{overlayarea}{\textwidth}{\textheight}
			\begin{tikzpicture}
					
				\onslide<1->{
					\node[inner sep=0pt] (B) at (-4.7,12.5){
						\input{modules/Module8/tikz_images/as3.tex}
					};
				}	
					
				\onslide<1->{\node[inner sep=0pt] (A) at (-5,8.5) {\includegraphics[scale=0.35]{images/sky.JPG}};
							
					\draw[thick] (-8.5,9) -- (-7.9,9);
					\draw[thick] (-8.5,9) -- (-8.5,11);
					\draw[->] (-8.5,11) -- (-8.15,11);
				}
				\onslide<1->{	
					\draw[blue] (-5.35,11.8) ellipse (0.05 and 0.09);		
					\draw[blue] (-5.35,12.25) ellipse (0.05 and 0.09);		
					\draw[blue] (-5.2,12.3) ellipse (0.05 and 0.09);				
					\draw[blue] (-5.4,12.5) ellipse (0.05 and 0.09);				
					\draw[blue] (-5.3,12.7) ellipse (0.05 and 0.09);				
					\draw[blue] (-5.1,13.1) ellipse (0.05 and 0.09);				
					\draw[blue] (-4.9,13.4) ellipse (0.05 and 0.09);				
					\draw[blue] (-4.85,13.65) ellipse (0.05 and 0.09);				
					\draw[blue] (-5.22,12.95) ellipse (0.05 and 0.09);	
					\draw[blue] (-5.16,12.65) ellipse (0.05 and 0.09);	
							
					\draw[blue] (-5.25,12) ellipse (0.05 and 0.09);		
					\draw[blue] (-5.05,12.25) ellipse (0.05 and 0.09);			
					\draw[blue] (-5,12.5) ellipse (0.05 and 0.09);			
					\draw[blue] (-5,12.8) ellipse (0.05 and 0.09);			
					\draw[blue] (-4.9,13) ellipse (0.05 and 0.09);			
					\draw[blue] (-4.8,12.6) ellipse (0.05 and 0.09);			
					\onslide<2->{	\draw[draw=blue,fill=cyan!50] (-5,12.8) ellipse (0.05 and 0.09);	
						\node at (-4.3,13.9) {$h_{ij}$};		
						\draw[->]  (-4.4,13.7) --  (-4.9,12.8);
					}	
				}
				\onslide<2->{
					\draw[draw=blue,fill=cyan!20] (-5,12.8) ellipse (0.05 and 0.09);	
						
				}
			\end{tikzpicture}
		\end{overlayarea}
		\column{0.5\textwidth}	
		\begin{overlayarea}{\textwidth}{\textheight}
			\begin{itemize}
				\justifying
				\onslide<1->{
					\item Once the image is updated $\Big(i_{mn} = i_{mn} + \frac{\partial \mathscr{L}(I)}{\partial i_{mn}}\Big)$ we feed it back to the network
				}	
				\onslide<2->{
					\item This time the target neurons should fire even more (because we have precisely modified the image to achieve this)
				}	
				\onslide<3->{
					\item Doing this iteratively would make the image more and more like the patterns that cause the neuron to fire
				}	
				\onslide<4->{
					\item Let us run this algorithm
				}	
								
								
			\end{itemize}
						
		\end{overlayarea}
	\end{columns}			
		
\end{frame}

%%%%%%%%%%%%%%%%%%%%%%%%%%%%%%%%%%%%%%%%%%%%%%%%%%%%%%%%%%%%%%%%%%%%%%%%%%%%%%%%%%%%%%%%%

\begin{frame}
	\foreach \x in {0,...,99}
	{
		\only<\x>{
			\centering
			\includegraphics[scale=0.4]{images/Sky-Frames/0_\x.jpg}
		}
	}
\end{frame}

%%%%%%%%%%%%%%%%%%%%%%%%%%%%%%%%%%%%%%%%%%%%%%%%%%%%%%%%%%%%%%%%%%%%%%%%%%%%%%%%%%%%%%%%%

\begin{frame}
	\foreach \x in {0,...,99}
	{
		\only<\x>{
			\centering
			\includegraphics[scale=0.4]{images/Monsoon-Frames/0_\x.jpg}
		}
	}
\end{frame}

%%%%%%%%%%%%%%%%%%%%%%%%%%%%%%%%%%%%%%%%%%%%%%%%%%%%%%%%%%%%%%%%%%%%%%%%%%%%%%%%%%%%%%%%%

\begin{frame}
	\foreach \x in {0,...,99}
	{
		\only<\x>{
			\centering
			\includegraphics[scale=0.33]{images/Leonardo-frames/0_\x.jpg}
		}
	}
\end{frame}

%%%%%%%%%%%%%%%%%%%%%%%%%%%%%%%%%%%%%%%%%%%%%%%%%%%%%%%%%%%%%%%%%%%%%%%%%%%%%%%%%%%%%%%%%

\begin{frame}
\begin{columns}
	\column{0.45\textwidth}	
	
	\begin{overlayarea}{\textwidth}{\textheight}
		\begin{tikzpicture}
				
			\onslide<1->{
				\node[inner sep=0pt] (B) at (-4.7,12.5){
					\input{modules/Module8/tikz_images/as3.tex}
				};}	
				
			\onslide<1->{\node[inner sep=0pt] (A) at (-5,8.5) {\includegraphics[scale=0.35]{images/sky.JPG}};
						
				\draw[thick] (-8.5,9) -- (-7.9,9);
				\draw[thick] (-8.5,9) -- (-8.5,11);
				\draw[->] (-8.5,11) -- (-8.15,11);
			}
			\onslide<1->{	
				\draw[draw=blue,fill=myblue1] (-5.35,11.8) ellipse (0.05 and 0.09);		
				\draw[draw=blue,fill=myblue2] (-5.35,12.25) ellipse (0.05 and 0.09);		
				\draw[draw=blue,fill=myblue3] (-5.2,12.3) ellipse (0.05 and 0.09);				
				\draw[draw=blue,fill=myblue4] (-5.4,12.5) ellipse (0.05 and 0.09);				
				\draw[draw=blue,fill=myblue1] (-5.3,12.7) ellipse (0.05 and 0.09);				
				\draw[draw=blue,fill=myblue2] (-5.1,13.1) ellipse (0.05 and 0.09);				
				\draw[draw=blue,fill=myblue3] (-4.9,13.4) ellipse (0.05 and 0.09);				
				\draw[draw=blue,fill=myblue4] (-4.85,13.65) ellipse (0.05 and 0.09);			
				\draw[draw=blue,fill=myblue1] (-5.22,12.95) ellipse (0.05 and 0.09);	
				\draw[draw=blue,fill=myblue2] (-5.16,12.65) ellipse (0.05 and 0.09);	
						
				\draw[draw=blue,fill=myblue3] (-5.25,12) ellipse (0.05 and 0.09);		
				\draw[draw=blue,fill=myblue4] (-5.05,12.25) ellipse (0.05 and 0.09);			
				\draw[draw=blue,fill=myblue1] (-5,12.5) ellipse (0.05 and 0.09);			
				\draw[draw=blue,fill=myblue2] (-5,12.8) ellipse (0.05 and 0.09);			
				\draw[draw=blue,fill=myblue3] (-4.9,13) ellipse (0.05 and 0.09);			
				\draw[draw=blue,fill=myblue4] (-4.8,12.6) ellipse (0.05 and 0.09);			
						
			}
				
		\end{tikzpicture}
	\end{overlayarea}
	
			
			
	\column{0.55\textwidth}	
	\begin{overlayarea}{\textwidth}{\textheight}
		\begin{itemize}
			\justifying
			\onslide<1->{
				\item So what exactly is happening here?	
			}	
			\onslide<2->{
				\item The network has been trained to detect certain patterns (dogs, cat, birds etc.) which appear frequently in the ImageNet data
			}	
			\onslide<3->{
				\item It starts seeing these patterns even when they hardly exist
			}	
			\onslide<4->{
				\item \it{If a cloud looks a little bit like a bird, the network will make it look more like a bird. This in turn will make the network recognize the bird even more strongly on the next pass and so forth, until a highly detailed bird appears seemingly out of nowhere. - Google$^*$\footnotetext{$^*$\url{research.googleblog.com/2015/06/inceptionism-}\\ \url{going-deeper-into-neural.html}}}}	
			      				
			      				
		\end{itemize}
					
	\end{overlayarea}
	\end{columns}			
		
\end{frame}
