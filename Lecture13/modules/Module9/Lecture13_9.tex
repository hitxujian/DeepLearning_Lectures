\begin{frame}
	\myheading{Module 13.9 : Deep Art}
\end{frame}

%%%%%%%%%%%%%%%%%%%%%%%%%%%%%%%%%%%%%%%%%%%%%%%%%%%%%%%%%%%%%%%%%%%%%%%%%%%%%%%%%%%%%%%%%

\begin{frame}
	\begin{center}
		\begin{tikzpicture}

	\onslide<1->{\node[inner sep=0pt] (A) at (-5,9) {\includegraphics[scale=0.45]{images/12.JPG}};}
	\onslide<2->{\node[inner sep=0pt] (A) at (-2.5,9) {\includegraphics[scale=0.1]{images/plus.png}};}
	
	\onslide<3->{\node[inner sep=0pt] (A) at (0,9) {\includegraphics[scale=0.6]{images/11.JPG}};}
	
	\onslide<4->{\node[inner sep=0pt] (A) at (2.5,9) {\includegraphics[scale=0.1]{images/equal.png}};}
	
	\onslide<5->{\node[inner sep=0pt] (A) at (5,9) {\includegraphics[scale=0.45]{images/13.JPG}};}

\end{tikzpicture}
	\end{center}

\end{frame}

%%%%%%%%%%%%%%%%%%%%%%%%%%%%%%%%%%%%%%%%%%%%%%%%%%%%%%%%%%%%%%%%%%%%%%%%%%%%%%%%%%%%%%%%%

\begin{frame}
	\begin{columns}
		\column{0.5\textwidth}	
		\vspace{0.15cm}
		\begin{overlayarea}{\textwidth}{\textheight}
			\input{modules/Module9/tikz_images/content_cnn.tex}
		\end{overlayarea}
		
		\column{0.5\textwidth}	
		\begin{overlayarea}{\textwidth}{\textheight}
			\begin{itemize}
				\justifying
				\onslide<1->{
				\item To design a network which can do this, we first define two quantities
				}	
				\onslide<2->{
				\item \textbf{Content Targets} : The activations of all layers for the given content image
				}	
				\onslide<3->{
				\item Ideally, we would want the new image to be such that \onslide<4->{it's activations are also close to those of the original content image }
				}	
				\onslide<5->{
				\item Let $\vec{p},\vec{x}$ be the activations of the content image and the new image (to be generated) respectively
				\begin{align*}
				     	\mathscr{L}_{content}(\vec{p},\vec{x}) = \sum_{ijk}^{}(\vec{p}_{ijk} - \vec{x}_{ijk})^2 
				\end{align*}
				}		
			\end{itemize}
			
		\end{overlayarea}
	\end{columns}			

\end{frame}

%%%%%%%%%%%%%%%%%%%%%%%%%%%%%%%%%%%%%%%%%%%%%%%%%%%%%%%%%%%%%%%%%%%%%%%%%%%%%%%%%%%%%%%%%

\begin{frame}
	\begin{columns}
		\column{0.5\textwidth}	
		\hspace{-0.5cm}
		\begin{overlayarea}{\textwidth}{\textheight}
			\begin{tikzpicture}
					
	\onslide<1->{\node[inner sep=0pt] (A) at (-5.4,4.8) {\includegraphics[scale=0.2]{images/starry.png}};
	
	%\node at (-5,3.8){\small Content Image};
	}
	\hspace{-0.5cm}
	\onslide<1->{\node[inner sep=0pt] (B) at (-1.1,4.8) {\input{modules/Module9/tikz_images/as3_1.tex}};
	\draw [->,thick] (-4.1,4.8) -- (-3.7,4.8);
	}
	\onslide<4->{\node[inner sep=0pt] (A) at (-5,1.5) {\includegraphics[scale=0.17]{images/a.JPG}};
	\draw	[->] (-3.1,3.4) -- (-4.5,2.2);
	}
	\onslide<5->{\node[inner sep=0pt] (A) at (-3.3,1.5) {\includegraphics[scale=0.17]{images/b.JPG}};
	\draw	[->] (-2,3.7) -- (-3.1,2.2);
	}
		
	\onslide<6->{\node[inner sep=0pt] (A) at (-1.65,1.5) {\includegraphics[scale=0.17]{images/c.JPG}};
	\draw	[->] (-0.6,3.9) -- (-1.5,2.2);
	}		
	\onslide<7->{\node[inner sep=0pt] (A) at (0,1.5) {\includegraphics[scale=0.17]{images/d.JPG}};
	\draw	[->] (0.2,4.1) -- (-0.3,2.2);
	}			\onslide<7->{\node[inner sep=0pt] (A) at (1.6,1.5) {\includegraphics[scale=0.17]{images/e.JPG}};
	\draw	[->] (0.8,4.2) -- (1.4,2.2);
	}						
						
\end{tikzpicture}
		\end{overlayarea}
		\column{0.5\textwidth}	
		\begin{overlayarea}{\textwidth}{\textheight}
			\begin{itemize}
				\justifying
				\onslide<1->{
					\item Next we would want the style of the generated image to be the same as the style image				}	
				\onslide<2->{
		     		\item How do we capture the style of the image?
		     	}	
		     	\onslide<3->{
		     		\item Turns out that if $V \in \mathbb{R}^{64 \times (256 \times 256)}$ is the activation at a layer then $V^{T}V \in$ $\mathbb{R}^{64\times64}$ captures the style of the image
	     		}	
	     		\onslide<4->{
	     			\item The deeper layers capture more of this style information
	   			}						
	   		\end{itemize}
	     					
    	\end{overlayarea}
	\end{columns}			
	     	
\end{frame}

%%%%%%%%%%%%%%%%%%%%%%%%%%%%%%%%%%%%%%%%%%%%%%%%%%%%%%%%%%%%%%%%%%%%%%%%%%%%%%%%%%%%%%%%%

\begin{frame}
	\begin{columns}
	\column{0.5\textwidth}	
	\vspace{1cm}
	\begin{overlayarea}{\textwidth}{\textheight}
		\begin{tikzpicture}
			
	\onslide<1->{\node[inner sep=0pt] (A) at (-5,4.8) {\includegraphics[scale=0.3]{images/11.JPG}};
		\node at (-5,3.8){\small Content Image};
	}
			
	\onslide<1>{\node[inner sep=0pt] (B) at (-1,4.8) {\input{modules/Module9/tikz_images/as3_1.tex}};
		\draw [->,thick] (-4.3,4.8) -- (-3.7,4.8);
	}
	\onslide<1->{\node[inner sep=0pt] (A) at (-5,1.7) {\includegraphics[scale=0.12]{images/ques.jpeg}};
		\node at (-5,3.8){\small Content Image};
		\draw [->] (-4.3,4.8) -- (-3.7,4.8);
	}
			
					
	\onslide<1>{\node[inner sep=0pt] (B) at (-1.2,2) {\input{modules/Module9/tikz_images/as3_1.tex}};
		\draw [->] (-4.3,2) -- (-3.6,2);
	}
					
	\onslide<2->{\node[inner sep=0pt] (B) at (-1.2,2) {\input{modules/Module9/tikz_images/as32.tex}};
		\draw [->] (-4.3,2) -- (-3.6,2);
							
		\draw [->,thick] (-4.3,4.8) -- (-3.7,4.8);
		\node[inner sep=0pt] (B) at (-1,4.8) {\input{modules/Module9/tikz_images/as32.tex}};
		\draw (-1.5,3) -- (-1.5,4.05);
		\node at (-1.5,3) {\tiny $V_1^T V_1$};
		\node at (-1.55,3.95) {\tiny $V_2^T V_2$};
		\node [text width = 30mm]at (0.2,3.4) {\tiny equal};
	}
							
\end{tikzpicture}
	\end{overlayarea}
	\column{0.5\textwidth}	
	\begin{overlayarea}{\textwidth}{\textheight}
		\begin{itemize}
			\justifying
			\onslide<1->{
				\item To ensure that the style of the new image captured by layer $\ell$ matches the style of the style image, we can use the following objective function :
				}	
				\onslide<2->{
					\begin{align*} 
						E_{\ell} & = \sum_{ij}{}(G^{\ell}_j - A^{\ell}_{ij})^2 
					\end{align*}
					where $G^\ell$ and 
					$A^\ell$ are the style gram matrices computed at layer $\ell$ for the style image and new image respectively.
										
				}	
				\onslide<3->{
					\vspace{-2em}
					\begin{align*}
						\mathscr{L}_{style}(\vec{a},\bar{x})=\sum_{\ell=0}^{L}w_{\ell} E_{\ell} 
					\end{align*}
				}	
								
								
			\end{itemize}
						
		\end{overlayarea}
	\end{columns}			
	
\end{frame}

%%%%%%%%%%%%%%%%%%%%%%%%%%%%%%%%%%%%%%%%%%%%%%%%%%%%%%%%%%%%%%%%%%%%%%%%%%%%%%%%%%%%%%%%%

\begin{frame}  
	\hspace{-1cm} 
	\begin{overlayarea}{\textwidth}{\textheight}
		\begin{tikzpicture}
	
	\onslide<1->{\node[inner sep=0pt] (A) at (-5,4.8) {\includegraphics[scale=0.2]{images/12.JPG}};
		%   \node at (-5,3.8){\small Content Image};
	}
	
	\onslide<1>{\node[inner sep=0pt] (B) at (-1,4.8) {\input{modules/Module9/tikz_images/as3_12.tex}};
		\draw [->,thick] (-4.3,4.8) -- (-3.7,4.8);
	}
	
	\onslide<1->{\node[inner sep=0pt] (A) at (3,4.8) {\includegraphics[scale=0.3]{images/11.JPG}};
		%	\node at (-5,3.8){\small Content Image};
	}
	
	\onslide<1>{\node[inner sep=0pt] (B) at (6.9,4.8) {\input{modules/Module9/tikz_images/as3_12.tex}};
		\draw [->,thick] (3.8,4.8) -- (4.3,4.8);
	}
	
	\onslide<1>{\node[inner sep=0pt] (A) at (-0.5,1.7) {\includegraphics[scale=0.12]{images/ques.jpeg}};
		
		%    \node at (-5,3.8){\small Content Image};
		%    \draw [->] (0.5,.5) -- (1.2,3.5);
	}
	

	\onslide<1>{\node[inner sep=0pt] (B) at (3.5,2) {\input{modules/Module9/tikz_images/as3_12.tex}};
		\draw [->] (0,1.9) -- (0.7,1.9);
	}

	\onslide<2->{
		\draw [->,thick] (-4.3,4.8) -- (-3.7,4.8);
		\draw [->,thick] (3.8,4.8) -- (4.3,4.8);
		\node[inner sep=0pt] (A) at (-0.5,1.7) {\includegraphics[scale=0.2]{images/13.JPG}};
		\draw [->] (0.1,1.9) -- (0.8,1.9);
		\node[inner sep=0pt] (B) at (3.5,2) {\input{modules/Module9/tikz_images/as32.tex}};
		\node[inner sep=0pt] (B) at (6.9,4.8) {\input{modules/Module9/tikz_images/as32.tex}};
		\node[inner sep=0pt] (B) at (-1,4.8) {\input{modules/Module9/tikz_images/as32.tex}};
					
					
		\draw [->,dashed] (-1.7,3.9) -- (3,2.8);
		\draw [->,dashed] (6.2,3.9) -- (3.3,2.75);
	}


\end{tikzpicture}
	\end{overlayarea}
	\vspace{-1.7cm}
	\begin{itemize}
		\justifying
		\onslide<3->{
			\item The total loss is given by :-
			\begin{align*}
		      	\mathscr{L}_{total} (\vec{p},\vec{a},\vec{x}) = \alpha \mathscr{L}_{content}(\vec{p},\vec{x}) + \beta \mathscr{L}_{style}(\vec{a},\vec{x}) 
			\end{align*}
			              		
		}   
			        	
			        	
	\end{itemize}   
\end{frame}   