\begin{frame}
	\myheading{Module 13.7 : Creating images from embeddings}
\end{frame}

%%%%%%%%%%%%%%%%%%%%%%%%%%%%%%%%%%%%%%%%%%%%%%%%%%%%%%%%%%%%%%%%%%%%%%%%%%%%%%%%%%%%%%%%%

\begin{frame}
	\begin{columns}
		\column{0.5\textwidth}
		\begin{overlayarea}{\textwidth}{\textheight}
			\input{modules/Module7/tikz_images/AlexNet_Slide4_4.tex}
		\end{overlayarea}
		\column{0.5\textwidth}
		\begin{overlayarea}{\textwidth}{\textheight}
			\begin{itemize}
				\justifying
				\onslide<1->{\item We could think of the fc7 layer as some kind of an embedding for the image}
				\onslide<2->{\item \textbf{Question: }Given this embedding can we reconstruct the image?}
				\onslide<3->{\item We can pose this as an optimization problem}
			\end{itemize}
		\end{overlayarea}
	\end{columns}
\end{frame}

%%%%%%%%%%%%%%%%%%%%%%%%%%%%%%%%%%%%%%%%%%%%%%%%%%%%%%%%%%%%%%%%%%%%%%%%%%%%%%%%%%%%%%%%%

\begin{frame}
	\begin{columns}
		\column{0.5\textwidth}
		\begin{overlayarea}{\textwidth}{\textheight}
			\input{modules/Module7/tikz_images/AlexNet_Slide4_4.tex}
		\end{overlayarea}
		\column{0.5\textwidth}
		\begin{overlayarea}{\textwidth}{\textheight}
			\begin{itemize}
				\justifying
				\onslide<1->{\item Find an image such that }
				\onslide<2->{\item Its embedding is similar to a given embedding}
				\onslide<3->{\item It looks natural (some prior regularization)}
			\end{itemize}
		\end{overlayarea}
	\end{columns}
\end{frame}

%%%%%%%%%%%%%%%%%%%%%%%%%%%%%%%%%%%%%%%%%%%%%%%%%%%%%%%%%%%%%%%%%%%%%%%%%%%%%%%%%%%%%%%%%

\begin{frame}
	\begin{columns}
		\column{0.5\textwidth}
		\begin{overlayarea}{\textwidth}{\textheight}
			\input{modules/Module7/tikz_images/AlexNet_Slide4_4.tex}
		\end{overlayarea}
		\column{0.5\textwidth}
		\begin{overlayarea}{\textwidth}{\textheight}
			\begin{itemize}
				\justifying
				\onslide<1->{\item $\phi_0: $Embedding of an image of interest}
				\onslide<2->{\item $X: $Random image (say zero image)}
				\onslide<3->{\item Repeat
					\begin{itemize}
						\justifying
						\onslide<4->{\item Forward pass using $X$ and compute $\phi(x).$}
						\onslide<5->{\item Compute 
							\begin{align*}
								\mathscr{L}(i) = ||\phi(x) - \phi_{0} ||^2 + \lambda ||\phi(x) ||_6^6 
							\end{align*}
						}
						\onslide<6->{\item  $i_k = i_k - \eta \frac{\mathscr{L}(i)}{\partial i_k} $}
					\end{itemize}								
				}
			\end{itemize}
		\end{overlayarea}
	\end{columns}
\end{frame}

%%%%%%%%%%%%%%%%%%%%%%%%%%%%%%%%%%%%%%%%%%%%%%%%%%%%%%%%%%%%%%%%%%%%%%%%%%%%%%%%%%%%%%%%%

\begin{frame}
	\begin{overlayarea}{\textwidth}{\textheight}
		\begin{columns}
			\column{0.5\textwidth}
			\begin{figure}[hbtp]
				\includegraphics[scale=0.5]{images/monkey/l01-orig.jpg}
				\caption{Original Image}
			\end{figure}
			\column{0.5\textwidth}
			\only<1>{
				\begin{figure}[hbtp]
					\includegraphics[scale=0.5]{images/monkey/l1-recon.jpg}
					\caption{Conv-1}
				\end{figure}
			}
			\only<2>{
				\begin{figure}[hbtp]
					\includegraphics[scale=0.5]{images/monkey/l2-recon.jpg}
					\caption{Relu-1}
				\end{figure}
			}
			\only<3>{
				\begin{figure}[hbtp]
					\includegraphics[scale=0.5]{images/monkey/l3-recon.jpg}
					\caption{Mpool-1}
				\end{figure}
			}
			\only<4>{
				\begin{figure}[hbtp]
					\includegraphics[scale=0.5]{images/monkey/l4-recon.jpg}
					\caption{Norm-1}
				\end{figure}
			}
			\only<5>{
				\begin{figure}[hbtp]
					\includegraphics[scale=0.5]{images/monkey/l5-recon.jpg}
					\caption{Conv-2}
				\end{figure}
			}
			\only<6>{
				\begin{figure}[hbtp]
					\includegraphics[scale=0.5]{images/monkey/l6-recon.jpg}
					\caption{Relu-2}
				\end{figure}
			}
			\only<7>{
				\begin{figure}[hbtp]
					\includegraphics[scale=0.5]{images/monkey/l7-recon.jpg}
					\caption{Mpool-2}
				\end{figure}
			}
			\only<8>{
				\begin{figure}[hbtp]
					\includegraphics[scale=0.5]{images/monkey/l8-recon.jpg}
					\caption{Norm-2}
				\end{figure}
			}
			\only<9>{
				\begin{figure}[hbtp]
					\includegraphics[scale=0.5]{images/monkey/l9-recon.jpg}
					\caption{Conv-3}
				\end{figure}
			}
			\only<10>{
				\begin{figure}[hbtp]
					\includegraphics[scale=0.5]{images/monkey/l10-recon.jpg}
					\caption{Relu-3}
				\end{figure}
			}
			\only<11>{
				\begin{figure}[hbtp]
					\includegraphics[scale=0.5]{images/monkey/l11-recon.jpg}
					\caption{Conv-4}
				\end{figure}
			}
			\only<12>{
				\begin{figure}[hbtp]
					\includegraphics[scale=0.5]{images/monkey/l12-recon.jpg}
					\caption{Relu-4}
				\end{figure}
			}
			\only<13>{
				\begin{figure}[hbtp]
					\includegraphics[scale=0.5]{images/monkey/l13-recon.jpg}
					\caption{Conv-5}
				\end{figure}
			}
			\only<14>{
				\begin{figure}[hbtp]
					\includegraphics[scale=0.5]{images/monkey/l14-recon.jpg}
					\caption{Relu-5}
				\end{figure}
			}
			\only<15>{
				\begin{figure}[hbtp]
					\includegraphics[scale=0.5]{images/monkey/l15-recon.jpg}
					\caption{Mpool-5}
				\end{figure}
			}
			\only<16>{
				\begin{figure}[hbtp]
					\includegraphics[scale=0.5]{images/monkey/l16-recon.jpg}
					\caption{FC-6}
				\end{figure}
			}
			\only<17>{
				\begin{figure}[hbtp]
					\includegraphics[scale=0.5]{images/monkey/l17-recon.jpg}
					\caption{Relu-6}
				\end{figure}
			}
			\only<18>{
				\begin{figure}[hbtp]
					\includegraphics[scale=0.5]{images/monkey/l18-recon.jpg}
					\caption{FC-7}
				\end{figure}
			}
			\only<19>{
				\begin{figure}[hbtp]
					\includegraphics[scale=0.5]{images/monkey/l19-recon.jpg}
					\caption{Relu-7}
				\end{figure}
			}
			\only<20>{
				\begin{figure}[hbtp]
					\includegraphics[scale=0.5]{images/monkey/l20-recon.jpg}
					\caption{FC-8}
				\end{figure}
			}
		\end{columns}
	\end{overlayarea}
\end{frame}
