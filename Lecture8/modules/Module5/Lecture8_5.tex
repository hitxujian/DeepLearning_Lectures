\begin{frame}
	\myheading{Module 8.5 : Dataset augmentation}
\end{frame}

\begin{frame}
\vspace{4em}
	\begin{overlayarea}{\textwidth}{\textheight}
		\begin{block}{Different forms of regularization}
			\begin{itemize}
				\item $L2$ regularization
				\item \textcolor<2->{red}{Dataset augmentation}
				\item Parameter Sharing and tying
				\item Adding Noise to the inputs 
				\item Adding Noise to the outputs 
				\item Early stopping
				\item Ensemble methods
				\item Dropout
			\end{itemize}
		\end{block}
	\end{overlayarea}
\end{frame}
\begin{frame}
	\begin{columns}
		\column{0.3\textwidth}
		\begin{center}
			\begin{tikzpicture}
	\draw (0,3) rectangle (1.5,4.5);
	\node at (0.75,2.7) {label $= 2$}; 
	\node[inner sep=0pt] (russell) at (0.75,3.75)
	{\includegraphics[scale=1.5]{images/two.jpg}};
\end{tikzpicture}
		\end{center}
		\onslide<2->{\hspace{1em}[given training data]\\}
		\onslide<12->{We exploit the fact that certain transformations to the image do not change the label of the image.}
					
		\column{0.7\textwidth}
		\onslide<3->{
			\hspace{2em}
			\vspace{2em}
			\begin{overlayarea}{\textwidth}{\textheight}
				\begin{tikzpicture}
	\onslide<3->{\draw (-2,-1.6) rectangle (7,4.5);} %} node[left =2,below=2] 
	\onslide<10->{\node at (2.4,-1.9) {label $= 2$};}
	\onslide<4->{\draw (-1.5,2.5) rectangle (0.5,4.3); %node[left=2,below=2] 
		\node[inner sep=0pt] (russell) at (-0.5,3.45)
		{\includegraphics[scale=1.5,angle=20]{images/two.jpg}};
		\node at (-0.5,2.2) { rotated by $20^{\circ}$}; }
						
	\onslide<5->{\draw (1.5,2.5) rectangle (3.5,4.3);% node[below=2] 
		\node[inner sep=0pt] (russell) at (2.5,3.45)
		{\includegraphics[scale=1.5,angle=65]{images/two.jpg}};
		\node at (2.6,2.2) { rotated  by $65^{\circ}$}; }
						
	\onslide<6->{\draw (4.5,2.5) rectangle (6.5,4.3); %node[below=2] 
		\node[inner sep=0pt] (russell) at (5.5,3.15)
		{\includegraphics[scale=1.5]{images/two.jpg}};
		\node at (5.5,2.2) { shifted vertically};  }
						
	\onslide<7->{\draw (-1.5,-1) rectangle (0.5,1); % node[below=2] 
		\node[inner sep=0pt] (russell) at (-0.8,0)
		{\includegraphics[scale=1.5]{images/two.jpg}};
		\node at (-0.4,-1.3) {shifted horizontally};}
						
	\onslide<8->{\draw (1.5,-1) rectangle (3.5,1); %  node[below=2] 
								
		\node[inner sep=0pt] (russell) at (2.5,0)
		{\includegraphics[scale=1.5]{images/two_blurred.jpg}};
		\node at (2.5,-1.3) { blurred};  }
						
	\onslide<9->{\draw (4.5,-1) rectangle (6.5,1); % node[below=2] 
		\node[inner sep=0pt] (russell) at (5.5,0)
		{\includegraphics[scale=1.5]{images/two_colored.jpg}};
	\node at (5.2,-1.3) {changed some pixels}; }
\end{tikzpicture}
				\onslide<11->{[augmented data = created using some knowledge of the task]}
			\end{overlayarea}}
	\end{columns}    
\end{frame}
		
\begin{frame}
	\vspace{4em}    
	\begin{overlayarea}{\textwidth}{\textheight}
						
		\begin{itemize}
			\item <1-> Typically, More data = better learning
			\item <2->  Works well for image classification / object recognition tasks
			\item <3->  Also shown to work well for speech
			\item <4->  For some tasks it may not be clear how to generate such data
		\end{itemize}
						
	\end{overlayarea}
\end{frame}
