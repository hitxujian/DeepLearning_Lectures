\PassOptionsToClass{}{beamer}
\documentclass[serif,aspectratio=169,dvipsnames]{beamer}
\usepackage[utf8]{inputenc}

%\StartShownPreambleCommands
\usepackage{amsmath,esint}
\usepackage{amssymb}
\usepackage[british]{babel}
\usepackage{multicol}
\usetheme{Warsaw}
\usecolortheme{rose}
\usepackage[normalem]{ulem}
%\usetheme{metropolis}
%\usepackage{appendixnumberbeamer}

%\StopShownPreambleCommands
\usepackage{pgfplots}

\usepackage{ mathrsfs }
\usepackage{hyperref}

\usepackage{gensymb}
\usepackage{color}
\usepackage{tkz-euclide}
\usetkzobj{all}
\usepackage{tkz-fct}  
\usetikzlibrary{calc}
\usepackage{ragged2e}
\usepackage[ruled]{algorithm2e}
\usepackage{tikz}
\usepackage{animate}
\usepackage{adjustbox}
\usepackage[labelformat=empty]{caption}
\usepackage{blindtext}
\usepackage{biblatex}


\usetikzlibrary{matrix,chains,positioning,decorations.pathreplacing,arrows} 
\usetikzlibrary{shapes.geometric}
\usetikzlibrary{intersections}


\makeatletter

\DeclareMathOperator*{\minimize}{minimize}

\newcommand\myheading[1]{%
\par\bigskip
{\Large\bfseries#1}\par\smallskip}


\addtobeamertemplate{navigation symbols}{}{%
    \usebeamerfont{footline}%
    \usebeamercolor[fg]{footline}%
    \hspace{1em}%
    \insertframenumber/\inserttotalframenumber
}

\setbeamertemplate{title page}
{
  \vspace{0.3in}
  \vbox{}
   %{\usebeamercolor[fg]{titlegraphic}\inserttitlegraphic\hfill\inserttitlegraphicii\par}
  \begin{centering}
    \begin{beamercolorbox}[sep=8pt,center]{title}
      \usebeamerfont{title}\inserttitle\par%
      \ifx\insertsubtitle\@empty%
      \else%
        \vskip0.25em%
        {\usebeamerfont{subtitle}\usebeamercolor[fg]{subtitle}\insertsubtitle\par}%
      \fi%     
    \end{beamercolorbox}%
    \vskip1em\par
    \begin{beamercolorbox}[sep=8pt,center]{date}
      \usebeamerfont{date}\insertdate
    \end{beamercolorbox}%\vskip0.5em
    \begin{beamercolorbox}[sep=8pt,center]{author}
      \usebeamerfont{author}\insertauthor
    \end{beamercolorbox}
    \begin{beamercolorbox}[sep=8pt,center]{institute}
      \usebeamerfont{institute}\insertinstitute
    \end{beamercolorbox}
  \end{centering}
  %\vfill
}
\makeatother

\author{Mitesh M. Khapra}
\title{CS7015 (Deep Learning) : Lecture 18}
\subtitle{Markov Networks}
\institute{Department of Computer Science and Engineering\\ Indian Institute of Technology Madras}
\date{}
\titlegraphic{\includegraphics[height=1cm,width=2cm]{images/iitm_logo.png}}

\begin{document}

\maketitle

\tikzstyle{input_neuron}=[circle,draw=red!50,fill=orange!10,thick,minimum size=1mm]
\tikzstyle{hidden_neuron}=[circle,draw=blue!50,fill=blue!10,thick,minimum size=10mm]
\tikzstyle{output_neuron}=[circle,draw=green!50,fill=green!20,thick,minimum size=4mm]

\tikzstyle{input}=[circle,draw=black!50,fill=black!20,thick,minimum size=.2mm]
\tikzstyle{neuron}=[circle,draw=blue!50,fill=blue!20,thick,minimum size=10mm]
\tikzstyle{neuron1}=[circle,draw=blue!50,fill=cyan!10, thick,minimum size=10mm]

\begin{frame}
\begin{block}{Acknowledgments}
	\begin{itemize}
		\item Probabilistic Graphical models: Principles and Techniques, Daphne Koller and Nir Friedman
	\end{itemize}
\end{block}
\end{frame}

\begin{frame}
	\myheading{Module 18.1: Markov Networks: Motivation}
\end{frame}

\begin{frame}
	\begin{columns}
	\column{0.5\textwidth}
	\begin{overlayarea}{\textwidth}{\textheight}
		\vspace{0.1in}
		\onslide<2->{
		\begin{center}
				\begin{tikzpicture}
				\node [input_neuron](input0) at (5.5,-0.1)  {D};
				\node [input_neuron] (input1) at (9.5,-0.1)  {B};
				\node [input_neuron] (input2) at (7.5, 1.1) {A};
				\node [input_neuron] (input3) at (7.5, -1.1) {C};

				\draw [line width=0.5mm, -] (input0) -- (input2);
				\draw [line width=0.5mm, -] (input0) -- (input3);
				\draw [line width=0.5mm, -] (input1) -- (input2);
				\draw [line width=0.5mm, -] (input1) -- (input3);
		\end{tikzpicture}
		\end{center}
		\begin{itemize}\justifying
			\item<3-> $A,B,C,D$ are four students
			\item<4-> $A$ and $B$ study together sometimes
			\item<5-> $B$ and $C$ study together sometimes
			\item<6-> $C$ and $D$ study together sometimes
			\item<7-> $A$ and $D$ study together sometimes
			\item<8-> $A$ and $C$ never study together
			\item<9-> $B$ and $D$ never study together
		\end{itemize}
		}
	\end{overlayarea}
	\column{0.5\textwidth}
	\begin{overlayarea}{\textwidth}{\textheight}
	\begin{itemize}\justifying
		\item <1-> To motivate undirected graphical models let us consider a new example
	\end{itemize}
	\end{overlayarea}
	\end{columns}
\end{frame}


\begin{frame}
	\begin{columns}
		\column{0.5\textwidth}
	\begin{overlayarea}{\textwidth}{\textheight}
		\vspace{0.1in}
		\begin{center}
				\begin{tikzpicture}
				\node [input_neuron](input0) at (5.5,-0.1)  {D};
				\node [input_neuron] (input1) at (9.5,-0.1)  {B};
				\node [input_neuron] (input2) at (7.5, 1.1) {A};
				\node [input_neuron] (input3) at (7.5, -1.1) {C};

				\draw [line width=0.5mm, -] (input0) -- (input2);
				\draw [line width=0.5mm, -] (input0) -- (input3);
				\draw [line width=0.5mm, -] (input1) -- (input2);
				\draw [line width=0.5mm, -] (input1) -- (input3);
		\end{tikzpicture}
		\end{center}
		\begin{itemize}\justifying
			\item<1-> $A,B,C,D$ are four students
			\item<1-> $A$ and $B$ study together sometimes
			\item<1-> $B$ and $C$ study together sometimes
			\item<1-> $C$ and $D$ study together sometimes
			\item<1-> $A$ and $D$ study together sometimes
			\item<1-> $A$ and $C$ never study together
			\item<1-> $B$ and $D$ never study together
		\end{itemize}
	\end{overlayarea}
		\column{0.5\textwidth}
		\begin{overlayarea}{\textwidth}{\textheight}
			\begin{itemize}\justifying
			\item<1-> To motivate undirected graphical models let us consider a new example
			\item<2-> Now suppose there was some misconception in the lecture due to some error made by the teacher
			\item<3-> Each one of A, B, C, D could have independently cleared this misconception by thinking about it after 
			the lecture
			\item<4-> In subsequent study pairs, each student could then pass on this information to their partner
			\end{itemize}
		\end{overlayarea}
	\end{columns}
\end{frame}


\begin{frame}
	\begin{columns}
		\column{0.5\textwidth}
		\begin{overlayarea}{\textwidth}{\textheight}
		\vspace{0.1in}
		\begin{center}
				\begin{tikzpicture}
				\node [input_neuron](input0) at (5.5,-0.1)  {D};
				\node [input_neuron] (input1) at (9.5,-0.1)  {B};
				\node [input_neuron] (input2) at (7.5, 1.1) {A};
				\node [input_neuron] (input3) at (7.5, -1.1) {C};

				\draw [line width=0.5mm, -] (input0) -- (input2);
				\draw [line width=0.5mm, -] (input0) -- (input3);
				\draw [line width=0.5mm, -] (input1) -- (input2);
				\draw [line width=0.5mm, -] (input1) -- (input3);
		\end{tikzpicture}
		\end{center}
		\begin{itemize}\justifying
			\item<1-> $A,B,C,D$ are four students
			\item<1-> $A$ and $B$ study together sometimes
			\item<1-> $B$ and $C$ study together sometimes
			\item<1-> $C$ and $D$ study together sometimes
			\item<1-> $A$ and $D$ study together sometimes
			\item<1-> $A$ and $C$ never study together
			\item<1-> $B$ and $D$ never study together
		\end{itemize}
		\end{overlayarea}
		\column{0.5\textwidth}
		\begin{overlayarea}{\textwidth}{\textheight}
			\begin{itemize}\justifying
			\item<1-> We are now interested in knowing whether a student still has the misconception or not
			\item<2-> Or we are interested in $P(A, B, C, D)$
			\item<3-> where A, B, C, D can take values $0$ (no misconception) or $1$ (misconception)
			\item<4-> How do we model this using a Bayesian Network ?
			\end{itemize}
		\end{overlayarea}
	\end{columns}
\end{frame}



\begin{frame}
	\begin{columns}
		\column{0.5\textwidth}
		\begin{overlayarea}{\textwidth}{\textheight}
		\vspace{0.1in}
		\begin{center}
		\begin{tikzpicture}
				\node [input_neuron](input0) at (5.5,-0.1)  {D};
				\node [input_neuron] (input1) at (9.5,-0.1)  {B};
				\node [input_neuron] (input2) at (7.5, 1.1) {A};
				\node [input_neuron] (input3) at (7.5, -1.1) {C};

				\draw [line width=0.5mm, -] (input0) -- (input2);
				\draw [line width=0.5mm, -] (input0) -- (input3);
				\draw [line width=0.5mm, -] (input1) -- (input2);
				\draw [line width=0.5mm, -] (input1) -- (input3);
		\end{tikzpicture}
		\end{center}
		\begin{itemize}\justifying
			\item<1-> $A,B,C,D$ are four students
			\item<1-> $A$ and $B$ study together sometimes
			\item<1-> $B$ and $C$ study together sometimes
			\item<1-> $C$ and $D$ study together sometimes
			\item<1-> $A$ and $D$ study together sometimes
			\item<1-> $A$ and $C$ never study together
			\item<1-> $B$ and $D$ never study together
		\end{itemize}
		\end{overlayarea}
		\column{0.5\textwidth}
		\begin{overlayarea}{\textwidth}{\textheight}
			\begin{itemize}\justifying
			\item<1-> First let us examine the conditional independencies in this problem
			\item<2-> $A \perp C | \{B,D\}$ (because A \& C never interact)
			\item<3-> $B \perp D | \{A,C\}$ (because B \& D never interact)
			\item<4-> There are no other conditional independencies in the problem
			\item<5-> Now let us try to represent this using a Bayesian Network
			\end{itemize}
		\end{overlayarea}
	\end{columns}
\end{frame}



\begin{frame}
	\begin{columns}
		\column{0.5\textwidth}
		\begin{overlayarea}{\textwidth}{\textheight}
		\vspace{0.1in}
		\begin{center}
				\begin{tikzpicture}
				\node [input_neuron](input0) at (5.5,-0.1)  {D};
				\node [input_neuron] (input1) at (9.5,-0.1)  {B};
				\node [input_neuron] (input2) at (7.5, 1.1) {A};
				\node [input_neuron] (input3) at (7.5, -1.1) {C};

				\draw [line width=0.5mm, ->] (input2) -- (input0);
				\draw [line width=0.5mm, ->] (input2) -- (input1);
				\draw [line width=0.5mm, ->] (input0) -- (input3);
				\draw [line width=0.5mm, ->] (input1) -- (input3);
		\end{tikzpicture}
		\end{center}
		\end{overlayarea}
		\column{0.5\textwidth}
		\begin{overlayarea}{\textwidth}{\textheight}
			\begin{itemize}\justifying
			\item<1-> How about this one?
			\item<2-> Indeed, it captures the following independencies relation 
			\begin{align*}
			A \perp C | \{B,D\}
			\end{align*}
			\item<3-> But, it also implies that 
			\begin{align*}
			B \not\perp D | \{A,C\}
			\end{align*}
		\end{itemize}
		\end{overlayarea}
	\end{columns}
\end{frame}

\begin{frame}
	\begin{columns}
		\column{0.5\textwidth}
		\begin{overlayarea}{\textwidth}{\textheight}
		\vspace{0.1in}
		\begin{center}
		\onslide<2->{
		\begin{tikzpicture}
				\node [input_neuron](input0) at (5.5,1.1)  {D};
				\node [input_neuron] (input1) at (9.5,1.1)  {B};
				\node [input_neuron] (input2) at (5.5,-1.1) {C};
				\node [input_neuron] (input3) at (9.5, -1.1) {A};

				\draw [line width=0.5mm, ->] (input0) -- (input2);
				\draw [line width=0.5mm, ->] (input0) -- (input3);
				\draw [line width=0.5mm, ->] (input1) -- (input2);
				\draw [line width=0.5mm, ->] (input1) -- (input3);
		\end{tikzpicture}}
		\end{center}
		\begin{itemize}\justifying
			\item<7-> \textbf{Perfect Map}: A graph $G$ is a Perfect Map for a distribution
			$P$ if the independance relations implied by the graph are exactly the same as those
			implied by the distribution
		\end{itemize}
		\end{overlayarea}
		\column{0.5\textwidth}
		\begin{overlayarea}{\textwidth}{\textheight}
			\begin{itemize}\justifying
			\item <1-> Let us try a different network
			\item <3-> Again 
			\begin{align*}
			A \perp C | \{B,D\}
			 \end{align*}
			\item <4-> But 
			\begin{align*}
			B \perp D  (\text{unconditional})
			\end{align*}
			\item <5-> You can try other networks 
			\item <6-> Turns out there is no Bayesian Network which can exactly capture independence relations that we are interested in
			\item <8-> There is no Perfect Map for the distribution 
			\end{itemize}
		\end{overlayarea}
	\end{columns}
\end{frame}

\begin{frame}
	\begin{columns}
		\column{0.5\textwidth}
		\begin{overlayarea}{\textwidth}{\textheight}
		\vspace{0.1in}
		\begin{center}
				\begin{tikzpicture}
				\node [input_neuron](input0) at (5.5,-0.1)  {D};
				\node [input_neuron] (input1) at (9.5,-0.1)  {B};
				\node [input_neuron] (input2) at (7.5, 1.1) {A};
				\node [input_neuron] (input3) at (7.5, -1.1) {C};

				\draw [line width=0.5mm, ->] (input2) -- (input0);
				\draw [line width=0.5mm, ->] (input2) -- (input1);
				\draw [line width=0.5mm, ->] (input0) -- (input3);
				\draw [line width=0.5mm, ->] (input1) -- (input3);
		\end{tikzpicture}
		\end{center}
		\end{overlayarea}
		\column{0.5\textwidth}
		\begin{overlayarea}{\textwidth}{\textheight}
			\begin{itemize}\justifying 
			\item <1->The problem is that a directed graphical model is not suitable for this example
			\item <2->A directed edge between two nodes implies some kind of direction in the interaction
			\item <3->For example $A \rightarrow B$ could indicate that $A$ influences $B$ but not the other way round
			\item <4->But in our example $A \& B$ are equal partners (they both contribute to the study discussion)
			\item <5-> We want to capture the strength of this interaction (and there is no direction here)
			\end{itemize}
		\end{overlayarea}
	\end{columns}
\end{frame}

\begin{frame}
	\begin{columns}
		\column{0.5\textwidth}
		\begin{overlayarea}{\textwidth}{\textheight}
		\vspace{0.1in}
		\begin{center}
				\begin{tikzpicture}
				\node [input_neuron](input0) at (5.5,-0.1)  {D};
				\node [input_neuron] (input1) at (9.5,-0.1)  {B};
				\node [input_neuron] (input2) at (7.5, 1.1) {A};
				\node [input_neuron] (input3) at (7.5, -1.1) {C};

				\draw [line width=0.5mm, -] (input2) -- (input0);
				\draw [line width=0.5mm, -] (input2) -- (input1);
				\draw [line width=0.5mm, -] (input0) -- (input3);
				\draw [line width=0.5mm, -] (input1) -- (input3);
		\end{tikzpicture}
		\end{center}
		\end{overlayarea}
		\column{0.5\textwidth}
		\begin{overlayarea}{\textwidth}{\textheight}
			\begin{itemize}\justifying
			\item<1-> We move on from Directed Graphical Models to Undirected Graphical Models
			\item<2-> Also known as \textbf{Markov Network}
			\item<3-> The Markov Network on the left exactly captures the interactions inherent in the problem 
			\item<4-> But how do we parameterize this graph?
			\end{itemize}
		\end{overlayarea}
	\end{columns}
\end{frame}

\begin{frame}
	\myheading{Module 18.2: Factors in Markov Network}
\end{frame}

\begin{frame}
	\begin{columns}
		\column{0.5\textwidth}
		\begin{overlayarea}{\textwidth}{\textheight}
		\vspace{0.1in}
		\begin{center}
                \tikzstyle{input_neuron}=[ellipse,draw=red!50,fill=orange!10,thick,scale=0.7]
                \begin{tikzpicture}
                    \node [input_neuron](input0) at (7,-0.1)  {Grade};
                    \node [input_neuron] (input1) at (10,-0.1)  {SAT};
                    \node [input_neuron] (input2) at (8.5, 1.1) {Intellligence};
                    \node [input_neuron] (input3) at (7, -1.1) {Letter};
                    \node [input_neuron](input4) at (5.5,1.1)  {Difficulty};

                    \draw [line width=0.2mm, ->] (input2) -- (input0);
                    \draw [line width=0.2mm, ->] (input0) -- (input3);
                    \draw [line width=0.2mm, ->] (input2) -- (input1);
                    \draw [line width=0.2mm, ->] (input4) -- (input0);
                \end{tikzpicture}
            \end{center}
         \vspace{0.1in}
        \begin{align*}
        P(G,&S,I,L,D) = \\
        &P(I)P(D)P(G|I,D)P(S|I)P(L|G)
        \end{align*}
		\end{overlayarea}
		\column{0.5\textwidth}
		\begin{overlayarea}{\textwidth}{\textheight}
			\begin{itemize}\justifying
			\item<1-> Recall that in the directed case the factors were Conditional Probability Distributions (CPDs)
			\item<2-> Each such factor captured interaction (dependence) between the connected nodes
			\item<3-> Can we use CPDs in the undirected case also ?
			\item<4-> CPDs don't make sense in the undirected case because there is no direction and hence no 
			natural conditioning (Is $A|B$ or $B|A$?)
			\end{itemize}
		\end{overlayarea}
	\end{columns}
\end{frame}

\begin{frame}
	\begin{columns}
		\column{0.5\textwidth}
		\begin{overlayarea}{\textwidth}{\textheight}
		\vspace{0.1in}
		\begin{center}
				\begin{tikzpicture}
				\node [input_neuron](input0) at (5.5,-0.1)  {D};
				\node [input_neuron] (input1) at (9.5,-0.1)  {B};
				\node [input_neuron] (input2) at (7.5, 1.1) {A};
				\node [input_neuron] (input3) at (7.5, -1.1) {C};

				\draw [line width=0.5mm, -] (input2) -- (input0);
				\draw [line width=0.5mm, -] (input2) -- (input1);
				\draw [line width=0.5mm, -] (input0) -- (input3);
				\draw [line width=0.5mm, -] (input1) -- (input3);
		\end{tikzpicture}
		\end{center}
		\end{overlayarea}
		\column{0.5\textwidth}
		\begin{overlayarea}{\textwidth}{\textheight}
			\begin{itemize}\justifying
			\item<1-> So what should be the factors or parameters in this case
			\item<2-> \textbf{Question:} What do we want these factors to capture ?
			\item<3-> \textbf{Answer:} The affinity between connected random variables
			\item<4-> Just as in the directed case the factors captured the conditional dependence between a set of random 
			variables, here we want them to capture the affinity between them
			\end{itemize}
		\end{overlayarea}
	\end{columns}
\end{frame}

\begin{frame}
	\begin{columns}
		\column{0.5\textwidth}
		\begin{overlayarea}{\textwidth}{\textheight}
		\vspace{0.1in}
		\begin{center}
				\begin{tikzpicture}
				\node [input_neuron](input0) at (5.5,-0.1)  {D};
				\node [input_neuron] (input1) at (9.5,-0.1)  {B};
				\node [input_neuron] (input2) at (7.5, 1.1) {A};
				\node [input_neuron] (input3) at (7.5, -1.1) {C};

				\draw [line width=0.5mm, -] (input2) -- (input0);
				\draw [line width=0.5mm, -] (input2) -- (input1);
				\draw [line width=0.5mm, -] (input0) -- (input3);
				\draw [line width=0.5mm, -] (input1) -- (input3);
		\end{tikzpicture}
		\end{center}
		\end{overlayarea}
		\column{0.5\textwidth}
		\begin{overlayarea}{\textwidth}{\textheight}
			\begin{itemize}\justifying
			\item<1-> However we can borrow the intuition from the directed case.
			\item<2-> Even in the undirected case, we want each such factor to capture interactions (affinity) between connected nodes
			\item<3-> We could have factors $\phi_1(A,B)$, $\phi_2(B,C)$, $\phi_3(C,D)$, $\phi_4(D,A)$ which capture the affinity between the corresponding nodes.
			\end{itemize}
		\end{overlayarea}
	\end{columns}
\end{frame}

\begin{frame}
	\begin{columns}
		\column{0.5\textwidth}
		\begin{overlayarea}{\textwidth}{\textheight}
		\vspace{0.1in}
		\begin{center}
				\begin{tikzpicture}
				\node [input_neuron](input0) at (5.5,-0.1)  {D};
				\node [input_neuron] (input1) at (9.5,-0.1)  {B};
				\node [input_neuron] (input2) at (7.5, 1.1) {A};
				\node [input_neuron] (input3) at (7.5, -1.1) {C};

				\draw [line width=0.5mm, -] (input2) -- (input0);
				\draw [line width=0.5mm, -] (input2) -- (input1);
				\draw [line width=0.5mm, -] (input0) -- (input3);
				\draw [line width=0.5mm, -] (input1) -- (input3);
		\end{tikzpicture}
		\end{center}
		\vspace{-0.2in}
		\begin{center}
		\begin{table}
\only<1>{\scalebox{0.6}{
\begin{tabular}{|ccp{1cm}|ccp{1cm}|ccp{1cm}|ccp{1cm}|}
\hline
\multicolumn{3}{|c|}{$\phi_1(A,B)$} & \multicolumn{3}{c|}{$\phi_2(B,C)$} & \multicolumn{3}{c|}{$\phi_3(C,D)$} & \multicolumn{3}{c|}{$\phi_4(D,A)$} \\ \hline
$a^0$        & $b^0$       &        & $a^0$        & $b^0$       &        & $a^0$       & $b^0$       &        & $a^0$       & $b^0$       &        \\
$a^0$        & $b^1$       &        & $a^0$        & $b^1$       &        & $a^0$       & $b^1$       &        & $a^0$       & $b^1$       &        \\
$a^1$        & $b^0$       &        & $a^1$        & $b^0$       &        & $a^1$       & $b^0$       &        & $a^1$       & $b^0$       &        \\
$a^1$        & $b^1$       &        & $a^1$        & $b^1$       &        & $a^1$       & $b^1$       &        & $a^1$       & $b^1$       &        \\ \hline
\end{tabular}}}
\only<2->{
	\scalebox{0.6}{
\begin{tabular}{|ccp{1cm}|ccp{1cm}|ccp{1cm}|ccp{1cm}|}
\hline
\multicolumn{3}{|c|}{$\phi_1(A,B)$} & \multicolumn{3}{c|}{$\phi_2(B,C)$} & \multicolumn{3}{c|}{$\phi_3(C,D)$} & \multicolumn{3}{c|}{$\phi_4(D,A)$} \\ \hline
$a^0$       & $b^0$       & 30      & $a^0$       & $b^0$      & 100      & $a^0$      & $b^0$      & 1        & $a^0$      & $b^0$      & 100      \\
$a^0$       & $b^1$       & 5       & $a^0$       & $b^1$      & 1        & $a^0$      & $b^0$      & 100      & $a^0$      & $b^1$      & 1        \\
$a^1$       & $b^0$       & 1       & $a^1$       & $b^0$      & 1        & $a^1$      & $b^1$      & 100      & $a^1$      & $b^0$      & 1        \\
$a^1$       & $b^1$       & 10      & $a^1$       & $b^1$      & 100      & $a^1$      & $b^1$      & 1        & $a^1$      & $b^1$      & 100      \\ \hline
\end{tabular}}
}
\end{table}
\end{center}
\vspace{-0.3in}
\footnotesize
\begin{itemize}\justifying
	\item<3-> But who will give us these values ?
	\item<4-> Well now you need to learn them from data (same as in the directed case)
	\item<5-> If you have access to a lot of past interactions between $A\&B$ then you could learn these values(more on this later)
	\end{itemize}
		\end{overlayarea}
		\column{0.5\textwidth}
		\begin{overlayarea}{\textwidth}{\textheight}

			\begin{itemize}\justifying
			\item<1-> Intuitively, it makes sense to have these factors associated with each pair of connected random variables.
			\item<2-> We could now assign some values of these factors
			\item<6-> Roughly speaking $\phi_1(A, B)$ asserts that it is more likely for $A$ and $B$ to agree [$\because$ weights for $a^0 b^0, a^1 b^1 > a^0b^1,a^1b^0]$
			\item <7-> $\phi_1(A,B)$ also assigns more weight to the case when both do not have a misconception as compared to the case when both have the misconception $a^0 b^0 > a^1b^1$
			\item<8-> We could have similar assignments for the other factors
			\end{itemize}
		\end{overlayarea}
	\end{columns}
\end{frame}


\begin{frame}
	\begin{columns}
		\column{0.5\textwidth}
		\begin{overlayarea}{\textwidth}{\textheight}
		\vspace{0.1in}
		\begin{center}
				\begin{tikzpicture}
				\node [input_neuron](input0) at (5.5,-0.1)  {D};
				\node [input_neuron] (input1) at (9.5,-0.1)  {B};
				\node [input_neuron] (input2) at (7.5, 1.1) {A};
				\node [input_neuron] (input3) at (7.5, -1.1) {C};

				\draw [line width=0.5mm, -] (input2) -- (input0);
				\draw [line width=0.5mm, -] (input2) -- (input1);
				\draw [line width=0.5mm, -] (input0) -- (input3);
				\draw [line width=0.5mm, -] (input1) -- (input3);
		\end{tikzpicture}
		\vspace{-0.2in}
		\begin{table}
\scalebox{0.6}{
\begin{tabular}{|ccp{1cm}|ccp{1cm}|ccp{1cm}|ccp{1cm}|}
\hline
\multicolumn{3}{|c|}{$\phi_1(A,B)$} & \multicolumn{3}{c|}{$\phi_2(B,C)$} & \multicolumn{3}{c|}{$\phi_3(C,D)$} & \multicolumn{3}{c|}{$\phi_4(D,A)$} \\ \hline
$a^0$       & $b^0$       & 30      & $a^0$       & $b^0$      & 100      & $a^0$      & $b^0$      & 1        & $a^0$      & $b^0$      & 100      \\
$a^0$       & $b^1$       & 5       & $a^0$       & $b^1$      & 1        & $a^0$      & $b^0$      & 100      & $a^0$      & $b^1$      & 1        \\
$a^1$       & $b^0$       & 1       & $a^1$       & $b^0$      & 1        & $a^1$      & $b^1$      & 100      & $a^1$      & $b^0$      & 1        \\
$a^1$       & $a^1$       & 10      & $a^1$       & $b^1$      & 100      & $a^1$      & $b^1$      & 1        & $a^1$      & $b^1$      & 100      \\ \hline
\end{tabular}}
\end{table}
\end{center}
		\end{overlayarea}
		\column{0.5\textwidth}
		\begin{overlayarea}{\textwidth}{\textheight}
		\begin{itemize}\justifying
			\item<1-> Notice a few things
			\item<2-> These tables do not represent probability distributions
			\item<3-> They are just weights which can be interpreted as the relative likelihood of an event 
			\item<4-> For example, $a=0, b=0$ is more likely than $a=1, b=1$
		\end{itemize}
		\end{overlayarea}
	\end{columns}
	\end{frame}

\begin{frame}
	\begin{columns}
		\column{0.5\textwidth}
		\begin{overlayarea}{\textwidth}{\textheight}
		\vspace{0.1in}
		\begin{center}
				\begin{tikzpicture}
				\node [input_neuron](input0) at (5.5,-0.1)  {D};
				\node [input_neuron] (input1) at (9.5,-0.1)  {B};
				\node [input_neuron] (input2) at (7.5, 1.1) {A};
				\node [input_neuron] (input3) at (7.5, -1.1) {C};

				\draw [line width=0.5mm, -] (input2) -- (input0);
				\draw [line width=0.5mm, -] (input2) -- (input1);
				\draw [line width=0.5mm, -] (input0) -- (input3);
				\draw [line width=0.5mm, -] (input1) -- (input3);
		\end{tikzpicture}
		\vspace{-0.2in}
		\begin{table}
\scalebox{0.6}{
\begin{tabular}{|ccp{1cm}|ccp{1cm}|ccp{1cm}|ccp{1cm}|}
\hline
\multicolumn{3}{|c|}{$\phi_1(A,B)$} & \multicolumn{3}{c|}{$\phi_2(B,C)$} & \multicolumn{3}{c|}{$\phi_3(C,D)$} & \multicolumn{3}{c|}{$\phi_4(D,A)$} \\ \hline
$a^0$       & $b^0$       & 30      & $a^0$       & $b^0$      & 100      & $a^0$      & $b^0$      & 1        & $a^0$      & $b^0$      & 100      \\
$a^0$       & $b^1$       & 5       & $a^0$       & $b^1$      & 1        & $a^0$      & $b^0$      & 100      & $a^0$      & $b^1$      & 1        \\
$a^1$       & $b^0$       & 1       & $a^1$       & $b^0$      & 1        & $a^1$      & $b^1$      & 100      & $a^1$      & $b^0$      & 1        \\
$a^1$       & $a^1$       & 10      & $a^1$       & $b^1$      & 100      & $a^1$      & $b^1$      & 1        & $a^1$      & $b^1$      & 100      \\ \hline
\end{tabular}}
\end{table}
\end{center}
		\end{overlayarea}
		\column{0.5\textwidth}
		\begin{overlayarea}{\textwidth}{\textheight}
		\begin{itemize}\justifying
			\item<1-> But eventually we are interested in probability distributions
			\item<2-> In the directed case going from factors to a joint probability distribution was easy as the factors were themselves conditional probability distributions
			\item <3->We could just write the joint probability distribution as the product of the factors (without violating the axioms of probability)
			\item<4-> What do we do in this case when the factors are not probability distributions
		\end{itemize}
		\end{overlayarea}
	\end{columns}
	\end{frame}


\begin{frame}
	\begin{columns}
		\column{0.45\textwidth}
		\begin{overlayarea}{\textwidth}{\textheight}
		\begin{center}
		\begin{table}
		\scalebox{0.7}{
\begin{tabular}{|cccc|r|r|}
\hline
\multicolumn{4}{|c|}{\textit{\textbf{Assignment}}} & \multicolumn{1}{c|}{\textit{\textbf{Unnormalized}}} & \multicolumn{1}{c|}{\textit{\textbf{Normalized}}} \\ \hline
$a^0$       & $b^0$      & $c^0$      & $d^0$      & 300,000                                             & 4.17E-02                                          \\
$a^0$       & $b^0$      & $c^0$      & $d^1$      & 300,000                                             & 4.17E-02                                          \\
$a^0$       & $b^0$      & $c^1$      & $d^0$      & 300,000                                             & 4.17E-02                                          \\
$a^0$       & $b^0$      & $c^1$      & $d^1$      & 30                                                  & 4.17E-06                                          \\
$a^0$       & $b^1$      & $c^0$      & $d^0$      & 500                                                 & 6.94E-05                                          \\
$a^0$       & $b^1$      & $c^0$      & $d^1$      & 500                                                 & 6.94E-05                                          \\
$a^0$       & $b^1$      & $c^1$      & $d^0$      & 5,000,000                                           & 6.94E-01                                          \\
$a^0$       & $b^1$      & $c^1$      & $d^1$      & 500                                                 & 6.94E-05                                          \\
$a^1$       & $b^0$      & $c^0$      & $d^0$      & 100                                                 & 1.39E-05                                          \\
$a^1$       & $b^0$      & $c^0$      & $d^1$      & 1,000,000                                           & 1.39E-01                                          \\
$a^1$       & $b^0$      & $c^1$      & $d^0$      & 100                                                 & 1.39E-05                                          \\
$a^1$       & $b^0$      & $c^1$      & $d^1$      & 100                                                 & 1.39E-05                                          \\
$a^1$       & $b^1$      & $c^0$      & $d^0$      & 10                                                  & 1.39E-06                                          \\
$a^1$       & $b^1$      & $c^0$      & $d^1$      & 100,000                                             & 1.39E-02                                          \\
$a^1$       & $b^1$      & $c^1$      & $d^0$      & 100,000                                             & 1.39E-02                                          \\
$a^1$       & $b^1$      & $c^1$      & $d^1$      & 100,000                                             & 1.39E-02                                          \\ \hline
\end{tabular}}
		\end{table}
		\end{center}
		\end{overlayarea}
		\column{0.55\textwidth}
		\begin{overlayarea}{\textwidth}{\textheight}
		\begin{itemize}\justifying
			\item<1-> Well we could still write it as a product of these factors \onslide<+-> and normalize it appropriately
			\onslide<2->{
			\begin{align*}
			P(a,b,c,d) &= \\
			& \frac{1}{Z}{\phi_1(a,b)\phi_2(b,c)\phi_3(c,d)\phi_4(d,a)}
			\end{align*}}
			\onslide<3->{where 
			\begin{align*}
			Z = \sum_{a,b,c,d} \phi_1(a,b)\phi_2(b,c)\phi_3(c,d)\phi_4(d,a)
			\end{align*}}
			\item<4-> Based on the values that we had assigned to the factors we can now compute the full joint probability distribution
			\item<5-> $Z$ is called the partition function.
		\end{itemize}
		\end{overlayarea}
	\end{columns}
\end{frame}


\begin{frame}
	\begin{columns}
		\column{0.5\textwidth}
		\begin{overlayarea}{\textwidth}{\textheight}
			\vspace{1cm}
			\centering
			\onslide<2->{\tikzstyle{input_neuron}=[circle,draw=red!50,fill=orange!10,thick,scale=0.7]
			\begin{tikzpicture}
				\node [input_neuron](input0) at (6.5,-0.1)  {D};
				\node [input_neuron] (input1) at (8.5,-0.1)  {B};
				\node [input_neuron] (input2) at (7.5, 1.1) {A};
				\node [input_neuron] (input3) at (7.5, -1.1) {C};
				\node [input_neuron](input4) at (5.5,1.1)  {E};
				\node [input_neuron](input5) at (9.5,1.1)  {F};

				\draw [line width=0.2mm, -] (input0) -- (input2);
				\draw [line width=0.2mm, -] (input0) -- (input3);
				\draw [line width=0.2mm, -] (input1) -- (input2);
				\draw [line width=0.2mm, -] (input1) -- (input3);
				\draw [line width=0.2mm, -] (input0) -- (input4);
				\draw [line width=0.2mm, -] (input2) -- (input4);
				\draw [line width=0.2mm, -] (input1) -- (input5);
				\draw [line width=0.2mm, -] (input2) -- (input5);
			\end{tikzpicture}
			}
		\end{overlayarea}
		\column{0.5\textwidth}
		\begin{overlayarea}{\textwidth}{\textheight}
			\begin{itemize}\justifying
				\item<1-> Let us build on the original example by adding some more students
				\item<3-> Once again there is an edge between two students if they study together
				\item<4-> One way of interpreting these new connections is that $\{A,D,E\}$ from a study group or a clique
				\item<5-> Similarly $\{A,F,B\}$ form a study group and $\{C,D\}$ form a study group and $\{B,C\}$ form a study group
			\end{itemize}
		\end{overlayarea}
	\end{columns}
\end{frame}

\begin{frame}
	\begin{columns}
		\column{0.5\textwidth}
		\begin{overlayarea}{\textwidth}{\textheight}
			\vspace{1cm}
			\centering
			\onslide<1->{\tikzstyle{input_neuron}=[circle,draw=red!50,fill=orange!10,thick,scale=0.7]
			\begin{tikzpicture}
				\node [input_neuron](input0) at (6.5,-0.1)  {D};
				\node [input_neuron] (input1) at (8.5,-0.1)  {B};
				\node [input_neuron] (input2) at (7.5, 1.1) {A};
				\node [input_neuron] (input3) at (7.5, -1.1) {C};
				\node [input_neuron](input4) at (5.5,1.1)  {E};
				\node [input_neuron](input5) at (9.5,1.1)  {F};

				\draw [line width=0.2mm, -] (input0) -- (input2);
				\draw [line width=0.2mm, -] (input0) -- (input3);
				\draw [line width=0.2mm, -] (input1) -- (input2);
				\draw [line width=0.2mm, -] (input1) -- (input3);
				\draw [line width=0.2mm, -] (input0) -- (input4);
				\draw [line width=0.2mm, -] (input2) -- (input4);
				\draw [line width=0.2mm, -] (input1) -- (input5);
				\draw [line width=0.2mm, -] (input2) -- (input5);
			\end{tikzpicture}
			}
			\onslide<3->{\begin{align*}
				&\phi_1(A,E) \phi_2(A,F) \phi_3(B,F) \phi_4(A,B)\\
				&\phi_5(A,D) \phi_6(D,E) \phi_7(B,C) \phi_8(C,D)
			\end{align*}
			}
			\onslide<6->{
			\begin{align*}
				&\phi_1(A,E,D) \phi_2(A,F,B) \phi_3(B,C) \phi_4(C,D)
			\end{align*}
			}
		\end{overlayarea}
		\column{0.5\textwidth}
		\begin{overlayarea}{\textwidth}{\textheight}
			\begin{itemize}\justifying
				\item<1-> Now, what should the factors be?
				\item<2-> We could still have factors which capture pairwise interactions
				\item<4-> But could we do something smarter (and more efficient)
				\item<5-> Instead of having a factor for each pair of nodes why not have it for each maximal clique?
			\end{itemize}
		\end{overlayarea}
	\end{columns}
\end{frame}

\begin{frame}
	\begin{columns}
		\column{0.5\textwidth}
		\begin{overlayarea}{\textwidth}{\textheight}
			\vspace{2cm}
			\centering
			\onslide<2->{
			\tikzstyle{input_neuron}=[circle,draw=red!50,fill=orange!10, thick,scale=0.7]
			\begin{tikzpicture}
				\node [input_neuron](input0) at (6.5,-0.1)  {D};
				\node [input_neuron] (input1) at (8.5,-0.1)  {B};
				\node [input_neuron] (input2) at (7.5, 1.1) {A};
				\node [input_neuron] (input3) at (7.5, -1.1) {C};
				\node [input_neuron](input5) at (9.5,1.1)  {F};
				\node [input_neuron](input4) at (5.5,1.1)  {E};
				\node [input_neuron](input6) at (6.5,0.7)  {G};
				

				\draw [line width=0.2mm, -] (input0) -- (input2);
				\draw [line width=0.2mm, -] (input0) -- (input3);
				\draw [line width=0.2mm, -] (input1) -- (input2);
				\draw [line width=0.2mm, -] (input1) -- (input3);
				\draw [line width=0.2mm, -] (input0) -- (input4);
				\draw [line width=0.2mm, -] (input2) -- (input4);
				\draw [line width=0.2mm, -] (input1) -- (input5);
				\draw [line width=0.2mm, -] (input2) -- (input5);
				\draw [line width=0.2mm, -] (input0) -- (input6);
				\draw [line width=0.2mm, -] (input2) -- (input6);
				\draw [line width=0.2mm, -] (input4) -- (input6);
			\end{tikzpicture}
			}
		\end{overlayarea}
		\column{0.5\textwidth}
		\begin{overlayarea}{\textwidth}{\textheight}
			\begin{itemize}\justifying
				\item<1-> What if we add one more student?
				\item<3-> What will be the factors in this case?
				\item<4-> Remember, we are interested in maximal cliques
				\item<5-> So instead of having factors $\phi(EAG)$ $\phi(GAD)$ $\phi(EGD)$ we will have a single factor $\phi(AEGD)$ corresponding to the maximal clique
			\end{itemize}
		\end{overlayarea}
	\end{columns}
\end{frame}

\begin{frame}
	\begin{columns}
		\column{0.5\textwidth}
		\begin{overlayarea}{\textwidth}{0.75\textheight}
			\vspace{0.2cm}
			\begin{center}
				\tikzstyle{input_neuron}=[ellipse,draw=red!50,fill=orange!10,thick,scale=0.7]
				\begin{tikzpicture}
					\node [input_neuron](input0) at (7,-0.1)  {Grade};
					\node [input_neuron] (input1) at (10,-0.1)  {SAT};
					\node [input_neuron] (input2) at (8.5, 1.1) {Intellligence};
					\node [input_neuron] (input3) at (7, -1.1) {Letter};
					\node [input_neuron](input4) at (5.5,1.1)  {Difficulty};

					\draw [line width=0.2mm, ->] (input2) -- (input0);
					\draw [line width=0.2mm, ->] (input0) -- (input3);
					\draw [line width=0.2mm, ->] (input2) -- (input1);
					\draw [line width=0.2mm, ->] (input4) -- (input0);
				\end{tikzpicture}
			\end{center}
			\vspace{-0.4cm}
			\onslide<2->{\footnotesize{
				\begin{itemize}
					\item A distribution P factorizes over a Bayesian Network G if P can be expressed as 
					\begin{align*}
						P(X_1,\ldots,X_n) = \prod_{i=1}^n P(X_i|P_{a_{X_i}})
					\end{align*}
				\end{itemize}
			}}
		\end{overlayarea}
		\column{0.5\textwidth}
		\begin{overlayarea}{\textwidth}{0.75\textheight}
			\onslide<3->{
			\footnotesize{
				\begin{center}
					\tikzstyle{input_neuron}=[circle,draw=red!50,fill=orange!10,thick,scale=0.7]
					\begin{tikzpicture}
						\node [input_neuron](input0) at (6.5,-0.1)  {B};
						\node [input_neuron] (input1) at (8.5,-0.1)  {C};
						\node [input_neuron] (input2) at (7.5, 1.1) {A};
						\node [input_neuron] (input3) at (7.5, -1.1) {D};
						\node [input_neuron](input4) at (5.5,1.1)  {E};
						\node [input_neuron](input5) at (9.5,1.1)  {F};

						\draw [line width=0.2mm, -] (input0) -- (input2);
						\draw [line width=0.2mm, -] (input0) -- (input3);
						\draw [line width=0.2mm, -] (input1) -- (input2);
						\draw [line width=0.2mm, -] (input1) -- (input3);
						\draw [line width=0.2mm, -] (input0) -- (input4);
						\draw [line width=0.2mm, -] (input2) -- (input4);
						\draw [line width=0.2mm, -] (input1) -- (input5);
						\draw [line width=0.2mm, -] (input2) -- (input5);
					\end{tikzpicture}
				\end{center}
			\vspace{-0.4cm}
			\onslide<4->{\begin{itemize}
				 \item A distribution factorizes over a Markov Network $H$ if P can be expressed as 
				\begin{align*}
					P(X_1,\ldots,X_n) = \frac{1}{Z}\prod_{i=1}^m \phi(D_i)
				\end{align*}
				where each $D_i$ is a complete sub-graph (maximal clique) in $H$
			\end{itemize}}
			}}
		\end{overlayarea}
	\end{columns}
	\onslide<5->{
	\footnotesize{
		\begin{block}{}
		A distribution is a Gibbs distribution parametrized by a set of factors $\Phi = \{ \phi_1(D_1),\ldots,\phi_m(D_m)\}$ if it is defined as
			\vspace{-0.4cm}
			\begin{align*}
				P(X_1, \ldots,X_n) = \frac{1}{Z} \prod_{i=1}^m \phi_i(D_i)
			\end{align*}
			\vspace{-0.3cm}
		\end{block}
	}}
\end{frame}

\begin{frame}
	\myheading{Module 18.3: Local Independencies in a Markov Network}
\end{frame}

\begin{frame}
	\begin{columns}
		\column{0.5\textwidth}
		\begin{overlayarea}{\textwidth}{\textheight}
		\end{overlayarea}
		\column{0.5\textwidth}
		\begin{overlayarea}{\textwidth}{\textheight}
			\begin{itemize}[<+->]\justifying
				\item Let $U$ be the set of all random variables in our joint distribution 
				\item Let $X,Y,Z$ be some distinct subsets of $U$
				\item A distribution $P$ over these RVs would imply $X\bot Y|Z$ if and only if we can write 
				\begin{align*}
				P(X) = \phi_1(X,Z)\phi_2(Y,Z)
				\end{align*}
				\item Let us see this in the context of our original example
			\end{itemize}
		\end{overlayarea}
	\end{columns}
\end{frame}

\begin{frame}
	\begin{columns}
		\column{0.5\textwidth}
		\begin{overlayarea}{\textwidth}{\textheight}
			\begin{center}
					\tikzstyle{input_neuron}=[circle,draw=red!50,fill=orange!10,thick]
					\begin{tikzpicture}
						\node [input_neuron](input0) at (5.5,-0.1)  {D};
						\node [input_neuron] (input1) at (9.5,-0.1)  {B};
						\node [input_neuron] (input2) at (7.5, 1.1) {A};
						\node [input_neuron] (input3) at (7.5, -1.1) {C};
						
						\draw [line width=0.2mm, -] (input0) -- (input2);
						\draw [line width=0.2mm, -] (input0) -- (input3);
						\draw [line width=0.2mm, -] (input1) -- (input2);
						\draw [line width=0.2mm, -] (input1) -- (input3);
					\end{tikzpicture}
				\end{center}

		\end{overlayarea}
		\column{0.5\textwidth}
		\begin{overlayarea}{\textwidth}{\textheight}
			\begin{itemize}[<+->]\justifying
				\item In this example
				\begin{align*}
					&P(A,B,C,D) =\\
					&\frac{1}{Z} [ \phi_1(A,B) \phi_2(B,C) \phi_3(C,D) \phi_4(D,A)]
				\end{align*}
				\item We can rewrite this as 
				\begin{align*}
					&P(A,B,C,D) =\\
					&\frac{1}{Z} \underbrace{[ \phi_1(A,B) \phi_2(B,C) ]}_{\phi_5(B,\{A,C\})} \underbrace{[\phi_3(C,D) \phi_4(D,A)]}_{\phi_6(D,\{A,C\})}
				\end{align*}
				\item We can say that $B \bot D|\{A,C\}$ which is indeed true
			\end{itemize}
		\end{overlayarea}
	\end{columns}
\end{frame}

\begin{frame}
	\begin{columns}
		\column{0.5\textwidth}
		\begin{overlayarea}{\textwidth}{\textheight}
			\begin{center}
					\tikzstyle{input_neuron}=[circle,draw=red!50,fill=orange!10,thick]
					\begin{tikzpicture}
						\node [input_neuron](input0) at (5.5,-0.1)  {D};
						\node [input_neuron] (input1) at (9.5,-0.1)  {B};
						\node [input_neuron] (input2) at (7.5, 1.1) {A};
						\node [input_neuron] (input3) at (7.5, -1.1) {C};
						
						\draw [line width=0.2mm, -] (input0) -- (input2);
						\draw [line width=0.2mm, -] (input0) -- (input3);
						\draw [line width=0.2mm, -] (input1) -- (input2);
						\draw [line width=0.2mm, -] (input1) -- (input3);
					\end{tikzpicture}
				\end{center}
		\end{overlayarea}
		\column{0.5\textwidth}
		\begin{overlayarea}{\textwidth}{\textheight}
			\begin{itemize}[<+->]\justifying
				\item In this example
				\begin{align*}
					&P(A,B,C,D) = \\
					&\frac{1}{Z} [ \phi_1(A,B) \phi_2(B,C) \phi_3(C,D) \phi_4(D,A)]
				\end{align*}
				\item Alternatively we can rewrite this as 
				\begin{align*}
					&P(A,B,C,D) = \\
					&\frac{1}{Z} \underbrace{[ \phi_1(A,B) \phi_2(D,A) ]}_{\phi_5(A,\{B,D\})} \underbrace{[\phi_3(C,D) \phi_4(B,C)]}_{\phi_6(C,\{B,D\})}
				\end{align*}
				\item We can say that $A \bot C|\{B,D\}$ which is indeed true
			\end{itemize}
		\end{overlayarea}
	\end{columns}
\end{frame}

\begin{frame}
	\begin{columns}
		\column{0.5\textwidth}
		\begin{overlayarea}{\textwidth}{\textheight}
			\centering
			\vspace{1cm}\
			\tikzset{mystyle/.style={shape=circle,fill=black,scale=0.3}}
			\tikzset{neigh/.style={shape=circle,fill=blue,scale=0.5}}
			\tikzset{cent/.style={shape=circle,fill=red,scale=0.5}}
			\tikzstyle{input_neuron}=[circle,draw=red!50,fill=orange!10,thick,minimum size=3mm]
			\begin{tikzpicture}[scale=.8]
            % setup the nodes
            \foreach \x in {0,...,5}
            \foreach \y in {0,...,5}
            {
            \ifnum\x=3
                \ifnum\y=2
                    \node[cent] (\x-\y) at (\x,\y){};
                \else
                    \node[mystyle] (\x-\y) at (\x,\y){};
                \fi
            \else
                \node[mystyle] (\x-\y) at (\x,\y){};
            \fi
            }
            % circle selected nodes with letters
            \foreach \mynode/\mytext in {2-2/A,3-3/B,3-1/C,4-2/D}
            {
                \draw[neigh] (\mynode) circle (0.2cm) node {};
            }
			\draw [line width=0.2mm, -] (2.1,2) -- (2.9,2);
			\draw [line width=0.2mm, -] (3,2.9) -- (3,2.1);
			\draw [line width=0.2mm, -] (3,1.1) -- (3,1.9);
			\draw [line width=0.2mm, -] (3.9,2) -- (3.1,2);
	        \end{tikzpicture}
		\end{overlayarea}
		\column{0.5\textwidth}
		\begin{overlayarea}{\textwidth}{\textheight}
			\begin{itemize}[<+->]\justifying
				\item For a given Markov network $H$ we define Markov Blanket of a RV $X$ to be the neighbors of $X$ in $H$
				\item Analogous to the case of Bayesian Networks we can define the local independences associated with $H$ to be
				\begin{align*}
					\color{red} X \color{black} \bot (U - \{X\} - MB_H) | \color{blue}MB_H(X)\color{black}
				\end{align*}  
			\end{itemize}
		\end{overlayarea}
	\end{columns}
\end{frame}

\begin{frame}
	\begin{columns}
		\column{0.5\textwidth}
		\begin{overlayarea}{\textwidth}{0.7\textheight}
			\begin{center}
			Bayesian network
			\end{center}
			\begin{center}
				\tikzstyle{input_neuron}=[ellipse,draw=red!50,fill=orange!10,thick,scale=0.7]
				\begin{tikzpicture}
					\node [input_neuron](input0) at (7,-0.1)  {Grade};
					\node [input_neuron] (input1) at (10,-0.1)  {SAT};
					\node [input_neuron] (input2) at (8.5, 1.1) {Intellligence};
					\node [input_neuron] (input3) at (7, -1.1) {Letter};
					\node [input_neuron](input4) at (5.5,1.1)  {Difficulty};

					\draw [line width=0.2mm, ->] (input2) -- (input0);
					\draw [line width=0.2mm, ->] (input0) -- (input3);
					\draw [line width=0.2mm, ->] (input2) -- (input1);
					\draw [line width=0.2mm, ->] (input4) -- (input0);
				\end{tikzpicture}
			\end{center}
	        \vspace{0.4cm}
			Local Independencies 
			\begin{align*}
				X_i \bot NonDescendents_{X_i} | Parent_{X_i}^G
			\end{align*}

		\end{overlayarea}
		\column{0.5\textwidth}
		\begin{overlayarea}{\textwidth}{0.7\textheight}
			\begin{center}
			Markov network\\
			\vspace{0.2cm}
			\tikzset{mystyle/.style={shape=circle,fill=black,scale=0.3}}
			\tikzset{neigh/.style={shape=circle,fill=blue,scale=0.5}}
			\tikzset{cent/.style={shape=circle,fill=red,scale=0.5}}
			\tikzstyle{input_neuron}=[circle,draw=red!50,fill=orange!10,thick,minimum size=3mm]
			\onslide<2->{\begin{tikzpicture}[scale=.6]
            % setup the nodes
            \foreach \x in {0,...,5}
            \foreach \y in {0,...,5}
            {
            \ifnum\x=3
                \ifnum\y=2
                    \node[cent] (\x-\y) at (\x,\y){};
                \else
                    \node[mystyle] (\x-\y) at (\x,\y){};
                \fi
            \else
                \node[mystyle] (\x-\y) at (\x,\y){};
            \fi
            }
            % circle selected nodes with letters
            \foreach \mynode/\mytext in {2-2/A,3-3/B,3-1/C,4-2/D}
            {
                \draw[neigh] (\mynode) circle (0.2cm) node {};
            }
			\draw [line width=0.2mm, -] (2.1,2) -- (2.9,2);
			\draw [line width=0.2mm, -] (3,2.9) -- (3,2.1);
			\draw [line width=0.2mm, -] (3,1.1) -- (3,1.9);
			\draw [line width=0.2mm, -] (3.9,2) -- (3.1,2);
	        \end{tikzpicture}\\}
			\end{center}
	        \vspace{0.4cm}
	        \onslide<3->{Local Independencies 
			\begin{align*}
				X_i \bot NonNeighbors_{X_i} | Neighbors_{X_i}^G
			\end{align*}}
		\end{overlayarea}
	\end{columns}
\end{frame}


% \begin{frame}
% 	\myheading{Module 18.1: Markov Networks: Motivation}
% \end{frame}

% \begin{frame}
% 	\begin{columns}
% 	\column{0.5\textwidth}
% 	\begin{overlayarea}{\textwidth}{\textheight}
% 		\vspace{0.1in}
% 		\onslide<2->{
% 		\begin{center}
% 				\begin{tikzpicture}
% 				\node [input_neuron](input0) at (5.5,-0.1)  {D};
% 				\node [input_neuron] (input1) at (9.5,-0.1)  {B};
% 				\node [input_neuron] (input2) at (7.5, 1.1) {A};
% 				\node [input_neuron] (input3) at (7.5, -1.1) {C};

% 				\draw [line width=0.5mm, -] (input0) -- (input2);
% 				\draw [line width=0.5mm, -] (input0) -- (input3);
% 				\draw [line width=0.5mm, -] (input1) -- (input2);
% 				\draw [line width=0.5mm, -] (input1) -- (input3);
% 		\end{tikzpicture}
% 		\end{center}
% 		\begin{itemize}\justifying
% 			\item<3-> $A,B,C,D$ are four students
% 			\item<4-> $A$ and $B$ study together sometimes
% 			\item<5-> $B$ and $C$ study together sometimes
% 			\item<6-> $C$ and $D$ study together sometimes
% 			\item<7-> $A$ and $D$ study together sometimes
% 			\item<8-> $A$ and $C$ never study together
% 			\item<9-> $B$ and $D$ never study together
% 		\end{itemize}
% 		}
% 	\end{overlayarea}
% 	\column{0.5\textwidth}
% 	\begin{overlayarea}{\textwidth}{\textheight}
% 	\begin{itemize}\justifying
% 		\item <1-> To motivate undirected graphical models let us consider a new example
% 	\end{itemize}
% 	\end{overlayarea}
% 	\end{columns}
% \end{frame}


% \begin{frame}
% 	\begin{columns}
% 		\column{0.5\textwidth}
% 	\begin{overlayarea}{\textwidth}{\textheight}
% 		\vspace{0.1in}
% 		\begin{center}
% 				\begin{tikzpicture}
% 				\node [input_neuron](input0) at (5.5,-0.1)  {D};
% 				\node [input_neuron] (input1) at (9.5,-0.1)  {B};
% 				\node [input_neuron] (input2) at (7.5, 1.1) {A};
% 				\node [input_neuron] (input3) at (7.5, -1.1) {C};

% 				\draw [line width=0.5mm, -] (input0) -- (input2);
% 				\draw [line width=0.5mm, -] (input0) -- (input3);
% 				\draw [line width=0.5mm, -] (input1) -- (input2);
% 				\draw [line width=0.5mm, -] (input1) -- (input3);
% 		\end{tikzpicture}
% 		\end{center}
% 		\begin{itemize}\justifying
% 			\item<1-> $A,B,C,D$ are four students
% 			\item<1-> $A$ and $B$ study together sometimes
% 			\item<1-> $B$ and $C$ study together sometimes
% 			\item<1-> $C$ and $D$ study together sometimes
% 			\item<1-> $A$ and $D$ study together sometimes
% 			\item<1-> $A$ and $C$ never study together
% 			\item<1-> $B$ and $D$ never study together
% 		\end{itemize}
% 	\end{overlayarea}
% 		\column{0.5\textwidth}
% 		\begin{overlayarea}{\textwidth}{\textheight}
% 			\begin{itemize}\justifying
% 			\item<1-> To motivate undirected graphical models let us consider a new example
% 			\item<2-> Now suppose there was some misconception in the lecture due to some error made by the teacher
% 			\item<3-> Each one of A, B, C, D could have independently cleared this misconception by thinking about it after 
% 			the lecture
% 			\item<4-> In subsequent study pairs, each student could then pass on this information to their partner
% 			\end{itemize}
% 		\end{overlayarea}
% 	\end{columns}
% \end{frame}


% \begin{frame}
% 	\begin{columns}
% 		\column{0.5\textwidth}
% 		\begin{overlayarea}{\textwidth}{\textheight}
% 		\vspace{0.1in}
% 		\begin{center}
% 				\begin{tikzpicture}
% 				\node [input_neuron](input0) at (5.5,-0.1)  {D};
% 				\node [input_neuron] (input1) at (9.5,-0.1)  {B};
% 				\node [input_neuron] (input2) at (7.5, 1.1) {A};
% 				\node [input_neuron] (input3) at (7.5, -1.1) {C};

% 				\draw [line width=0.5mm, -] (input0) -- (input2);
% 				\draw [line width=0.5mm, -] (input0) -- (input3);
% 				\draw [line width=0.5mm, -] (input1) -- (input2);
% 				\draw [line width=0.5mm, -] (input1) -- (input3);
% 		\end{tikzpicture}
% 		\end{center}
% 		\begin{itemize}\justifying
% 			\item<1-> $A,B,C,D$ are four students
% 			\item<1-> $A$ and $B$ study together sometimes
% 			\item<1-> $B$ and $C$ study together sometimes
% 			\item<1-> $C$ and $D$ study together sometimes
% 			\item<1-> $A$ and $D$ study together sometimes
% 			\item<1-> $A$ and $C$ never study together
% 			\item<1-> $B$ and $D$ never study together
% 		\end{itemize}
% 		\end{overlayarea}
% 		\column{0.5\textwidth}
% 		\begin{overlayarea}{\textwidth}{\textheight}
% 			\begin{itemize}\justifying
% 			\item<1-> We are now interested in knowing whether a student still has the misconception or not
% 			\item<2-> Or we are interested in $P(A, B, C, D)$
% 			\item<3-> where A, B, C, D can take values $0$ (no misconception) or $1$ (misconception)
% 			\item<4-> How do we model this using a Bayesian Network ?
% 			\end{itemize}
% 		\end{overlayarea}
% 	\end{columns}
% \end{frame}



% \begin{frame}
% 	\begin{columns}
% 		\column{0.5\textwidth}
% 		\begin{overlayarea}{\textwidth}{\textheight}
% 		\vspace{0.1in}
% 		\begin{center}
% 		\begin{tikzpicture}
% 				\node [input_neuron](input0) at (5.5,-0.1)  {D};
% 				\node [input_neuron] (input1) at (9.5,-0.1)  {B};
% 				\node [input_neuron] (input2) at (7.5, 1.1) {A};
% 				\node [input_neuron] (input3) at (7.5, -1.1) {C};

% 				\draw [line width=0.5mm, -] (input0) -- (input2);
% 				\draw [line width=0.5mm, -] (input0) -- (input3);
% 				\draw [line width=0.5mm, -] (input1) -- (input2);
% 				\draw [line width=0.5mm, -] (input1) -- (input3);
% 		\end{tikzpicture}
% 		\end{center}
% 		\begin{itemize}\justifying
% 			\item<1-> $A,B,C,D$ are four students
% 			\item<1-> $A$ and $B$ study together sometimes
% 			\item<1-> $B$ and $C$ study together sometimes
% 			\item<1-> $C$ and $D$ study together sometimes
% 			\item<1-> $A$ and $D$ study together sometimes
% 			\item<1-> $A$ and $C$ never study together
% 			\item<1-> $B$ and $D$ never study together
% 		\end{itemize}
% 		\end{overlayarea}
% 		\column{0.5\textwidth}
% 		\begin{overlayarea}{\textwidth}{\textheight}
% 			\begin{itemize}\justifying
% 			\item<1-> First let us examine the conditional independencies in this problem
% 			\item<2-> $A \perp C | \{B,D\}$ (because A \& C never interact)
% 			\item<3-> $B \perp D | \{A,C\}$ (because B \& D never interact)
% 			\item<4-> There are no other conditional independencies in the problem
% 			\item<5-> Now let us try to represent this using a Bayesian Network
% 			\end{itemize}
% 		\end{overlayarea}
% 	\end{columns}
% \end{frame}



% \begin{frame}
% 	\begin{columns}
% 		\column{0.5\textwidth}
% 		\begin{overlayarea}{\textwidth}{\textheight}
% 		\vspace{0.1in}
% 		\begin{center}
% 				\begin{tikzpicture}
% 				\node [input_neuron](input0) at (5.5,-0.1)  {D};
% 				\node [input_neuron] (input1) at (9.5,-0.1)  {B};
% 				\node [input_neuron] (input2) at (7.5, 1.1) {A};
% 				\node [input_neuron] (input3) at (7.5, -1.1) {C};

% 				\draw [line width=0.5mm, ->] (input2) -- (input0);
% 				\draw [line width=0.5mm, ->] (input2) -- (input1);
% 				\draw [line width=0.5mm, ->] (input0) -- (input3);
% 				\draw [line width=0.5mm, ->] (input1) -- (input3);
% 		\end{tikzpicture}
% 		\end{center}
% 		\end{overlayarea}
% 		\column{0.5\textwidth}
% 		\begin{overlayarea}{\textwidth}{\textheight}
% 			\begin{itemize}\justifying
% 			\item<1-> How about this one?
% 			\item<2-> Indeed, it captures the following independencies relation 
% 			\begin{align*}
% 			A \perp C | \{B,D\}
% 			\end{align*}
% 			\item<3-> But, it also implies that 
% 			\begin{align*}
% 			B \not\perp D | \{A,C\}
% 			\end{align*}
% 		\end{itemize}
% 		\end{overlayarea}
% 	\end{columns}
% \end{frame}

% \begin{frame}
% 	\begin{columns}
% 		\column{0.5\textwidth}
% 		\begin{overlayarea}{\textwidth}{\textheight}
% 		\vspace{0.1in}
% 		\begin{center}
% 		\onslide<2->{
% 		\begin{tikzpicture}
% 				\node [input_neuron](input0) at (5.5,1.1)  {D};
% 				\node [input_neuron] (input1) at (9.5,1.1)  {B};
% 				\node [input_neuron] (input2) at (5.5,-1.1) {C};
% 				\node [input_neuron] (input3) at (9.5, -1.1) {A};

% 				\draw [line width=0.5mm, ->] (input0) -- (input2);
% 				\draw [line width=0.5mm, ->] (input0) -- (input3);
% 				\draw [line width=0.5mm, ->] (input1) -- (input2);
% 				\draw [line width=0.5mm, ->] (input1) -- (input3);
% 		\end{tikzpicture}}
% 		\end{center}
% 		\begin{itemize}\justifying
% 			\item<7-> \textbf{Perfect Map}: A graph $G$ is a Perfect Map for a distribution
% 			$P$ if the independance relations implied by the graph are exactly the same as those
% 			implied by the distribution
% 		\end{itemize}
% 		\end{overlayarea}
% 		\column{0.5\textwidth}
% 		\begin{overlayarea}{\textwidth}{\textheight}
% 			\begin{itemize}\justifying
% 			\item <1-> Let us try a different network
% 			\item <3-> Again 
% 			\begin{align*}
% 			A \perp C | \{B,D\}
% 			 \end{align*}
% 			\item <4-> But 
% 			\begin{align*}
% 			B \perp D  (\text{unconditional})
% 			\end{align*}
% 			\item <5-> You can try other networks 
% 			\item <6-> Turns out there is no Bayesian Network which can exactly capture independence relations that we are interested in
% 			\item <8-> There is no Perfect Map for the distribution 
% 			\end{itemize}
% 		\end{overlayarea}
% 	\end{columns}
% \end{frame}

% \begin{frame}
% 	\begin{columns}
% 		\column{0.5\textwidth}
% 		\begin{overlayarea}{\textwidth}{\textheight}
% 		\vspace{0.1in}
% 		\begin{center}
% 				\begin{tikzpicture}
% 				\node [input_neuron](input0) at (5.5,-0.1)  {D};
% 				\node [input_neuron] (input1) at (9.5,-0.1)  {B};
% 				\node [input_neuron] (input2) at (7.5, 1.1) {A};
% 				\node [input_neuron] (input3) at (7.5, -1.1) {C};

% 				\draw [line width=0.5mm, ->] (input2) -- (input0);
% 				\draw [line width=0.5mm, ->] (input2) -- (input1);
% 				\draw [line width=0.5mm, ->] (input0) -- (input3);
% 				\draw [line width=0.5mm, ->] (input1) -- (input3);
% 		\end{tikzpicture}
% 		\end{center}
% 		\end{overlayarea}
% 		\column{0.5\textwidth}
% 		\begin{overlayarea}{\textwidth}{\textheight}
% 			\begin{itemize}\justifying 
% 			\item <1->The problem is that a directed graphical model is not suitable for this example
% 			\item <2->A directed edge between two nodes implies some kind of direction in the interaction
% 			\item <3->For example $A \rightarrow B$ could indicate that $A$ influences $B$ but not the other way round
% 			\item <4->But in our example $A \& B$ are equal partners (they both contribute to the study discussion)
% 			\item <5-> We want to capture the strength of this interaction (and there is no direction here)
% 			\end{itemize}
% 		\end{overlayarea}
% 	\end{columns}
% \end{frame}

% \begin{frame}
% 	\begin{columns}
% 		\column{0.5\textwidth}
% 		\begin{overlayarea}{\textwidth}{\textheight}
% 		\vspace{0.1in}
% 		\begin{center}
% 				\begin{tikzpicture}
% 				\node [input_neuron](input0) at (5.5,-0.1)  {D};
% 				\node [input_neuron] (input1) at (9.5,-0.1)  {B};
% 				\node [input_neuron] (input2) at (7.5, 1.1) {A};
% 				\node [input_neuron] (input3) at (7.5, -1.1) {C};

% 				\draw [line width=0.5mm, -] (input2) -- (input0);
% 				\draw [line width=0.5mm, -] (input2) -- (input1);
% 				\draw [line width=0.5mm, -] (input0) -- (input3);
% 				\draw [line width=0.5mm, -] (input1) -- (input3);
% 		\end{tikzpicture}
% 		\end{center}
% 		\end{overlayarea}
% 		\column{0.5\textwidth}
% 		\begin{overlayarea}{\textwidth}{\textheight}
% 			\begin{itemize}\justifying
% 			\item<1-> We move on from Directed Graphical Models to Undirected Graphical Models
% 			\item<2-> Also known as \textbf{Markov Network}
% 			\item<3-> The Markov Network on the left exactly captures the interactions inherent in the problem 
% 			\item<4-> But how do we parameterize this graph?
% 			\end{itemize}
% 		\end{overlayarea}
% 	\end{columns}
% \end{frame}

% \begin{frame}
% 	\myheading{Module 18.2: Factors in Markov Network}
% \end{frame}

% \begin{frame}
% 	\begin{columns}
% 		\column{0.5\textwidth}
% 		\begin{overlayarea}{\textwidth}{\textheight}
% 		\vspace{0.1in}
% 		\begin{center}
%                 \tikzstyle{input_neuron}=[ellipse,draw=red!50,fill=orange!10,thick,scale=0.7]
%                 \begin{tikzpicture}
%                     \node [input_neuron](input0) at (7,-0.1)  {Grade};
%                     \node [input_neuron] (input1) at (10,-0.1)  {SAT};
%                     \node [input_neuron] (input2) at (8.5, 1.1) {Intellligence};
%                     \node [input_neuron] (input3) at (7, -1.1) {Letter};
%                     \node [input_neuron](input4) at (5.5,1.1)  {Difficulty};

%                     \draw [line width=0.2mm, ->] (input2) -- (input0);
%                     \draw [line width=0.2mm, ->] (input0) -- (input3);
%                     \draw [line width=0.2mm, ->] (input2) -- (input1);
%                     \draw [line width=0.2mm, ->] (input4) -- (input0);
%                 \end{tikzpicture}
%             \end{center}
%          \vspace{0.1in}
%         \begin{align*}
%         P(G,&S,I,L,D) = \\
%         &P(I)P(D)P(G|I,D)P(S|I)P(L|G)
%         \end{align*}
% 		\end{overlayarea}
% 		\column{0.5\textwidth}
% 		\begin{overlayarea}{\textwidth}{\textheight}
% 			\begin{itemize}\justifying
% 			\item<1-> Recall that in the directed case the factors were Conditional Probability Distributions (CPDs)
% 			\item<2-> Each such factor captured interaction (dependence) between the connected nodes
% 			\item<3-> Can we use CPDs in the undirected case also ?
% 			\item<4-> CPDs don't make sense in the undirected case because there is no direction and hence no 
% 			natural conditioning (Is $A|B$ or $B|A$?)
% 			\end{itemize}
% 		\end{overlayarea}
% 	\end{columns}
% \end{frame}

% \begin{frame}
% 	\begin{columns}
% 		\column{0.5\textwidth}
% 		\begin{overlayarea}{\textwidth}{\textheight}
% 		\vspace{0.1in}
% 		\begin{center}
% 				\begin{tikzpicture}
% 				\node [input_neuron](input0) at (5.5,-0.1)  {D};
% 				\node [input_neuron] (input1) at (9.5,-0.1)  {B};
% 				\node [input_neuron] (input2) at (7.5, 1.1) {A};
% 				\node [input_neuron] (input3) at (7.5, -1.1) {C};

% 				\draw [line width=0.5mm, -] (input2) -- (input0);
% 				\draw [line width=0.5mm, -] (input2) -- (input1);
% 				\draw [line width=0.5mm, -] (input0) -- (input3);
% 				\draw [line width=0.5mm, -] (input1) -- (input3);
% 		\end{tikzpicture}
% 		\end{center}
% 		\end{overlayarea}
% 		\column{0.5\textwidth}
% 		\begin{overlayarea}{\textwidth}{\textheight}
% 			\begin{itemize}\justifying
% 			\item<1-> So what should be the factors or parameters in this case
% 			\item<2-> \textbf{Question:} What do we want these factors to capture ?
% 			\item<3-> \textbf{Answer:} The affinity between connected random variables
% 			\item<4-> Just as in the directed case the factors captured the conditional dependence between a set of random 
% 			variables, here we want them to capture the affinity between them
% 			\end{itemize}
% 		\end{overlayarea}
% 	\end{columns}
% \end{frame}

% \begin{frame}
% 	\begin{columns}
% 		\column{0.5\textwidth}
% 		\begin{overlayarea}{\textwidth}{\textheight}
% 		\vspace{0.1in}
% 		\begin{center}
% 				\begin{tikzpicture}
% 				\node [input_neuron](input0) at (5.5,-0.1)  {D};
% 				\node [input_neuron] (input1) at (9.5,-0.1)  {B};
% 				\node [input_neuron] (input2) at (7.5, 1.1) {A};
% 				\node [input_neuron] (input3) at (7.5, -1.1) {C};

% 				\draw [line width=0.5mm, -] (input2) -- (input0);
% 				\draw [line width=0.5mm, -] (input2) -- (input1);
% 				\draw [line width=0.5mm, -] (input0) -- (input3);
% 				\draw [line width=0.5mm, -] (input1) -- (input3);
% 		\end{tikzpicture}
% 		\end{center}
% 		\end{overlayarea}
% 		\column{0.5\textwidth}
% 		\begin{overlayarea}{\textwidth}{\textheight}
% 			\begin{itemize}\justifying
% 			\item<1-> However we can borrow the intuition from the directed case.
% 			\item<2-> Even in the undirected case, we want each such factor to capture interactions (affinity) between connected nodes
% 			\item<3-> We could have factors $\phi_1(A,B)$, $\phi_2(B,C)$, $\phi_3(C,D)$, $\phi_4(D,A)$ which capture the affinity between the corresponding nodes.
% 			\end{itemize}
% 		\end{overlayarea}
% 	\end{columns}
% \end{frame}

% \begin{frame}
% 	\begin{columns}
% 		\column{0.5\textwidth}
% 		\begin{overlayarea}{\textwidth}{\textheight}
% 		\vspace{0.1in}
% 		\begin{center}
% 				\begin{tikzpicture}
% 				\node [input_neuron](input0) at (5.5,-0.1)  {D};
% 				\node [input_neuron] (input1) at (9.5,-0.1)  {B};
% 				\node [input_neuron] (input2) at (7.5, 1.1) {A};
% 				\node [input_neuron] (input3) at (7.5, -1.1) {C};

% 				\draw [line width=0.5mm, -] (input2) -- (input0);
% 				\draw [line width=0.5mm, -] (input2) -- (input1);
% 				\draw [line width=0.5mm, -] (input0) -- (input3);
% 				\draw [line width=0.5mm, -] (input1) -- (input3);
% 		\end{tikzpicture}
% 		\end{center}
% 		\vspace{-0.2in}
% 		\begin{center}
% 		\begin{table}
% \only<1>{\scalebox{0.6}{
% \begin{tabular}{|ccp{1cm}|ccp{1cm}|ccp{1cm}|ccp{1cm}|}
% \hline
% \multicolumn{3}{|c|}{$\phi_1(A,B)$} & \multicolumn{3}{c|}{$\phi_2(B,C)$} & \multicolumn{3}{c|}{$\phi_3(C,D)$} & \multicolumn{3}{c|}{$\phi_4(D,A)$} \\ \hline
% $a^0$        & $b^0$       &        & $a^0$        & $b^0$       &        & $a^0$       & $b^0$       &        & $a^0$       & $b^0$       &        \\
% $a^0$        & $b^1$       &        & $a^0$        & $b^1$       &        & $a^0$       & $b^1$       &        & $a^0$       & $b^1$       &        \\
% $a^1$        & $b^0$       &        & $a^1$        & $b^0$       &        & $a^1$       & $b^0$       &        & $a^1$       & $b^0$       &        \\
% $a^1$        & $b^1$       &        & $a^1$        & $b^1$       &        & $a^1$       & $b^1$       &        & $a^1$       & $b^1$       &        \\ \hline
% \end{tabular}}}
% \only<2->{
% 	\scalebox{0.6}{
% \begin{tabular}{|ccp{1cm}|ccp{1cm}|ccp{1cm}|ccp{1cm}|}
% \hline
% \multicolumn{3}{|c|}{$\phi_1(A,B)$} & \multicolumn{3}{c|}{$\phi_2(B,C)$} & \multicolumn{3}{c|}{$\phi_3(C,D)$} & \multicolumn{3}{c|}{$\phi_4(D,A)$} \\ \hline
% $a^0$       & $b^0$       & 30      & $a^0$       & $b^0$      & 100      & $a^0$      & $b^0$      & 1        & $a^0$      & $b^0$      & 100      \\
% $a^0$       & $b^1$       & 5       & $a^0$       & $b^1$      & 1        & $a^0$      & $b^0$      & 100      & $a^0$      & $b^1$      & 1        \\
% $a^1$       & $b^0$       & 1       & $a^1$       & $b^0$      & 1        & $a^1$      & $b^1$      & 100      & $a^1$      & $b^0$      & 1        \\
% $a^1$       & $a^1$       & 10      & $a^1$       & $b^1$      & 100      & $a^1$      & $b^1$      & 1        & $a^1$      & $b^1$      & 100      \\ \hline
% \end{tabular}}
% }
% \end{table}
% \end{center}
% \vspace{-0.3in}
% \footnotesize
% \begin{itemize}\justifying
% 	\item<3-> But who will give us these values ?
% 	\item<4-> Well now you need to learn them from data (same as in the directed case)
% 	\item<5-> If you have access to a lot of past interactions between $A\&B$ then you could learn these values(more on this later)
% 	\end{itemize}
% 		\end{overlayarea}
% 		\column{0.5\textwidth}
% 		\begin{overlayarea}{\textwidth}{\textheight}

% 			\begin{itemize}\justifying
% 			\item<1-> Intuitively, it makes sense to have these factors associated with each pair of connected random variables.
% 			\item<2-> We could now assign some values of these factors
% 			\item<6-> Roughly speaking $\phi_1(A, B)$ asserts that it is more likely for $A$ and $B$ to agree [$\because$ weights for $a^0 b^0, a^1 b^1 > a^0b^1,a^1b^0]$
% 			\item <7-> $\phi_1(A,B)$ also assigns more weight to the case when both do not have a misconception as compared to the case when both have the misconception $a^0 b^0 > a^1b^1$
% 			\item<8-> We could have similar assignments for the other factors
% 			\end{itemize}
% 		\end{overlayarea}
% 	\end{columns}
% \end{frame}


% \begin{frame}
% 	\begin{columns}
% 		\column{0.5\textwidth}
% 		\begin{overlayarea}{\textwidth}{\textheight}
% 		\vspace{0.1in}
% 		\begin{center}
% 				\begin{tikzpicture}
% 				\node [input_neuron](input0) at (5.5,-0.1)  {D};
% 				\node [input_neuron] (input1) at (9.5,-0.1)  {B};
% 				\node [input_neuron] (input2) at (7.5, 1.1) {A};
% 				\node [input_neuron] (input3) at (7.5, -1.1) {C};

% 				\draw [line width=0.5mm, -] (input2) -- (input0);
% 				\draw [line width=0.5mm, -] (input2) -- (input1);
% 				\draw [line width=0.5mm, -] (input0) -- (input3);
% 				\draw [line width=0.5mm, -] (input1) -- (input3);
% 		\end{tikzpicture}
% 		\vspace{-0.2in}
% 		\begin{table}
% \scalebox{0.6}{
% \begin{tabular}{|ccp{1cm}|ccp{1cm}|ccp{1cm}|ccp{1cm}|}
% \hline
% \multicolumn{3}{|c|}{$\phi_1(A,B)$} & \multicolumn{3}{c|}{$\phi_2(B,C)$} & \multicolumn{3}{c|}{$\phi_3(C,D)$} & \multicolumn{3}{c|}{$\phi_4(D,A)$} \\ \hline
% $a^0$       & $b^0$       & 30      & $a^0$       & $b^0$      & 100      & $a^0$      & $b^0$      & 1        & $a^0$      & $b^0$      & 100      \\
% $a^0$       & $b^1$       & 5       & $a^0$       & $b^1$      & 1        & $a^0$      & $b^0$      & 100      & $a^0$      & $b^1$      & 1        \\
% $a^1$       & $b^0$       & 1       & $a^1$       & $b^0$      & 1        & $a^1$      & $b^1$      & 100      & $a^1$      & $b^0$      & 1        \\
% $a^1$       & $a^1$       & 10      & $a^1$       & $b^1$      & 100      & $a^1$      & $b^1$      & 1        & $a^1$      & $b^1$      & 100      \\ \hline
% \end{tabular}}
% \end{table}
% \end{center}
% 		\end{overlayarea}
% 		\column{0.5\textwidth}
% 		\begin{overlayarea}{\textwidth}{\textheight}
% 		\begin{itemize}\justifying
% 			\item<1-> Notice a few things
% 			\item<2-> These tables do not represent probability distributions
% 			\item<3-> They are just weights which can be interpreted as the relative likelihood of an event 
% 			\item<4-> For example, $a=0, b=0$ is more likely than $a=1, b=1$
% 		\end{itemize}
% 		\end{overlayarea}
% 	\end{columns}
% 	\end{frame}

% \begin{frame}
% 	\begin{columns}
% 		\column{0.5\textwidth}
% 		\begin{overlayarea}{\textwidth}{\textheight}
% 		\vspace{0.1in}
% 		\begin{center}
% 				\begin{tikzpicture}
% 				\node [input_neuron](input0) at (5.5,-0.1)  {D};
% 				\node [input_neuron] (input1) at (9.5,-0.1)  {B};
% 				\node [input_neuron] (input2) at (7.5, 1.1) {A};
% 				\node [input_neuron] (input3) at (7.5, -1.1) {C};

% 				\draw [line width=0.5mm, -] (input2) -- (input0);
% 				\draw [line width=0.5mm, -] (input2) -- (input1);
% 				\draw [line width=0.5mm, -] (input0) -- (input3);
% 				\draw [line width=0.5mm, -] (input1) -- (input3);
% 		\end{tikzpicture}
% 		\vspace{-0.2in}
% 		\begin{table}
% \scalebox{0.6}{
% \begin{tabular}{|ccp{1cm}|ccp{1cm}|ccp{1cm}|ccp{1cm}|}
% \hline
% \multicolumn{3}{|c|}{$\phi_1(A,B)$} & \multicolumn{3}{c|}{$\phi_2(B,C)$} & \multicolumn{3}{c|}{$\phi_3(C,D)$} & \multicolumn{3}{c|}{$\phi_4(D,A)$} \\ \hline
% $a^0$       & $b^0$       & 30      & $a^0$       & $b^0$      & 100      & $a^0$      & $b^0$      & 1        & $a^0$      & $b^0$      & 100      \\
% $a^0$       & $b^1$       & 5       & $a^0$       & $b^1$      & 1        & $a^0$      & $b^0$      & 100      & $a^0$      & $b^1$      & 1        \\
% $a^1$       & $b^0$       & 1       & $a^1$       & $b^0$      & 1        & $a^1$      & $b^1$      & 100      & $a^1$      & $b^0$      & 1        \\
% $a^1$       & $a^1$       & 10      & $a^1$       & $b^1$      & 100      & $a^1$      & $b^1$      & 1        & $a^1$      & $b^1$      & 100      \\ \hline
% \end{tabular}}
% \end{table}
% \end{center}
% 		\end{overlayarea}
% 		\column{0.5\textwidth}
% 		\begin{overlayarea}{\textwidth}{\textheight}
% 		\begin{itemize}\justifying
% 			\item<1-> But eventually we are interested in probability distributions
% 			\item<2-> In the directed case going from factors to a joint probability distribution was easy as the factors were themselves conditional probability distributions
% 			\item <3->We could just write the joint probability distribution as the product of the factors (without violating the axioms of probability)
% 			\item<4-> What do we do in this case when the factors are not probability distributions
% 		\end{itemize}
% 		\end{overlayarea}
% 	\end{columns}
% 	\end{frame}


% \begin{frame}
% 	\begin{columns}
% 		\column{0.45\textwidth}
% 		\begin{overlayarea}{\textwidth}{\textheight}
% 		\begin{center}
% 		\begin{table}
% 		\scalebox{0.7}{
% \begin{tabular}{|cccc|r|r|}
% \hline
% \multicolumn{4}{|c|}{\textit{\textbf{Assignment}}} & \multicolumn{1}{c|}{\textit{\textbf{Unnormalized}}} & \multicolumn{1}{c|}{\textit{\textbf{Normalized}}} \\ \hline
% $a^0$       & $b^0$      & $c^0$      & $d^0$      & 300,000                                             & 4.17E-02                                          \\
% $a^0$       & $b^0$      & $c^0$      & $d^1$      & 300,000                                             & 4.17E-02                                          \\
% $a^0$       & $b^0$      & $c^1$      & $d^0$      & 300,000                                             & 4.17E-02                                          \\
% $a^0$       & $b^0$      & $c^1$      & $d^1$      & 30                                                  & 4.17E-06                                          \\
% $a^0$       & $b^1$      & $c^0$      & $d^0$      & 500                                                 & 6.94E-05                                          \\
% $a^0$       & $b^1$      & $c^0$      & $d^1$      & 500                                                 & 6.94E-05                                          \\
% $a^0$       & $b^1$      & $c^1$      & $d^0$      & 5,000,000                                           & 6.94E-01                                          \\
% $a^0$       & $b^1$      & $c^1$      & $d^1$      & 500                                                 & 6.94E-05                                          \\
% $a^1$       & $b^0$      & $c^0$      & $d^0$      & 100                                                 & 1.39E-05                                          \\
% $a^1$       & $b^0$      & $c^0$      & $d^1$      & 1,000,000                                           & 1.39E-01                                          \\
% $a^1$       & $b^0$      & $c^1$      & $d^0$      & 100                                                 & 1.39E-05                                          \\
% $a^1$       & $b^0$      & $c^1$      & $d^1$      & 100                                                 & 1.39E-05                                          \\
% $a^1$       & $b^1$      & $c^0$      & $d^0$      & 10                                                  & 1.39E-06                                          \\
% $a^1$       & $b^1$      & $c^0$      & $d^1$      & 100,000                                             & 1.39E-02                                          \\
% $a^1$       & $b^1$      & $c^1$      & $d^0$      & 100,000                                             & 1.39E-02                                          \\
% $a^1$       & $b^1$      & $c^1$      & $d^1$      & 100,000                                             & 1.39E-02                                          \\ \hline
% \end{tabular}}
% 		\end{table}
% 		\end{center}
% 		\end{overlayarea}
% 		\column{0.55\textwidth}
% 		\begin{overlayarea}{\textwidth}{\textheight}
% 		\begin{itemize}\justifying
% 			\item<1-> Well we could still write it as a product of these factors \onslide<+-> and normalize it appropriately
% 			\onslide<2->{
% 			\begin{align*}
% 			P(a,b,c,d) &= \\
% 			& \frac{1}{Z}{\phi_1(a,b)\phi_2(b,c)\phi_3(c,d)\phi_4(d,a)}
% 			\end{align*}}
% 			\onslide<3->{where 
% 			\begin{align*}
% 			Z = \sum_{a,b,c,d} \phi_1(a,b)\phi_2(b,c)\phi_3(c,d)\phi_4(d,a)
% 			\end{align*}}
% 			\item<4-> Based on the values that we had assigned to the factors we can now compute the full joint probability distribution
% 			\item<5-> $Z$ is called the partition function.
% 		\end{itemize}
% 		\end{overlayarea}
% 	\end{columns}
% \end{frame}


% \begin{frame}
% 	\begin{columns}
% 		\column{0.5\textwidth}
% 		\begin{overlayarea}{\textwidth}{\textheight}
% 			\vspace{1cm}
% 			\centering
% 			\onslide<2->{\tikzstyle{input_neuron}=[circle,draw=red!50,fill=orange!10,thick,scale=0.7]
% 			\begin{tikzpicture}
% 				\node [input_neuron](input0) at (6.5,-0.1)  {D};
% 				\node [input_neuron] (input1) at (8.5,-0.1)  {B};
% 				\node [input_neuron] (input2) at (7.5, 1.1) {A};
% 				\node [input_neuron] (input3) at (7.5, -1.1) {C};
% 				\node [input_neuron](input4) at (5.5,1.1)  {E};
% 				\node [input_neuron](input5) at (9.5,1.1)  {F};

% 				\draw [line width=0.2mm, -] (input0) -- (input2);
% 				\draw [line width=0.2mm, -] (input0) -- (input3);
% 				\draw [line width=0.2mm, -] (input1) -- (input2);
% 				\draw [line width=0.2mm, -] (input1) -- (input3);
% 				\draw [line width=0.2mm, -] (input0) -- (input4);
% 				\draw [line width=0.2mm, -] (input2) -- (input4);
% 				\draw [line width=0.2mm, -] (input1) -- (input5);
% 				\draw [line width=0.2mm, -] (input2) -- (input5);
% 			\end{tikzpicture}
% 			}
% 		\end{overlayarea}
% 		\column{0.5\textwidth}
% 		\begin{overlayarea}{\textwidth}{\textheight}
% 			\begin{itemize}\justifying
% 				\item<1-> Let us build on the original example by adding some more students
% 				\item<3-> Once again there is an edge between two students if they study together
% 				\item<4-> One way of interpreting these new connections is that $\{A,D,E\}$ from a study group or a clique
% 				\item<5-> Similarly $\{A,F,B\}$ form a study group and $\{C,D\}$ form a study group and $\{B,C\}$ form a study group
% 			\end{itemize}
% 		\end{overlayarea}
% 	\end{columns}
% \end{frame}

% \begin{frame}
% 	\begin{columns}
% 		\column{0.5\textwidth}
% 		\begin{overlayarea}{\textwidth}{\textheight}
% 			\vspace{1cm}
% 			\centering
% 			\onslide<1->{\tikzstyle{input_neuron}=[circle,draw=red!50,fill=orange!10,thick,scale=0.7]
% 			\begin{tikzpicture}
% 				\node [input_neuron](input0) at (6.5,-0.1)  {D};
% 				\node [input_neuron] (input1) at (8.5,-0.1)  {B};
% 				\node [input_neuron] (input2) at (7.5, 1.1) {A};
% 				\node [input_neuron] (input3) at (7.5, -1.1) {C};
% 				\node [input_neuron](input4) at (5.5,1.1)  {E};
% 				\node [input_neuron](input5) at (9.5,1.1)  {F};

% 				\draw [line width=0.2mm, -] (input0) -- (input2);
% 				\draw [line width=0.2mm, -] (input0) -- (input3);
% 				\draw [line width=0.2mm, -] (input1) -- (input2);
% 				\draw [line width=0.2mm, -] (input1) -- (input3);
% 				\draw [line width=0.2mm, -] (input0) -- (input4);
% 				\draw [line width=0.2mm, -] (input2) -- (input4);
% 				\draw [line width=0.2mm, -] (input1) -- (input5);
% 				\draw [line width=0.2mm, -] (input2) -- (input5);
% 			\end{tikzpicture}
% 			}
% 			\onslide<3->{\begin{align*}
% 				&\phi_1(A,E) \phi_2(A,F) \phi_3(B,F) \phi_4(A,B)\\
% 				&\phi_5(A,D) \phi_6(D,E) \phi_7(B,C) \phi_8(C,D)
% 			\end{align*}
% 			}
% 			\onslide<6->{
% 			\begin{align*}
% 				&\phi_1(A,E,D) \phi_2(A,F,B) \phi_3(B,C) \phi_4(C,D)
% 			\end{align*}
% 			}
% 		\end{overlayarea}
% 		\column{0.5\textwidth}
% 		\begin{overlayarea}{\textwidth}{\textheight}
% 			\begin{itemize}\justifying
% 				\item<1-> Now, what should the factors be?
% 				\item<2-> We could still have factors which capture pairwise interactions
% 				\item<4-> But could we do something smarter (and more efficient)
% 				\item<5-> Instead of having a factor for each pair of nodes why not have it for each maximal clique?
% 			\end{itemize}
% 		\end{overlayarea}
% 	\end{columns}
% \end{frame}

% \begin{frame}
% 	\begin{columns}
% 		\column{0.5\textwidth}
% 		\begin{overlayarea}{\textwidth}{\textheight}
% 			\vspace{2cm}
% 			\centering
% 			\onslide<2->{
% 			\tikzstyle{input_neuron}=[circle,draw=red!50,fill=orange!10, thick,scale=0.7]
% 			\begin{tikzpicture}
% 				\node [input_neuron](input0) at (6.5,-0.1)  {D};
% 				\node [input_neuron] (input1) at (8.5,-0.1)  {B};
% 				\node [input_neuron] (input2) at (7.5, 1.1) {A};
% 				\node [input_neuron] (input3) at (7.5, -1.1) {C};
% 				\node [input_neuron](input5) at (9.5,1.1)  {F};
% 				\node [input_neuron](input4) at (5.5,1.1)  {E};
% 				\node [input_neuron](input6) at (6.5,0.7)  {G};
				

% 				\draw [line width=0.2mm, -] (input0) -- (input2);
% 				\draw [line width=0.2mm, -] (input0) -- (input3);
% 				\draw [line width=0.2mm, -] (input1) -- (input2);
% 				\draw [line width=0.2mm, -] (input1) -- (input3);
% 				\draw [line width=0.2mm, -] (input0) -- (input4);
% 				\draw [line width=0.2mm, -] (input2) -- (input4);
% 				\draw [line width=0.2mm, -] (input1) -- (input5);
% 				\draw [line width=0.2mm, -] (input2) -- (input5);
% 				\draw [line width=0.2mm, -] (input0) -- (input6);
% 				\draw [line width=0.2mm, -] (input2) -- (input6);
% 				\draw [line width=0.2mm, -] (input4) -- (input6);
% 			\end{tikzpicture}
% 			}
% 		\end{overlayarea}
% 		\column{0.5\textwidth}
% 		\begin{overlayarea}{\textwidth}{\textheight}
% 			\begin{itemize}\justifying
% 				\item<1-> What if we add one more student?
% 				\item<3-> What will be the factors in this case?
% 				\item<4-> Remember, we are interested in maximal cliques
% 				\item<5-> So instead of having factors $\phi(EAG)$ $\phi(GAD)$ $\phi(EGD)$ we will have a single factor $\phi(AEGD)$ corresponding to the maximal clique
% 			\end{itemize}
% 		\end{overlayarea}
% 	\end{columns}
% \end{frame}

% \begin{frame}
% 	\begin{columns}
% 		\column{0.5\textwidth}
% 		\begin{overlayarea}{\textwidth}{0.75\textheight}
% 			\vspace{0.2cm}
% 			\begin{center}
% 				\tikzstyle{input_neuron}=[ellipse,draw=red!50,fill=orange!10,thick,scale=0.7]
% 				\begin{tikzpicture}
% 					\node [input_neuron](input0) at (7,-0.1)  {Grade};
% 					\node [input_neuron] (input1) at (10,-0.1)  {SAT};
% 					\node [input_neuron] (input2) at (8.5, 1.1) {Intellligence};
% 					\node [input_neuron] (input3) at (7, -1.1) {Letter};
% 					\node [input_neuron](input4) at (5.5,1.1)  {Difficulty};

% 					\draw [line width=0.2mm, ->] (input2) -- (input0);
% 					\draw [line width=0.2mm, ->] (input0) -- (input3);
% 					\draw [line width=0.2mm, ->] (input2) -- (input1);
% 					\draw [line width=0.2mm, ->] (input4) -- (input0);
% 				\end{tikzpicture}
% 			\end{center}
% 			\vspace{-0.4cm}
% 			\onslide<2->{\footnotesize{
% 				\begin{itemize}
% 					\item A distribution P factorizes over a Bayesian Network G if P can be expressed as 
% 					\begin{align*}
% 						P(X_1,\ldots,X_n) = \prod_{i=1}^n P(X_i|P_{a_{X_i}})
% 					\end{align*}
% 				\end{itemize}
% 			}}
% 		\end{overlayarea}
% 		\column{0.5\textwidth}
% 		\begin{overlayarea}{\textwidth}{0.75\textheight}
% 			\onslide<3->{
% 			\footnotesize{
% 				\begin{center}
% 					\tikzstyle{input_neuron}=[circle,draw=red!50,fill=orange!10,thick,scale=0.7]
% 					\begin{tikzpicture}
% 						\node [input_neuron](input0) at (6.5,-0.1)  {B};
% 						\node [input_neuron] (input1) at (8.5,-0.1)  {C};
% 						\node [input_neuron] (input2) at (7.5, 1.1) {A};
% 						\node [input_neuron] (input3) at (7.5, -1.1) {D};
% 						\node [input_neuron](input4) at (5.5,1.1)  {E};
% 						\node [input_neuron](input5) at (9.5,1.1)  {F};

% 						\draw [line width=0.2mm, -] (input0) -- (input2);
% 						\draw [line width=0.2mm, -] (input0) -- (input3);
% 						\draw [line width=0.2mm, -] (input1) -- (input2);
% 						\draw [line width=0.2mm, -] (input1) -- (input3);
% 						\draw [line width=0.2mm, -] (input0) -- (input4);
% 						\draw [line width=0.2mm, -] (input2) -- (input4);
% 						\draw [line width=0.2mm, -] (input1) -- (input5);
% 						\draw [line width=0.2mm, -] (input2) -- (input5);
% 					\end{tikzpicture}
% 				\end{center}
% 			\vspace{-0.4cm}
% 			\onslide<4->{\begin{itemize}
% 				 \item A distribution factorizes over a Markov Network $H$ if P can be expressed as 
% 				\begin{align*}
% 					P(X_1,\ldots,X_n) = \prod_{i=1}^m \phi(D_i)
% 				\end{align*}
% 				where each $D_i$ is a complete sub-graph (maximal clique) in $H$
% 			\end{itemize}}
% 			}}
% 		\end{overlayarea}
% 	\end{columns}
% 	\onslide<5->{
% 	\footnotesize{
% 		\begin{block}{}
% 		A distribution is a Gibbs distribution parametrized by a set of factors $\Phi = \{ \phi_1(D_1),\ldots,\phi_m(D_m)\}$ if it is defined as
% 			\vspace{-0.4cm}
% 			\begin{align*}
% 				P(X_1, \ldots,X_n) = \frac{1}{Z} \prod_{i=1}^m \phi_i(D_i)
% 			\end{align*}
% 			\vspace{-0.3cm}
% 		\end{block}
% 	}}
% \end{frame}

% \begin{frame}
% 	\myheading{Module 18.3: Local Independencies in a Markov Network}
% \end{frame}

% \begin{frame}
% 	\begin{columns}
% 		\column{0.5\textwidth}
% 		\begin{overlayarea}{\textwidth}{\textheight}
% 		\end{overlayarea}
% 		\column{0.5\textwidth}
% 		\begin{overlayarea}{\textwidth}{\textheight}
% 			\begin{itemize}[<+->]\justifying
% 				\item Let $U$ be the set of all random variables in our joint distribution 
% 				\item Let $X,Y,Z$ be some distinct subsets of $U$
% 				\item A distribution $P$ over these RVs would imply $X\bot Y|Z$ if and only if we can write 
% 				\begin{align*}
% 				P(X) = \phi_1(X,Z)\phi_2(Y,Z)
% 				\end{align*}
% 				\item Let us see this in the context of our original example
% 			\end{itemize}
% 		\end{overlayarea}
% 	\end{columns}
% \end{frame}

% \begin{frame}
% 	\begin{columns}
% 		\column{0.5\textwidth}
% 		\begin{overlayarea}{\textwidth}{\textheight}
% 			\begin{center}
% 					\tikzstyle{input_neuron}=[circle,draw=red!50,fill=orange!10,thick]
% 					\begin{tikzpicture}
% 						\node [input_neuron](input0) at (5.5,-0.1)  {D};
% 						\node [input_neuron] (input1) at (9.5,-0.1)  {B};
% 						\node [input_neuron] (input2) at (7.5, 1.1) {A};
% 						\node [input_neuron] (input3) at (7.5, -1.1) {C};
						
% 						\draw [line width=0.2mm, -] (input0) -- (input2);
% 						\draw [line width=0.2mm, -] (input0) -- (input3);
% 						\draw [line width=0.2mm, -] (input1) -- (input2);
% 						\draw [line width=0.2mm, -] (input1) -- (input3);
% 					\end{tikzpicture}
% 				\end{center}

% 		\end{overlayarea}
% 		\column{0.5\textwidth}
% 		\begin{overlayarea}{\textwidth}{\textheight}
% 			\begin{itemize}[<+->]\justifying
% 				\item In this example
% 				\begin{align*}
% 					&P(A,B,C,D) =\\
% 					&\frac{1}{Z} [ \phi_1(A,B) \phi_2(B,C) \phi_3(C,D) \phi_4(D,A)]
% 				\end{align*}
% 				\item We can rewrite this as 
% 				\begin{align*}
% 					&P(A,B,C,D) =\\
% 					&\frac{1}{Z} \underbrace{[ \phi_1(A,B) \phi_2(B,C) ]}_{\phi_5(B,\{A,C\})} \underbrace{[\phi_3(C,D) \phi_4(D,A)]}_{\phi_6(D,\{A,C\})}
% 				\end{align*}
% 				\item We can say that $B \bot D|\{A,C\}$ which is indeed true
% 			\end{itemize}
% 		\end{overlayarea}
% 	\end{columns}
% \end{frame}

% \begin{frame}
% 	\begin{columns}
% 		\column{0.5\textwidth}
% 		\begin{overlayarea}{\textwidth}{\textheight}
% 			\begin{center}
% 					\tikzstyle{input_neuron}=[circle,draw=red!50,fill=orange!10,thick]
% 					\begin{tikzpicture}
% 						\node [input_neuron](input0) at (5.5,-0.1)  {D};
% 						\node [input_neuron] (input1) at (9.5,-0.1)  {B};
% 						\node [input_neuron] (input2) at (7.5, 1.1) {A};
% 						\node [input_neuron] (input3) at (7.5, -1.1) {C};
						
% 						\draw [line width=0.2mm, -] (input0) -- (input2);
% 						\draw [line width=0.2mm, -] (input0) -- (input3);
% 						\draw [line width=0.2mm, -] (input1) -- (input2);
% 						\draw [line width=0.2mm, -] (input1) -- (input3);
% 					\end{tikzpicture}
% 				\end{center}
% 		\end{overlayarea}
% 		\column{0.5\textwidth}
% 		\begin{overlayarea}{\textwidth}{\textheight}
% 			\begin{itemize}[<+->]\justifying
% 				\item In this example
% 				\begin{align*}
% 					&P(A,B,C,D) = \\
% 					&\frac{1}{Z} [ \phi_1(A,B) \phi_2(B,C) \phi_3(C,D) \phi_4(D,A)]
% 				\end{align*}
% 				\item Alternatively we can rewrite this as 
% 				\begin{align*}
% 					&P(A,B,C,D) = \\
% 					&\frac{1}{Z} \underbrace{[ \phi_1(A,B) \phi_2(D,A) ]}_{\phi_5(A,\{B,D\})} \underbrace{[\phi_3(C,D) \phi_4(B,C)]}_{\phi_6(C,\{B,D\})}
% 				\end{align*}
% 				\item We can say that $A \bot C|\{B,D\}$ which is indeed true
% 			\end{itemize}
% 		\end{overlayarea}
% 	\end{columns}
% \end{frame}

% \begin{frame}
% 	\begin{columns}
% 		\column{0.5\textwidth}
% 		\begin{overlayarea}{\textwidth}{\textheight}
% 			\centering
% 			\vspace{1cm}\
% 			\tikzset{mystyle/.style={shape=circle,fill=black,scale=0.3}}
% 			\tikzset{neigh/.style={shape=circle,fill=blue,scale=0.5}}
% 			\tikzset{cent/.style={shape=circle,fill=red,scale=0.5}}
% 			\tikzstyle{input_neuron}=[circle,draw=red!50,fill=orange!10,thick,minimum size=3mm]
% 			\begin{tikzpicture}[scale=.8]
%             % setup the nodes
%             \foreach \x in {0,...,5}
%             \foreach \y in {0,...,5}
%             {
%             \ifnum\x=3
%                 \ifnum\y=2
%                     \node[cent] (\x-\y) at (\x,\y){};
%                 \else
%                     \node[mystyle] (\x-\y) at (\x,\y){};
%                 \fi
%             \else
%                 \node[mystyle] (\x-\y) at (\x,\y){};
%             \fi
%             }
%             % circle selected nodes with letters
%             \foreach \mynode/\mytext in {2-2/A,3-3/B,3-1/C,4-2/D}
%             {
%                 \draw[neigh] (\mynode) circle (0.2cm) node {};
%             }
% 			\draw [line width=0.2mm, -] (2.1,2) -- (2.9,2);
% 			\draw [line width=0.2mm, -] (3,2.9) -- (3,2.1);
% 			\draw [line width=0.2mm, -] (3,1.1) -- (3,1.9);
% 			\draw [line width=0.2mm, -] (3.9,2) -- (3.1,2);
% 	        \end{tikzpicture}
% 		\end{overlayarea}
% 		\column{0.5\textwidth}
% 		\begin{overlayarea}{\textwidth}{\textheight}
% 			\begin{itemize}[<+->]\justifying
% 				\item For a given Markov network $H$ we define Markov Blanket of a RV $X$ to be the neighbors of $X$ in $H$
% 				\item Analogous to the case of Bayesian Networks we can define the local independences associated with $H$ to be
% 				\begin{align*}
% 					\color{red} X \color{black} \bot (U - \{X\} - MB_H) | \color{blue}MB_H(X)\color{black}
% 				\end{align*}  
% 			\end{itemize}
% 		\end{overlayarea}
% 	\end{columns}
% \end{frame}

% \begin{frame}
% 	\begin{columns}
% 		\column{0.5\textwidth}
% 		\begin{overlayarea}{\textwidth}{0.7\textheight}
% 			\begin{center}
% 			Bayesian network
% 			\end{center}
% 			\begin{center}
% 				\tikzstyle{input_neuron}=[ellipse,draw=red!50,fill=orange!10,thick,scale=0.7]
% 				\begin{tikzpicture}
% 					\node [input_neuron](input0) at (7,-0.1)  {Grade};
% 					\node [input_neuron] (input1) at (10,-0.1)  {SAT};
% 					\node [input_neuron] (input2) at (8.5, 1.1) {Intellligence};
% 					\node [input_neuron] (input3) at (7, -1.1) {Letter};
% 					\node [input_neuron](input4) at (5.5,1.1)  {Difficulty};

% 					\draw [line width=0.2mm, ->] (input2) -- (input0);
% 					\draw [line width=0.2mm, ->] (input0) -- (input3);
% 					\draw [line width=0.2mm, ->] (input2) -- (input1);
% 					\draw [line width=0.2mm, ->] (input4) -- (input0);
% 				\end{tikzpicture}
% 			\end{center}
% 	        \vspace{0.4cm}
% 			Local Independencies 
% 			\begin{align*}
% 				X_i \bot NonDescendents_{X_i} | Parent_{X_i}^G
% 			\end{align*}

% 		\end{overlayarea}
% 		\column{0.5\textwidth}
% 		\begin{overlayarea}{\textwidth}{0.7\textheight}
% 			\begin{center}
% 			Markov network\\
% 			\vspace{0.2cm}
% 			\tikzset{mystyle/.style={shape=circle,fill=black,scale=0.3}}
% 			\tikzset{neigh/.style={shape=circle,fill=blue,scale=0.5}}
% 			\tikzset{cent/.style={shape=circle,fill=red,scale=0.5}}
% 			\tikzstyle{input_neuron}=[circle,draw=red!50,fill=orange!10,thick,minimum size=3mm]
% 			\onslide<2->{\begin{tikzpicture}[scale=.6]
%             % setup the nodes
%             \foreach \x in {0,...,5}
%             \foreach \y in {0,...,5}
%             {
%             \ifnum\x=3
%                 \ifnum\y=2
%                     \node[cent] (\x-\y) at (\x,\y){};
%                 \else
%                     \node[mystyle] (\x-\y) at (\x,\y){};
%                 \fi
%             \else
%                 \node[mystyle] (\x-\y) at (\x,\y){};
%             \fi
%             }
%             % circle selected nodes with letters
%             \foreach \mynode/\mytext in {2-2/A,3-3/B,3-1/C,4-2/D}
%             {
%                 \draw[neigh] (\mynode) circle (0.2cm) node {};
%             }
% 			\draw [line width=0.2mm, -] (2.1,2) -- (2.9,2);
% 			\draw [line width=0.2mm, -] (3,2.9) -- (3,2.1);
% 			\draw [line width=0.2mm, -] (3,1.1) -- (3,1.9);
% 			\draw [line width=0.2mm, -] (3.9,2) -- (3.1,2);
% 	        \end{tikzpicture}\\}
% 			\end{center}
% 	        \vspace{0.4cm}
% 	        \onslide<3->{Local Independencies 
% 			\begin{align*}
% 				X_i \bot NonNeighbors_{X_i} | Neighbors_{X_i}^G
% 			\end{align*}}
% 		\end{overlayarea}
% 	\end{columns}
% \end{frame}

\end{document}
