\begin{frame}
	\myheading{Module 10.7: Hierarchical softmax}
\end{frame}

\begin{frame}
	\begin{columns}
		\column{0.5\textwidth}
		\begin{overlayarea}{\textwidth}{\textheight}
			\begin{tikzpicture}

\node (R)[text width=\textwidth] at (0,3) {\footnotesize{\begin{table}
\begin{tabular}{|c|c|c|c|c|c|c|}
\hline
0 & 0 & 1 & ... & 0 & 0 & 0\\
\hline
\end{tabular}
\end{table}}
};
\node (H) [text width=\textwidth] at ($ (R) + (0,2) $) {\begin{table}
\begin{tabular}{l!{\color{white}\vrule}l!{\color{white}\vrule}l!{\color{white}\vrule}l!{\color{white}\vrule}l!{\color{white}\vrule}l!{\color{white}\vrule}l!{\color{white}\vrule}l!{\color{white}\vrule}l!{\color{white}\vrule}l}
\hline
\rowcolor{blue!50}
 .&.&.&.&.&.&.&.&.&.\\
\hline
\end{tabular}
\end{table}
};

\draw [fill=red!50] ($(H) + (-3.75,1.5)$) rectangle ($ (H) + (-2.25, 1.8)$);
\draw [fill=red!50] ($(H) + (-1.75,1.5)$) rectangle ($ (H) + (-0.25, 1.8)$);
\draw [fill=red!50] ($(H) + (0.25,1.5)$) rectangle ($ (H) + (1.75, 1.8)$);
\draw [fill=red!50] ($(H) + (2.25,1.5)$) rectangle ($ (H) + (3.75, 1.8)$);
\node [color=red!50] at ($ (H) + (-3,2)$) {\textit{he}};
\node[color=red!50] at ($ (H) + (-1,2)$) {\textit{sat}};
\node[color=red!50] at ($(H) + (1,2)$) {\textit{a}};
\node[color=red!50] at ($(H) + (3,2)$) {\textit{chair}};
\if 0
\node(V) [text width=\textwidth] at ($ (H) + (0,1.5) $) {\large{\begin{table}
\begin{tabular}{l!{\color{white}\vrule}l!{\color{white}\vrule}l!{\color{white}\vrule}l!{\color{white}\vrule}l!{\color{white}\vrule}l!{\color{white}\vrule}l!{\color{white}\vrule}l!{\color{white}\vrule}l!{\color{white}\vrule}l!{\color{white}\vrule}l!{\color{white}\vrule}l}
\hline
\rowcolor{red!50}
 .&.&.&.&.&.&.&.&.&.&.&.\\
\hline
\end{tabular}
\end{table}}
};
 
\node[text width=0.15\textwidth, rotate=90] at ($(V) + (-3,0.9)$ ) {\tiny{$P(he|sat)$}};
\node[text width=0.15\textwidth, rotate=90] at ($(V) + (-2.5,0.9)$ ) {\tiny{$P(chair|sat)$}};
\node[text width=0.15\textwidth, rotate=90] at ($(V) + (-2.0,0.9)$ ) {\tiny{$P(man|sat)$}};
\node[text width=0.15\textwidth, rotate=90] at ($(V) + (1.4,0.9)$ ) {\tiny\textcolor{red}{{$P(on|sat)$}}};
\fi

%\draw[line width=1][->] (0,1.6) -- (0,2.75);
\node[text width=0.3\textwidth, ] at ($(H) + (3.8,0)$ ) {\tiny{$\mathbf{h} \in \mathbb{R}^{|k|}$}};
\node[text width=0.3\textwidth, ] at ($(H) + (2.5,0.7)$ ) {\tiny{$W_{context} \in \mathbb{R}^{k\times |V|}$}};
\node[text width=0.3\textwidth, ] at ($(0,2.7) + (3.3,0.4)$ ) {\tiny{$\mathbf{x} \in \mathbb{R}^{|V|}$}};
\node[text width = 0.3\textwidth] at (0.7, 4){\tiny{$W_{word} \in \mathbb{R}^{k\times |V|}$}};
\draw[line width=0.5] (-2.15,3.15) -- (-2.7,4.7);
\draw[line width=0.5] (2.15,3.15) -- (2.7,4.7);

\draw[line width=0.5] [->](0,5.20) -- ($(H) + (-3,1.5)$);
\draw[line width=0.5] [->] (0,5.20) -- ($(H) + (3,1.5)$);
\draw[line width=0.5] [->] (0,5.20) -- ($(H) + (-1,1.5)$);
\draw[line width=0.5] [->] (0,5.20) -- ($(H) + (1,1.5)$);
\end{tikzpicture}

		\end{overlayarea}

		\column{0.5\textwidth}
		\begin{overlayarea}{\textwidth}{\textheight}
			\onslide<1->{
				\textbf{Some problems}
				\begin{itemize}\justifying
					\item Same as bag of words
					\item The softmax function at the output is computationally expensive
					\item Solution 1: Use negative sampling
					\item Solution 2: Use contrastive estimation
					\item \textcolor<1->{red}{Solution 3: Use hierarchical softmax}
				\end{itemize}
			}
		\end{overlayarea}


	\end{columns}
\end{frame}

\begin{frame}
	\begin{columns}
		\column{0.5\textwidth}
		\begin{overlayarea}{\textwidth}{\textheight}
			\include{modules/Module7/tikz_images/hier_soft}
		\end{overlayarea}
		\column{0.5\textwidth}
		\begin{overlayarea}{\textwidth}{\textheight}
			\footnotesize{
				\begin{itemize}
					\justifying
					\item<1-> Construct a binary tree such that there are $|V|$ leaf nodes each corresponding
					      to one word in the vocabulary
					\item<3-> There exists a unique path from the root node to a leaf node.
					\item<4-> Let $l(w_1),\ l(w_2),\ ...,\ l(w_p)$ be the nodes on the path from root to $w$
					\item<5-> Let $\pi(w)$ be a binary vector such that:
					      \footnotesize{\begin{align*}
							      \pi(w)_k & = 1 \quad \text{{path branches left at node }} l(w_k) \\
							               & = 0 \quad otherwise
						      \end{align*}}
					\item<6-> Finally each internal node is associated with a vector $u_i$
					\item<7-> So the parameters of the module are $\mathbf{W}_{context}$ and $u_1, u_2, \dots, u_v$ (in effect, we have the same number of parameters as before)
				\end{itemize}
			}
		\end{overlayarea}
	\end{columns}
\end{frame}


\begin{frame}
	\begin{columns}
		\column{0.5\textwidth}
		\begin{overlayarea}{\textwidth}{\textheight}
			\begin{tikzpicture}
\node (R) [text width=\textwidth] at (0,1) {\footnotesize{\begin{table}
\begin{tabular}{|c|c|c|c|c|c|c|}
\hline
0 & 1 & 0 & ... & 0 & 0 & 0\\
\hline
\end{tabular}
\caption*{\textit{sat}}
\end{table}}
};

\node (H) [text width=\textwidth] at ($ (R) + (0,2) $) {\begin{table}
\begin{tabular}{l!{\color{white}\vrule}l!{\color{white}\vrule}l!{\color{white}\vrule}l!{\color{white}\vrule}l!{\color{white}\vrule}l!{\color{white}\vrule}l!{\color{white}\vrule}l!{\color{white}\vrule}l!{\color{white}\vrule}l}
\hline
\rowcolor{blue!50}
 .&.&.&.&.&.&.&.&.&.\\
\hline
\end{tabular}
\end{table}
};

\onslide<1->{
\node(V) [text width=\textwidth] at ($ (H) + (0,1.5) $) {\large{\begin{table}
\begin{tabular}{l!{\color{white}\vrule}l!{\color{white}\vrule}l!{\color{white}\vrule}l!{\color{white}\vrule}l!{\color{white}\vrule}l!{\color{white}\vrule}l!{\color{white}\vrule}l!{\color{white}\vrule}l!{\color{white}\vrule}l!{\color{white}\vrule}l!{\color{white}\vrule}l}
\hline
\rowcolor{black!10}
 .&.&.&1&.&.&.&.&.&.&.&.\\
\hline
\end{tabular}
\end{table}}
};

\node [circle,line width=1,draw=red!40, minimum size=2mm] (A) at ($(V) + (-3.2,0.6)$ ) {};
\node [circle,line width=1,draw=red!40, minimum size=2mm] (B) at ($(V) + (-2.6,0.6)$ ) {};
\node [circle,line width=1,draw=red!40, minimum size=2mm] (C) at ($(V) + (-2.0,0.6)$ ) {};
\node [circle,line width=1,draw=red!40, minimum size=2mm] (D) at ($(V) + (-1.4,0.6)$ ) {};
\node [circle,line width=1,draw=red!40, minimum size=2mm] (W) at ($(V) + (1.4,0.6)$ ) {};
\node [circle,line width=1,draw=red!40, minimum size=2mm] (X) at ($(V) + (2.0,0.6)$ ) {};
\node [circle,line width=1,draw=red!40, minimum size=2mm] (Y) at ($(V) + (2.6,0.6)$ ) {};
\node [circle,line width=1,draw=red!40, minimum size=2mm] (Z) at ($(V) + (3.2,0.6)$ ) {};


\node [circle,line width=1,draw=blue!40, minimum size=2mm] (AB) at ($(V) + (-2.9,1.4)$ ) {};
\node [circle,line width=1,draw=blue!40, minimum size=2mm] (CD) at ($(V) + (-1.7,1.4)$ ) {};
\node [circle,line width=1,draw=blue!40, minimum size=2mm] (WX) at ($(V) + (1.7,1.4)$ ) {};
\node [circle,line width=1,draw=blue!40, minimum size=2mm] (YZ) at ($(V) + (2.9,1.4)$ ) {};

\node [circle,line width=1,draw=blue!40, minimum size=2mm] (WXYZ) at ($(V) + (2.3,2.3)$ ) {};
\node [circle,line width=1,draw=blue!40, minimum size=2mm] (ABCD) at ($(V) + (-2.3,2.3)$ ) {};

\node [circle,line width=1,draw=blue!40, minimum size=2mm] (AL) at ($(V) + (0,3.3)$ ) {};


\draw[line width=0.8] (A) -- (AB);
\draw[line width=0.8] (B) -- (AB);

\draw[line width=0.8] (C) -- (CD);
\draw[line width=0.8] (D) -- (CD);
\draw[line width=0.8] (W) -- (WX);
\draw[line width=0.8] (X) -- (WX);
\draw[line width=0.8] (Y) -- (YZ);
\draw[line width=0.8] (Z) -- (YZ);

\draw[line width=0.8] (WX) -- (WXYZ);
\draw[line width=0.8] (YZ) -- (WXYZ);

\draw[line width=0.8] (AB) -- (ABCD);
\draw[line width=0.8] (CD) -- (ABCD);

\draw[line width=0.8] (ABCD) -- (AL);
\draw[line width=0.8] (WXYZ) -- (AL);
\node[text width=0.1\textwidth, ] at ($(V) + (-0.8,0.5)$ ) {.};
\node[text width=0.1\textwidth, ] at ($(V) + (-0.2,0.5)$ ) {.};
\node[text width=0.1\textwidth, ] at ($(V) + (0.4,0.5)$ ) {.};
\node[text width=0.1\textwidth, ] at ($(V) + (1,0.5)$ ) {.};
}
\onslide<1->{\node [circle,line width=1,draw=red!40, minimum size=2mm, fill=red!60] (D) at ($(V) + (-1.4,0.6)$ ) {};
\node [circle,line width=1,draw=blue!40, minimum size=2mm, fill=blue!60] (CD) at ($(V) + (-1.7,1.4)$ ) {};
\node [circle,line width=1,draw=blue!40, minimum size=2mm,  fill=blue!60] (ABCD) at ($(V) + (-2.3,2.3)$ ) {};

\node [circle,line width=1,draw=blue!40, minimum size=2mm, fill=blue!60] (AL) at ($(V) + (0,3.3)$ ) {};


}

\onslide<1->{
	\node[text width=0.2\textwidth, ] at ($(AL) + (-0.8,0)$ ) {\textcolor{blue!50}{\tiny{$\pi(on)_1=1$}}};
	\node[text width=0.2\textwidth, ] at ($(ABCD) + (-0.8,0)$) {\textcolor{blue!50}{\tiny{$\pi(on)_2=0$}}};
	\node[text width=0.2\textwidth, ] at ($(CD) + (1,0)$ ) {\textcolor{blue!50}{\tiny{$\pi(on)_3=0$}}};
}

\onslide<1->{
	\draw [color=black!50, fill=black!40]  ($(AL) + (0.4,-0.1)$) rectangle ($(AL) + (1.1,0.1)$);
	\draw [color=black!50, fill=black!40]  ($(ABCD) + (0.4,-0.1)$) rectangle ($(ABCD) + (1.1,0.1)$);
	\draw [color=black!50,fill=black!40]  ($(YZ) + (0.4,-0.1)$) rectangle ($(YZ) + (1.1,0.1)$);
	\node[text width=0.2\textwidth, ] at ($(AL) + (1.8,0)$ ) {\textcolor{black}{\tiny{$u_1$}}};
	\node[text width=0.2\textwidth, ] at ($(ABCD) + (1.8,0)$) {\textcolor{black}{\tiny{$u_2$}}};
	\node[text width=0.2\textwidth, ] at ($(YZ) + (1.5,-0.2)$ ) {\textcolor{black}{\tiny{$u_V$}}};
}


\node[text width=0.1\textwidth, ] at ($(V) + (-1.1,-0.5)$ ) {\textcolor{red!40}{\footnotesize{\textit{on}}}};
\node[text width=0.2\textwidth, ] at ($(H) + (3.5,0)$) {\textcolor{black}{\footnotesize{$h=v_c$}}};

\draw[line width=0.8,color=blue!50] (-2.15,1.6) -- (-2.7,2.7);
\draw[line width=0.8,color=blue!50] (2.15,1.6) -- (2.7,2.7);

\draw[line width=0.8,color=red!50] (-2.67,3.21) -- (-3.29,4.21);
\draw[line width=0.8,color=red!50] (2.67,3.21) -- (3.29,4.21);
\end{tikzpicture}

		\end{overlayarea}
		\column{0.5\textwidth}
		\begin{overlayarea}{\textwidth}{\textheight}
			\onslide<1-4>{
				\footnotesize{\begin{itemize}
						\justifying
						\item<1-> For a given pair $(w,c)$ we are interested in the probability $p(w|v_c)$
						\item<2-> We model this probability as
						      \begin{align*}
							      p(w|v_c) = \prod\limits_{k}(\pi(w_k)|v_c)
						      \end{align*}
						\item <3-> For example
						      \begin{align*}
							      P(on|v_{sat}) = P(\pi(on)_1 = 1| v_{sat}) &   \\
							      * P(\pi(on)_2 = 0 | v_{sat})              &   \\
							      * P(\pi(on)_3 = 0| v_{sat})               &
						      \end{align*}
						\item<4-> In effect, we are saying that the probability of predicting a word is the same as predicting the correct unique path from the root node to that word
					\end{itemize}}}
		\end{overlayarea}
	\end{columns}
\end{frame}



\begin{frame}
	\begin{columns}
		\column{0.5\textwidth}
		\begin{overlayarea}{\textwidth}{\textheight}
			\begin{tikzpicture}
\node (R) [text width=\textwidth] at (0,1) {\footnotesize{\begin{table}
\begin{tabular}{|c|c|c|c|c|c|c|}
\hline
0 & 1 & 0 & ... & 0 & 0 & 0\\
\hline
\end{tabular}
\caption*{\textit{sat}}
\end{table}}
};

\node (H) [text width=\textwidth] at ($ (R) + (0,2) $) {\begin{table}
\begin{tabular}{l!{\color{white}\vrule}l!{\color{white}\vrule}l!{\color{white}\vrule}l!{\color{white}\vrule}l!{\color{white}\vrule}l!{\color{white}\vrule}l!{\color{white}\vrule}l!{\color{white}\vrule}l!{\color{white}\vrule}l}
\hline
\rowcolor{blue!50}
 .&.&.&.&.&.&.&.&.&.\\
\hline
\end{tabular}
\end{table}
};

\onslide<1->{
\node(V) [text width=\textwidth] at ($ (H) + (0,1.5) $) {\large{\begin{table}
\begin{tabular}{l!{\color{white}\vrule}l!{\color{white}\vrule}l!{\color{white}\vrule}l!{\color{white}\vrule}l!{\color{white}\vrule}l!{\color{white}\vrule}l!{\color{white}\vrule}l!{\color{white}\vrule}l!{\color{white}\vrule}l!{\color{white}\vrule}l!{\color{white}\vrule}l}
\hline
\rowcolor{black!10}
 .&.&.&1&.&.&.&.&.&.&.&.\\
\hline
\end{tabular}
\end{table}}
};

\node [circle,line width=1,draw=red!40, minimum size=2mm] (A) at ($(V) + (-3.2,0.6)$ ) {};
\node [circle,line width=1,draw=red!40, minimum size=2mm] (B) at ($(V) + (-2.6,0.6)$ ) {};
\node [circle,line width=1,draw=red!40, minimum size=2mm] (C) at ($(V) + (-2.0,0.6)$ ) {};
\node [circle,line width=1,draw=red!40, minimum size=2mm] (D) at ($(V) + (-1.4,0.6)$ ) {};
\node [circle,line width=1,draw=red!40, minimum size=2mm] (W) at ($(V) + (1.4,0.6)$ ) {};
\node [circle,line width=1,draw=red!40, minimum size=2mm] (X) at ($(V) + (2.0,0.6)$ ) {};
\node [circle,line width=1,draw=red!40, minimum size=2mm] (Y) at ($(V) + (2.6,0.6)$ ) {};
\node [circle,line width=1,draw=red!40, minimum size=2mm] (Z) at ($(V) + (3.2,0.6)$ ) {};


\node [circle,line width=1,draw=blue!40, minimum size=2mm] (AB) at ($(V) + (-2.9,1.4)$ ) {};
\node [circle,line width=1,draw=blue!40, minimum size=2mm] (CD) at ($(V) + (-1.7,1.4)$ ) {};
\node [circle,line width=1,draw=blue!40, minimum size=2mm] (WX) at ($(V) + (1.7,1.4)$ ) {};
\node [circle,line width=1,draw=blue!40, minimum size=2mm] (YZ) at ($(V) + (2.9,1.4)$ ) {};

\node [circle,line width=1,draw=blue!40, minimum size=2mm] (WXYZ) at ($(V) + (2.3,2.3)$ ) {};
\node [circle,line width=1,draw=blue!40, minimum size=2mm] (ABCD) at ($(V) + (-2.3,2.3)$ ) {};

\node [circle,line width=1,draw=blue!40, minimum size=2mm] (AL) at ($(V) + (0,3.3)$ ) {};


\draw[line width=0.8] (A) -- (AB);
\draw[line width=0.8] (B) -- (AB);

\draw[line width=0.8] (C) -- (CD);
\draw[line width=0.8] (D) -- (CD);
\draw[line width=0.8] (W) -- (WX);
\draw[line width=0.8] (X) -- (WX);
\draw[line width=0.8] (Y) -- (YZ);
\draw[line width=0.8] (Z) -- (YZ);

\draw[line width=0.8] (WX) -- (WXYZ);
\draw[line width=0.8] (YZ) -- (WXYZ);

\draw[line width=0.8] (AB) -- (ABCD);
\draw[line width=0.8] (CD) -- (ABCD);

\draw[line width=0.8] (ABCD) -- (AL);
\draw[line width=0.8] (WXYZ) -- (AL);
\node[text width=0.1\textwidth, ] at ($(V) + (-0.8,0.5)$ ) {.};
\node[text width=0.1\textwidth, ] at ($(V) + (-0.2,0.5)$ ) {.};
\node[text width=0.1\textwidth, ] at ($(V) + (0.4,0.5)$ ) {.};
\node[text width=0.1\textwidth, ] at ($(V) + (1,0.5)$ ) {.};
}
\onslide<1->{\node [circle,line width=1,draw=red!40, minimum size=2mm, fill=red!60] (D) at ($(V) + (-1.4,0.6)$ ) {};
\node [circle,line width=1,draw=blue!40, minimum size=2mm, fill=blue!60] (CD) at ($(V) + (-1.7,1.4)$ ) {};
\node [circle,line width=1,draw=blue!40, minimum size=2mm,  fill=blue!60] (ABCD) at ($(V) + (-2.3,2.3)$ ) {};

\node [circle,line width=1,draw=blue!40, minimum size=2mm, fill=blue!60] (AL) at ($(V) + (0,3.3)$ ) {};


}

\onslide<1->{
	\node[text width=0.2\textwidth, ] at ($(AL) + (-0.8,0)$ ) {\textcolor{blue!50}{\tiny{$\pi(on)_1=1$}}};
	\node[text width=0.2\textwidth, ] at ($(ABCD) + (-0.8,0)$) {\textcolor{blue!50}{\tiny{$\pi(on)_2=0$}}};
	\node[text width=0.2\textwidth, ] at ($(CD) + (1,0)$ ) {\textcolor{blue!50}{\tiny{$\pi(on)_3=0$}}};
}

\onslide<1->{
	\draw [color=black!50, fill=black!40]  ($(AL) + (0.4,-0.1)$) rectangle ($(AL) + (1.1,0.1)$);
	\draw [color=black!50, fill=black!40]  ($(ABCD) + (0.4,-0.1)$) rectangle ($(ABCD) + (1.1,0.1)$);
	\draw [color=black!50,fill=black!40]  ($(YZ) + (0.4,-0.1)$) rectangle ($(YZ) + (1.1,0.1)$);
	\node[text width=0.2\textwidth, ] at ($(AL) + (1.8,0)$ ) {\textcolor{black}{\tiny{$u_1$}}};
	\node[text width=0.2\textwidth, ] at ($(ABCD) + (1.8,0)$) {\textcolor{black}{\tiny{$u_2$}}};
	\node[text width=0.2\textwidth, ] at ($(YZ) + (1.5,-0.2)$ ) {\textcolor{black}{\tiny{$u_V$}}};
}


\node[text width=0.1\textwidth, ] at ($(V) + (-1.1,-0.5)$ ) {\textcolor{red!40}{\footnotesize{\textit{on}}}};
\node[text width=0.2\textwidth, ] at ($(H) + (3.5,0)$) {\textcolor{black}{\footnotesize{$h=v_c$}}};

\draw[line width=0.8,color=blue!50] (-2.15,1.6) -- (-2.7,2.7);
\draw[line width=0.8,color=blue!50] (2.15,1.6) -- (2.7,2.7);

\draw[line width=0.8,color=red!50] (-2.67,3.21) -- (-3.29,4.21);
\draw[line width=0.8,color=red!50] (2.67,3.21) -- (3.29,4.21);
\end{tikzpicture}

		\end{overlayarea}
		\column{0.5\textwidth}
		\begin{overlayarea}{\textwidth}{\textheight}
			\footnotesize{
				\begin{itemize}
					\justifying
						\item<1-> We model
						    \begin{align*}
							    P(\pi(on)_i = 1) & = \frac{1}{1 + e^{-v_c^{T}u_i}} \\
							    P(\pi(on)_i = 0) & = 1 - P(\pi(on)_i = 1)          \\
							    P(\pi(on)_i = 0) & = \frac{1}{1 + e^{v_c^{T}u_i}}
						    \end{align*}

					\item<2-> The above model ensures that the representation
					      of a context word $v_c$ will have a high(low) similarity
					      with the representation of the node $u_i$ if $u_i$ appears
					      and the path branches to the left(right) at $u_i$
					\item<3-> Again, transitively the representations of contexts
					      which appear with the same words will have high similarity
				\end{itemize}
			}
		\end{overlayarea}
	\end{columns}
\end{frame}


\begin{frame}
	\begin{columns}
		\column{0.5\textwidth}
		\begin{overlayarea}{\textwidth}{\textheight}
			\begin{tikzpicture}
\node (R) [text width=\textwidth] at (0,1) {\footnotesize{\begin{table}
\begin{tabular}{|c|c|c|c|c|c|c|}
\hline
0 & 1 & 0 & ... & 0 & 0 & 0\\
\hline
\end{tabular}
\caption*{\textit{sat}}
\end{table}}
};

\node (H) [text width=\textwidth] at ($ (R) + (0,2) $) {\begin{table}
\begin{tabular}{l!{\color{white}\vrule}l!{\color{white}\vrule}l!{\color{white}\vrule}l!{\color{white}\vrule}l!{\color{white}\vrule}l!{\color{white}\vrule}l!{\color{white}\vrule}l!{\color{white}\vrule}l!{\color{white}\vrule}l}
\hline
\rowcolor{blue!50}
 .&.&.&.&.&.&.&.&.&.\\
\hline
\end{tabular}
\end{table}
};

\onslide<1->{
\node(V) [text width=\textwidth] at ($ (H) + (0,1.5) $) {\large{\begin{table}
\begin{tabular}{l!{\color{white}\vrule}l!{\color{white}\vrule}l!{\color{white}\vrule}l!{\color{white}\vrule}l!{\color{white}\vrule}l!{\color{white}\vrule}l!{\color{white}\vrule}l!{\color{white}\vrule}l!{\color{white}\vrule}l!{\color{white}\vrule}l!{\color{white}\vrule}l}
\hline
\rowcolor{black!10}
 .&.&.&1&.&.&.&.&.&.&.&.\\
\hline
\end{tabular}
\end{table}}
};

\node [circle,line width=1,draw=red!40, minimum size=2mm] (A) at ($(V) + (-3.2,0.6)$ ) {};
\node [circle,line width=1,draw=red!40, minimum size=2mm] (B) at ($(V) + (-2.6,0.6)$ ) {};
\node [circle,line width=1,draw=red!40, minimum size=2mm] (C) at ($(V) + (-2.0,0.6)$ ) {};
\node [circle,line width=1,draw=red!40, minimum size=2mm] (D) at ($(V) + (-1.4,0.6)$ ) {};
\node [circle,line width=1,draw=red!40, minimum size=2mm] (W) at ($(V) + (1.4,0.6)$ ) {};
\node [circle,line width=1,draw=red!40, minimum size=2mm] (X) at ($(V) + (2.0,0.6)$ ) {};
\node [circle,line width=1,draw=red!40, minimum size=2mm] (Y) at ($(V) + (2.6,0.6)$ ) {};
\node [circle,line width=1,draw=red!40, minimum size=2mm] (Z) at ($(V) + (3.2,0.6)$ ) {};


\node [circle,line width=1,draw=blue!40, minimum size=2mm] (AB) at ($(V) + (-2.9,1.4)$ ) {};
\node [circle,line width=1,draw=blue!40, minimum size=2mm] (CD) at ($(V) + (-1.7,1.4)$ ) {};
\node [circle,line width=1,draw=blue!40, minimum size=2mm] (WX) at ($(V) + (1.7,1.4)$ ) {};
\node [circle,line width=1,draw=blue!40, minimum size=2mm] (YZ) at ($(V) + (2.9,1.4)$ ) {};

\node [circle,line width=1,draw=blue!40, minimum size=2mm] (WXYZ) at ($(V) + (2.3,2.3)$ ) {};
\node [circle,line width=1,draw=blue!40, minimum size=2mm] (ABCD) at ($(V) + (-2.3,2.3)$ ) {};

\node [circle,line width=1,draw=blue!40, minimum size=2mm] (AL) at ($(V) + (0,3.3)$ ) {};


\draw[line width=0.8] (A) -- (AB);
\draw[line width=0.8] (B) -- (AB);

\draw[line width=0.8] (C) -- (CD);
\draw[line width=0.8] (D) -- (CD);
\draw[line width=0.8] (W) -- (WX);
\draw[line width=0.8] (X) -- (WX);
\draw[line width=0.8] (Y) -- (YZ);
\draw[line width=0.8] (Z) -- (YZ);

\draw[line width=0.8] (WX) -- (WXYZ);
\draw[line width=0.8] (YZ) -- (WXYZ);

\draw[line width=0.8] (AB) -- (ABCD);
\draw[line width=0.8] (CD) -- (ABCD);

\draw[line width=0.8] (ABCD) -- (AL);
\draw[line width=0.8] (WXYZ) -- (AL);
\node[text width=0.1\textwidth, ] at ($(V) + (-0.8,0.5)$ ) {.};
\node[text width=0.1\textwidth, ] at ($(V) + (-0.2,0.5)$ ) {.};
\node[text width=0.1\textwidth, ] at ($(V) + (0.4,0.5)$ ) {.};
\node[text width=0.1\textwidth, ] at ($(V) + (1,0.5)$ ) {.};
}
\onslide<1->{\node [circle,line width=1,draw=red!40, minimum size=2mm, fill=red!60] (D) at ($(V) + (-1.4,0.6)$ ) {};
\node [circle,line width=1,draw=blue!40, minimum size=2mm, fill=blue!60] (CD) at ($(V) + (-1.7,1.4)$ ) {};
\node [circle,line width=1,draw=blue!40, minimum size=2mm,  fill=blue!60] (ABCD) at ($(V) + (-2.3,2.3)$ ) {};

\node [circle,line width=1,draw=blue!40, minimum size=2mm, fill=blue!60] (AL) at ($(V) + (0,3.3)$ ) {};


}

\onslide<1->{
	\node[text width=0.2\textwidth, ] at ($(AL) + (-0.8,0)$ ) {\textcolor{blue!50}{\tiny{$\pi(on)_1=1$}}};
	\node[text width=0.2\textwidth, ] at ($(ABCD) + (-0.8,0)$) {\textcolor{blue!50}{\tiny{$\pi(on)_2=0$}}};
	\node[text width=0.2\textwidth, ] at ($(CD) + (1,0)$ ) {\textcolor{blue!50}{\tiny{$\pi(on)_3=0$}}};
}

\onslide<1->{
	\draw [color=black!50, fill=black!40]  ($(AL) + (0.4,-0.1)$) rectangle ($(AL) + (1.1,0.1)$);
	\draw [color=black!50, fill=black!40]  ($(ABCD) + (0.4,-0.1)$) rectangle ($(ABCD) + (1.1,0.1)$);
	\draw [color=black!50,fill=black!40]  ($(YZ) + (0.4,-0.1)$) rectangle ($(YZ) + (1.1,0.1)$);
	\node[text width=0.2\textwidth, ] at ($(AL) + (1.8,0)$ ) {\textcolor{black}{\tiny{$u_1$}}};
	\node[text width=0.2\textwidth, ] at ($(ABCD) + (1.8,0)$) {\textcolor{black}{\tiny{$u_2$}}};
	\node[text width=0.2\textwidth, ] at ($(YZ) + (1.5,-0.2)$ ) {\textcolor{black}{\tiny{$u_V$}}};
}


\node[text width=0.1\textwidth, ] at ($(V) + (-1.1,-0.5)$ ) {\textcolor{red!40}{\footnotesize{\textit{on}}}};
\node[text width=0.2\textwidth, ] at ($(H) + (3.5,0)$) {\textcolor{black}{\footnotesize{$h=v_c$}}};

\draw[line width=0.8,color=blue!50] (-2.15,1.6) -- (-2.7,2.7);
\draw[line width=0.8,color=blue!50] (2.15,1.6) -- (2.7,2.7);

\draw[line width=0.8,color=red!50] (-2.67,3.21) -- (-3.29,4.21);
\draw[line width=0.8,color=red!50] (2.67,3.21) -- (3.29,4.21);
\end{tikzpicture}

		\end{overlayarea}
		\column{0.5\textwidth}
		\begin{overlayarea}{\textwidth}{\textheight}
			\vspace{0.1in}
			\begin{align*}
				P(w|v_c) = \prod\limits_{k=1}^{|\pi(w)|}P(\pi(w_k)|v_c)
			\end{align*}
			\begin{itemize}
				\justifying
				\item<2-> Note that $p(w|v_c)$ can now be computed using
				      $|\pi(w)|$ computations instead of $|V|$ required by softmax
				\item<3-> How do we construct the binary tree?
				\item<4-> Turns out that even a random arrangement of the
				      words on leaf nodes does well in practice
			\end{itemize}
		\end{overlayarea}
	\end{columns}
\end{frame}
