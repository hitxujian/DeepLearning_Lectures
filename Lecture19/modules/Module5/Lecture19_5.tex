\begin{frame}
	\myheading{Module 19.5: Unsupervised Learning with RBMs}
\end{frame}

\begin{frame}
	\begin{columns}
		\column{0.4\textwidth}
			\begin{overlayarea}{\textwidth}{\textheight}
				\centering
\vspace{0.5cm}
\tikzstyle{neuronv}=[circle,minimum size=20pt,inner sep=0pt, thick, fill=orange!30, draw=red!50]
\tikzstyle{neuronh}=[circle,minimum size=20pt,inner sep=0pt, thick, fill=blue!20, draw=blue!60]
\tikzstyle{stateTransition}=[thick]
\tikzstyle{learned}=[text=black]
\begin{tikzpicture}[scale=1.9]
    % \draw ;
    \draw[rounded corners=0.5cm, draw=red!60, thick] (-0.4, -0.25) rectangle (2.5, 0.25) {};
    \draw[rounded corners=0.5cm, draw=red!60, thick] (-0.4, 1.25) rectangle (2.5, 1.75) {};

    \node (v1)[neuronv] at (0, 0) {$v_1$};
    \node (v2)[neuronv] at (0.7, 0) {$v_2$};
    \node (v3)[] at (1.4, 0) {$\cdots$};
    \node (v4)[neuronv] at (2.1, 0) {$v_m$};
    \node[below=0.5cm of v2] (v) {$V \in \{0, 1\}^m$};
    \node[learned,below=0.1cm of v1] (bv1) {$b_1$};
    \node[learned,below=0.1cm of v2] (bv2) {$b_2$};
    % \node[learned,below=0.1cm of v3, scale=0.7] (bv3) {$b_{v_3}$};
    \node[learned,below=0.1cm of v4] (bv4) {$b_m$};

    \node (h1)[neuronh] at (0, 1.5) {$h_1$};
    \node (h2)[neuronh] at (0.7, 1.5) {$h_2$};
    \node (h3)[] at (1.4, 1.5) {$\cdots$};
    \node (h4)[neuronh] at (2.1, 1.5) {$h_n$};
    \node[above=0.5cm of h2] (h) {$H \in \{0, 1\}^n$};
    \node[learned,above=0.1cm of h1] (bv1) {$c_1$};
    \node[learned,above=0.1cm of h2] (bv2) {$c_2$};
    % \node[learned,below=0.1cm of v3, scale=0.7] (bv3) {$b_{v_3}$};
    \node[learned,above=0.1cm of h4] (bv4) {$c_n$};

    \node[learned, scale=0.7] (W) at (2.5, 0.75) {$W \in \mathbb{R}^{m \times n}$};

    \draw[learned,stateTransition] (0,0.17) -- (0,1.33) node [midway,left=-0.1cm] {$w_{1,1}$};
    \draw[stateTransition] (0,0.17) -- (0.7,1.33) node [midway,above=-0.06cm,sloped] {};
    \draw[stateTransition] (0,0.17) -- (2.1,1.33) node [midway,above=-0.06cm,sloped] {};

    \draw[stateTransition] (0.7,0.17) -- (0,1.33) node [midway,above=-0.06cm,sloped] {};
    \draw[stateTransition] (0.7,0.17) -- (0.7,1.33) node [midway,above=-0.06cm,sloped] {};
    \draw[stateTransition] (0.7,0.17) -- (2.1,1.33) node [midway,above=-0.06cm,sloped] {};

    \draw[stateTransition] (2.1,0.17) -- (0,1.33) node [midway,above=-0.06cm,sloped] {};
    \draw[stateTransition] (2.1,0.17) -- (0.7,1.33) node [midway,above=-0.06cm,sloped] {};
    \draw[learned,stateTransition] (2.1,0.17) -- (2.1,1.33) node [midway,left=-0.1cm] {$w_{m,n}$};

\end{tikzpicture}
			\end{overlayarea}
		\column{0.6\textwidth}
			\begin{overlayarea}{\textwidth}{\textheight}
				\begin{itemize}\justifying
					\item<1-> So far, we have mainly dealt with supervised learning where we are given $\{x_i, y_i\}_{i=1}^{n}$ for training
					\item<2-> In other words, for every training example we are given a label (or class) associated with it
					\item<3-> Our job was then to learn a model which predicts $\hat{y}$ such that the difference between $y$ and $\hat{y}$ is minimized
				\end{itemize}
			\end{overlayarea}
	\end{columns}
\end{frame}

\begin{frame}
	\begin{columns}
		\column{0.4\textwidth}
			\begin{overlayarea}{\textwidth}{\textheight}
				\centering
\vspace{0.5cm}
\tikzstyle{neuronv}=[circle,minimum size=20pt,inner sep=0pt, thick, fill=orange!30, draw=red!50]
\tikzstyle{neuronh}=[circle,minimum size=20pt,inner sep=0pt, thick, fill=blue!20, draw=blue!60]
\tikzstyle{stateTransition}=[thick]
\tikzstyle{learned}=[text=black]
\begin{tikzpicture}[scale=1.9]
    % \draw ;
    \draw[rounded corners=0.5cm, draw=red!60, thick] (-0.4, -0.25) rectangle (2.5, 0.25) {};
    \draw[rounded corners=0.5cm, draw=red!60, thick] (-0.4, 1.25) rectangle (2.5, 1.75) {};

    \node (v1)[neuronv] at (0, 0) {$v_1$};
    \node (v2)[neuronv] at (0.7, 0) {$v_2$};
    \node (v3)[] at (1.4, 0) {$\cdots$};
    \node (v4)[neuronv] at (2.1, 0) {$v_m$};
    \node[below=0.5cm of v2] (v) {$V \in \{0, 1\}^m$};
    \node[learned,below=0.1cm of v1] (bv1) {$b_1$};
    \node[learned,below=0.1cm of v2] (bv2) {$b_2$};
    % \node[learned,below=0.1cm of v3, scale=0.7] (bv3) {$b_{v_3}$};
    \node[learned,below=0.1cm of v4] (bv4) {$b_m$};

    \node (h1)[neuronh] at (0, 1.5) {$h_1$};
    \node (h2)[neuronh] at (0.7, 1.5) {$h_2$};
    \node (h3)[] at (1.4, 1.5) {$\cdots$};
    \node (h4)[neuronh] at (2.1, 1.5) {$h_n$};
    \node[above=0.5cm of h2] (h) {$H \in \{0, 1\}^n$};
    \node[learned,above=0.1cm of h1] (bv1) {$c_1$};
    \node[learned,above=0.1cm of h2] (bv2) {$c_2$};
    % \node[learned,below=0.1cm of v3, scale=0.7] (bv3) {$b_{v_3}$};
    \node[learned,above=0.1cm of h4] (bv4) {$c_n$};

    \node[learned, scale=0.7] (W) at (2.5, 0.75) {$W \in \mathbb{R}^{m \times n}$};

    \draw[learned,stateTransition] (0,0.17) -- (0,1.33) node [midway,left=-0.1cm] {$w_{1,1}$};
    \draw[stateTransition] (0,0.17) -- (0.7,1.33) node [midway,above=-0.06cm,sloped] {};
    \draw[stateTransition] (0,0.17) -- (2.1,1.33) node [midway,above=-0.06cm,sloped] {};

    \draw[stateTransition] (0.7,0.17) -- (0,1.33) node [midway,above=-0.06cm,sloped] {};
    \draw[stateTransition] (0.7,0.17) -- (0.7,1.33) node [midway,above=-0.06cm,sloped] {};
    \draw[stateTransition] (0.7,0.17) -- (2.1,1.33) node [midway,above=-0.06cm,sloped] {};

    \draw[stateTransition] (2.1,0.17) -- (0,1.33) node [midway,above=-0.06cm,sloped] {};
    \draw[stateTransition] (2.1,0.17) -- (0.7,1.33) node [midway,above=-0.06cm,sloped] {};
    \draw[learned,stateTransition] (2.1,0.17) -- (2.1,1.33) node [midway,left=-0.1cm] {$w_{m,n}$};

\end{tikzpicture}
			\end{overlayarea}
		\column{0.6\textwidth}
			\begin{overlayarea}{\textwidth}{\textheight}
				\begin{itemize}\justifying
					\item<1-> But in the case of RBMs, our training data only contains $x$ (for example, images)
					\item<2-> There is no explicit label $(y)$ associated with the input
					\item<3-> Of course, in addition to $x$ we have the latent variable $h$ but we don't know what these h's are
					\item<4-> We are interested in learning $P(x, h)$ which we have parameterized as
					\begin{align*}
						P(V, H) = \frac{1}{Z} e^{-(-\sum_i\sum_j w_{ij} v_i h_j  -\sum_i b_i v_i -\sum_j c_j h_j)}
					\end{align*} 
				\end{itemize}
			\end{overlayarea}
	\end{columns}
\end{frame}

\begin{frame}
	\begin{columns}
		\column{0.4\textwidth}
			\begin{overlayarea}{\textwidth}{\textheight}
			\centering
\vspace{0.5cm}
\tikzstyle{neuronv}=[circle,minimum size=20pt,inner sep=0pt, thick, fill=orange!30, draw=red!50]
\tikzstyle{neuronh}=[circle,minimum size=20pt,inner sep=0pt, thick, fill=blue!20, draw=blue!60]
\tikzstyle{stateTransition}=[thick]
\tikzstyle{learned}=[text=black]
\begin{tikzpicture}[scale=1.9]
    % \draw ;
    \draw[rounded corners=0.5cm, draw=red!60, thick] (-0.4, -0.25) rectangle (2.5, 0.25) {};
    \draw[rounded corners=0.5cm, draw=red!60, thick] (-0.4, 1.25) rectangle (2.5, 1.75) {};

    \node (v1)[neuronv] at (0, 0) {$v_1$};
    \node (v2)[neuronv] at (0.7, 0) {$v_2$};
    \node (v3)[] at (1.4, 0) {$\cdots$};
    \node (v4)[neuronv] at (2.1, 0) {$v_m$};
    \node[below=0.5cm of v2] (v) {$V \in \{0, 1\}^m$};
    \node[learned,below=0.1cm of v1] (bv1) {$b_1$};
    \node[learned,below=0.1cm of v2] (bv2) {$b_2$};
    % \node[learned,below=0.1cm of v3, scale=0.7] (bv3) {$b_{v_3}$};
    \node[learned,below=0.1cm of v4] (bv4) {$b_m$};

    \node (h1)[neuronh] at (0, 1.5) {$h_1$};
    \node (h2)[neuronh] at (0.7, 1.5) {$h_2$};
    \node (h3)[] at (1.4, 1.5) {$\cdots$};
    \node (h4)[neuronh] at (2.1, 1.5) {$h_n$};
    \node[above=0.5cm of h2] (h) {$H \in \{0, 1\}^n$};
    \node[learned,above=0.1cm of h1] (bv1) {$c_1$};
    \node[learned,above=0.1cm of h2] (bv2) {$c_2$};
    % \node[learned,below=0.1cm of v3, scale=0.7] (bv3) {$b_{v_3}$};
    \node[learned,above=0.1cm of h4] (bv4) {$c_n$};

    \node[learned, scale=0.7] (W) at (2.5, 0.75) {$W \in \mathbb{R}^{m \times n}$};

    \draw[learned,stateTransition] (0,0.17) -- (0,1.33) node [midway,left=-0.1cm] {$w_{1,1}$};
    \draw[stateTransition] (0,0.17) -- (0.7,1.33) node [midway,above=-0.06cm,sloped] {};
    \draw[stateTransition] (0,0.17) -- (2.1,1.33) node [midway,above=-0.06cm,sloped] {};

    \draw[stateTransition] (0.7,0.17) -- (0,1.33) node [midway,above=-0.06cm,sloped] {};
    \draw[stateTransition] (0.7,0.17) -- (0.7,1.33) node [midway,above=-0.06cm,sloped] {};
    \draw[stateTransition] (0.7,0.17) -- (2.1,1.33) node [midway,above=-0.06cm,sloped] {};

    \draw[stateTransition] (2.1,0.17) -- (0,1.33) node [midway,above=-0.06cm,sloped] {};
    \draw[stateTransition] (2.1,0.17) -- (0.7,1.33) node [midway,above=-0.06cm,sloped] {};
    \draw[learned,stateTransition] (2.1,0.17) -- (2.1,1.33) node [midway,left=-0.1cm] {$w_{m,n}$};

\end{tikzpicture}
			\end{overlayarea}
		\column{0.6\textwidth}
			\begin{overlayarea}{\textwidth}{\textheight}
				\footnotesize{\begin{itemize}[<+->]\justifying
					\item What is the objective function that we should use?
					\item First note that if we have learnt $P(x,h)$ we can compute $P(x)$
					\item What would we want $P(X=x)$ to be for any $x$ belonging to our training data? 
					\item We would want it to be high
					\item So now can you think of an objective function
						\onslide<6->{\begin{align*}
							maximize \prod_{i=1}^{N} P(X=x_i)
						\end{align*}}
					\item \onslide<7->{Or, log-likelihood
					\begin{align*}
						\ln \mathscr{L}(\theta) = \ln \prod_{i=1}^l p(x_i|\theta)=\sum_{i=1}^l \ln p(x_i|\theta)
					\end{align*}}
					\onslide<8->{where $\theta$ are the parameters}
				\end{itemize}}
			\end{overlayarea}
	\end{columns}
\end{frame}

\begin{frame}
	\begin{columns}
		\column{0.4\textwidth}
			\begin{overlayarea}{\textwidth}{\textheight}
			\centering
\vspace{0.5cm}
\tikzstyle{neuronv}=[circle,minimum size=20pt,inner sep=0pt, thick, fill=orange!30, draw=red!50]
\tikzstyle{neuronh}=[circle,minimum size=20pt,inner sep=0pt, thick, fill=blue!20, draw=blue!60]
\tikzstyle{stateTransition}=[thick]
\tikzstyle{learned}=[text=black]
\begin{tikzpicture}[scale=1.9]
    % \draw ;
    \draw[rounded corners=0.5cm, draw=red!60, thick] (-0.4, -0.25) rectangle (2.5, 0.25) {};
    \draw[rounded corners=0.5cm, draw=red!60, thick] (-0.4, 1.25) rectangle (2.5, 1.75) {};

    \node (v1)[neuronv] at (0, 0) {$v_1$};
    \node (v2)[neuronv] at (0.7, 0) {$v_2$};
    \node (v3)[] at (1.4, 0) {$\cdots$};
    \node (v4)[neuronv] at (2.1, 0) {$v_m$};
    \node[below=0.5cm of v2] (v) {$V \in \{0, 1\}^m$};
    \node[learned,below=0.1cm of v1] (bv1) {$b_1$};
    \node[learned,below=0.1cm of v2] (bv2) {$b_2$};
    % \node[learned,below=0.1cm of v3, scale=0.7] (bv3) {$b_{v_3}$};
    \node[learned,below=0.1cm of v4] (bv4) {$b_m$};

    \node (h1)[neuronh] at (0, 1.5) {$h_1$};
    \node (h2)[neuronh] at (0.7, 1.5) {$h_2$};
    \node (h3)[] at (1.4, 1.5) {$\cdots$};
    \node (h4)[neuronh] at (2.1, 1.5) {$h_n$};
    \node[above=0.5cm of h2] (h) {$H \in \{0, 1\}^n$};
    \node[learned,above=0.1cm of h1] (bv1) {$c_1$};
    \node[learned,above=0.1cm of h2] (bv2) {$c_2$};
    % \node[learned,below=0.1cm of v3, scale=0.7] (bv3) {$b_{v_3}$};
    \node[learned,above=0.1cm of h4] (bv4) {$c_n$};

    \node[learned, scale=0.7] (W) at (2.5, 0.75) {$W \in \mathbb{R}^{m \times n}$};

    \draw[learned,stateTransition] (0,0.17) -- (0,1.33) node [midway,left=-0.1cm] {$w_{1,1}$};
    \draw[stateTransition] (0,0.17) -- (0.7,1.33) node [midway,above=-0.06cm,sloped] {};
    \draw[stateTransition] (0,0.17) -- (2.1,1.33) node [midway,above=-0.06cm,sloped] {};

    \draw[stateTransition] (0.7,0.17) -- (0,1.33) node [midway,above=-0.06cm,sloped] {};
    \draw[stateTransition] (0.7,0.17) -- (0.7,1.33) node [midway,above=-0.06cm,sloped] {};
    \draw[stateTransition] (0.7,0.17) -- (2.1,1.33) node [midway,above=-0.06cm,sloped] {};

    \draw[stateTransition] (2.1,0.17) -- (0,1.33) node [midway,above=-0.06cm,sloped] {};
    \draw[stateTransition] (2.1,0.17) -- (0.7,1.33) node [midway,above=-0.06cm,sloped] {};
    \draw[learned,stateTransition] (2.1,0.17) -- (2.1,1.33) node [midway,left=-0.1cm] {$w_{m,n}$};

\end{tikzpicture}
			\end{overlayarea}
		\column{0.6\textwidth}
			\begin{overlayarea}{\textwidth}{\textheight}
				\begin{itemize}\justifying
					\item<1-> Okay so we have the objective function now! What next?
					\item<2-> We need a learning algorithm
					\item<3-> We can just use gradient descent if \onslide{we are able to compute the gradient of the loss function w.r.t. the parameters}
					\item<4-> Let us see if we can do that
				\end{itemize}
			\end{overlayarea}
	\end{columns}
\end{frame}
