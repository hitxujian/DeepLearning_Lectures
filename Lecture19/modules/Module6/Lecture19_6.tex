%----------------------------------------------------------------
\begin{frame}
	\myheading{Module 19.6: Computing the gradient of the log likelihood}
\end{frame}

\begin{frame}
	\begin{columns}
		\column{0.4\textwidth}
		\begin{overlayarea}{\textwidth}{\textheight}
			\centering
\vspace{0.5cm}
\tikzstyle{neuronv}=[circle,minimum size=20pt,inner sep=0pt, thick, fill=orange!30, draw=red!50]
\tikzstyle{neuronh}=[circle,minimum size=20pt,inner sep=0pt, thick, fill=blue!20, draw=blue!60]
\tikzstyle{stateTransition}=[thick]
\tikzstyle{learned}=[text=black]
\begin{tikzpicture}[scale=1.9]
    % \draw ;
    \draw[rounded corners=0.5cm, draw=red!60, thick] (-0.4, -0.25) rectangle (2.5, 0.25) {};
    \draw[rounded corners=0.5cm, draw=red!60, thick] (-0.4, 1.25) rectangle (2.5, 1.75) {};

    \node (v1)[neuronv] at (0, 0) {$v_1$};
    \node (v2)[neuronv] at (0.7, 0) {$v_2$};
    \node (v3)[] at (1.4, 0) {$\cdots$};
    \node (v4)[neuronv] at (2.1, 0) {$v_m$};
    \node[below=0.5cm of v2] (v) {$V \in \{0, 1\}^m$};
    \node[learned,below=0.1cm of v1] (bv1) {$b_1$};
    \node[learned,below=0.1cm of v2] (bv2) {$b_2$};
    % \node[learned,below=0.1cm of v3, scale=0.7] (bv3) {$b_{v_3}$};
    \node[learned,below=0.1cm of v4] (bv4) {$b_m$};

    \node (h1)[neuronh] at (0, 1.5) {$h_1$};
    \node (h2)[neuronh] at (0.7, 1.5) {$h_2$};
    \node (h3)[] at (1.4, 1.5) {$\cdots$};
    \node (h4)[neuronh] at (2.1, 1.5) {$h_n$};
    \node[above=0.5cm of h2] (h) {$H \in \{0, 1\}^n$};
    \node[learned,above=0.1cm of h1] (bv1) {$c_1$};
    \node[learned,above=0.1cm of h2] (bv2) {$c_2$};
    % \node[learned,below=0.1cm of v3, scale=0.7] (bv3) {$b_{v_3}$};
    \node[learned,above=0.1cm of h4] (bv4) {$c_n$};

    \node[learned, scale=0.7] (W) at (2.5, 0.75) {$W \in \mathbb{R}^{m \times n}$};

    \draw[learned,stateTransition] (0,0.17) -- (0,1.33) node [midway,left=-0.1cm] {$w_{1,1}$};
    \draw[stateTransition] (0,0.17) -- (0.7,1.33) node [midway,above=-0.06cm,sloped] {};
    \draw[stateTransition] (0,0.17) -- (2.1,1.33) node [midway,above=-0.06cm,sloped] {};

    \draw[stateTransition] (0.7,0.17) -- (0,1.33) node [midway,above=-0.06cm,sloped] {};
    \draw[stateTransition] (0.7,0.17) -- (0.7,1.33) node [midway,above=-0.06cm,sloped] {};
    \draw[stateTransition] (0.7,0.17) -- (2.1,1.33) node [midway,above=-0.06cm,sloped] {};

    \draw[stateTransition] (2.1,0.17) -- (0,1.33) node [midway,above=-0.06cm,sloped] {};
    \draw[stateTransition] (2.1,0.17) -- (0.7,1.33) node [midway,above=-0.06cm,sloped] {};
    \draw[learned,stateTransition] (2.1,0.17) -- (2.1,1.33) node [midway,left=-0.1cm] {$w_{m,n}$};

\end{tikzpicture}
		\end{overlayarea}
		\column{0.6\textwidth}
		\begin{overlayarea}{\textwidth}{\textheight}
			\footnotesize{\begin{itemize}
				\item We will just consider the loss for a single training example
			\end{itemize}
			\vspace{-0.5cm}
			\begin{align*}
				\ln \mathscr{L}(\theta&) \onslide<2->{= \ln p(V|\theta)} \onslide<3->{= \ln \frac{1}{Z} \sum_H e^{- E(V,H)}} \\
				\onslide<4->{&= \ln \sum_H e^{- E(V,H)} - \ln \sum_{V,H} e^{- E(V,H)}}  
			\end{align*}
			\vspace{-0.5cm}
			\begin{align*}
				\onslide<5->{&\frac{\partial \ln \mathscr{L}(\theta)}{\partial \theta}} \onslide<6->{= \frac{\partial}{\partial \theta} \bigg (\ln \sum_H e^{- E(V,H)} - \ln \sum_{V,H} e^{- E(V,H)} \bigg)\\}
				\onslide<7->{& \hspace{1cm}= -\frac{1}{\sum_H e^{- E(V,H)}} \sum_H e^{- E(V,H)} \frac{\partial E(V,H)}{\partial \theta}\\ 
				& \hspace{1.5cm}+ \frac{1}{\sum_{V,H} e^{- E(V,H)}} \sum_{V,H} e^{- E(V,H)} \frac{\partial E(V,H)}{\partial \theta}\\}
				\onslide<8->{& \hspace{1cm}= -\sum_H \frac{e^{- E(V,H)}}{\sum_H e^{- E(V,H)}} \frac{\partial E(V,H)}{\partial \theta}\\ 
				& \hspace{1.5cm}+ \sum_{V,H} \frac{e^{- E(V,H)}}{\sum_{V,H} e^{- E(V,H)}} \frac{\partial E(V,H)}{\partial \theta}}
			\end{align*}}
		\end{overlayarea}
	\end{columns}
\end{frame}

\begin{frame}
	\begin{columns}
		\column{0.4\textwidth}
		\begin{overlayarea}{\textwidth}{\textheight}
			\centering
\vspace{0.5cm}
\tikzstyle{neuronv}=[circle,minimum size=20pt,inner sep=0pt, thick, fill=orange!30, draw=red!50]
\tikzstyle{neuronh}=[circle,minimum size=20pt,inner sep=0pt, thick, fill=blue!20, draw=blue!60]
\tikzstyle{stateTransition}=[thick]
\tikzstyle{learned}=[text=black]
\begin{tikzpicture}[scale=1.9]
    % \draw ;
    \draw[rounded corners=0.5cm, draw=red!60, thick] (-0.4, -0.25) rectangle (2.5, 0.25) {};
    \draw[rounded corners=0.5cm, draw=red!60, thick] (-0.4, 1.25) rectangle (2.5, 1.75) {};

    \node (v1)[neuronv] at (0, 0) {$v_1$};
    \node (v2)[neuronv] at (0.7, 0) {$v_2$};
    \node (v3)[] at (1.4, 0) {$\cdots$};
    \node (v4)[neuronv] at (2.1, 0) {$v_m$};
    \node[below=0.5cm of v2] (v) {$V \in \{0, 1\}^m$};
    \node[learned,below=0.1cm of v1] (bv1) {$b_1$};
    \node[learned,below=0.1cm of v2] (bv2) {$b_2$};
    % \node[learned,below=0.1cm of v3, scale=0.7] (bv3) {$b_{v_3}$};
    \node[learned,below=0.1cm of v4] (bv4) {$b_m$};

    \node (h1)[neuronh] at (0, 1.5) {$h_1$};
    \node (h2)[neuronh] at (0.7, 1.5) {$h_2$};
    \node (h3)[] at (1.4, 1.5) {$\cdots$};
    \node (h4)[neuronh] at (2.1, 1.5) {$h_n$};
    \node[above=0.5cm of h2] (h) {$H \in \{0, 1\}^n$};
    \node[learned,above=0.1cm of h1] (bv1) {$c_1$};
    \node[learned,above=0.1cm of h2] (bv2) {$c_2$};
    % \node[learned,below=0.1cm of v3, scale=0.7] (bv3) {$b_{v_3}$};
    \node[learned,above=0.1cm of h4] (bv4) {$c_n$};

    \node[learned, scale=0.7] (W) at (2.5, 0.75) {$W \in \mathbb{R}^{m \times n}$};

    \draw[learned,stateTransition] (0,0.17) -- (0,1.33) node [midway,left=-0.1cm] {$w_{1,1}$};
    \draw[stateTransition] (0,0.17) -- (0.7,1.33) node [midway,above=-0.06cm,sloped] {};
    \draw[stateTransition] (0,0.17) -- (2.1,1.33) node [midway,above=-0.06cm,sloped] {};

    \draw[stateTransition] (0.7,0.17) -- (0,1.33) node [midway,above=-0.06cm,sloped] {};
    \draw[stateTransition] (0.7,0.17) -- (0.7,1.33) node [midway,above=-0.06cm,sloped] {};
    \draw[stateTransition] (0.7,0.17) -- (2.1,1.33) node [midway,above=-0.06cm,sloped] {};

    \draw[stateTransition] (2.1,0.17) -- (0,1.33) node [midway,above=-0.06cm,sloped] {};
    \draw[stateTransition] (2.1,0.17) -- (0.7,1.33) node [midway,above=-0.06cm,sloped] {};
    \draw[learned,stateTransition] (2.1,0.17) -- (2.1,1.33) node [midway,left=-0.1cm] {$w_{m,n}$};

\end{tikzpicture}
		\end{overlayarea}
		\column{0.6\textwidth}
		\begin{overlayarea}{\textwidth}{\textheight}
			\footnotesize{\begin{itemize}
				\item Now,
			\end{itemize}
			\begin{align*}
				\onslide<2->{\frac{e^{-E(V,H)}}{\sum_{V,H} e^{-E(V,H)}} = p(V,H)}
			\end{align*}
			\begin{align*}
				\onslide<3->{\frac{e^{-E(V,H)}}{\sum_{H} e^{-E(V,H)}}} \onslide<4->{&= \frac{\frac{1}{Z} e^{-E(V,H)}}{\frac{1}{Z} \sum_{H} e^{-E(V,H)}} \\}
				\onslide<5->{&= \frac{p(V,H)}{p(V)}} \onslide<6->{= p(H|V)}
			\end{align*}
			\begin{align*}
				\onslide<7->{&\frac{\partial \ln \mathscr{L}(\theta)}{\partial \theta}} \onslide<8->{= -\sum_H \frac{e^{- E(V,H)}}{\sum_H e^{- E(V,H)}} \frac{\partial E(V,H)}{\partial \theta}\\ 
				& \hspace{2.5cm}+ \sum_{V,H} \frac{e^{- E(V,H)}}{\sum_{V,H} e^{- E(V,H)}} \frac{\partial E(V,H)}{\partial \theta}\\}
				\onslide<9->{& = -\sum_H p(H|V) \frac{\partial E(V,H)}{\partial \theta} + \sum_{V,H} p(V,H) \frac{\partial E(V,H)}{\partial \theta}}
			\end{align*}}
		\end{overlayarea}
	\end{columns}
\end{frame}

\begin{frame}
	\begin{columns}
		\column{0.4\textwidth}
		\begin{overlayarea}{\textwidth}{\textheight}
			\centering
\vspace{0.5cm}
\tikzstyle{neuronv}=[circle,minimum size=20pt,inner sep=0pt, thick, fill=orange!30, draw=red!50]
\tikzstyle{neuronh}=[circle,minimum size=20pt,inner sep=0pt, thick, fill=blue!20, draw=blue!60]
\tikzstyle{stateTransition}=[thick]
\tikzstyle{learned}=[text=black]
\begin{tikzpicture}[scale=1.9]
    % \draw ;
    \draw[rounded corners=0.5cm, draw=red!60, thick] (-0.4, -0.25) rectangle (2.5, 0.25) {};
    \draw[rounded corners=0.5cm, draw=red!60, thick] (-0.4, 1.25) rectangle (2.5, 1.75) {};

    \node (v1)[neuronv] at (0, 0) {$v_1$};
    \node (v2)[neuronv] at (0.7, 0) {$v_2$};
    \node (v3)[] at (1.4, 0) {$\cdots$};
    \node (v4)[neuronv] at (2.1, 0) {$v_m$};
    \node[below=0.5cm of v2] (v) {$V \in \{0, 1\}^m$};
    \node[learned,below=0.1cm of v1] (bv1) {$b_1$};
    \node[learned,below=0.1cm of v2] (bv2) {$b_2$};
    % \node[learned,below=0.1cm of v3, scale=0.7] (bv3) {$b_{v_3}$};
    \node[learned,below=0.1cm of v4] (bv4) {$b_m$};

    \node (h1)[neuronh] at (0, 1.5) {$h_1$};
    \node (h2)[neuronh] at (0.7, 1.5) {$h_2$};
    \node (h3)[] at (1.4, 1.5) {$\cdots$};
    \node (h4)[neuronh] at (2.1, 1.5) {$h_n$};
    \node[above=0.5cm of h2] (h) {$H \in \{0, 1\}^n$};
    \node[learned,above=0.1cm of h1] (bv1) {$c_1$};
    \node[learned,above=0.1cm of h2] (bv2) {$c_2$};
    % \node[learned,below=0.1cm of v3, scale=0.7] (bv3) {$b_{v_3}$};
    \node[learned,above=0.1cm of h4] (bv4) {$c_n$};

    \node[learned, scale=0.7] (W) at (2.5, 0.75) {$W \in \mathbb{R}^{m \times n}$};

    \draw[learned,stateTransition] (0,0.17) -- (0,1.33) node [midway,left=-0.1cm] {$w_{1,1}$};
    \draw[stateTransition] (0,0.17) -- (0.7,1.33) node [midway,above=-0.06cm,sloped] {};
    \draw[stateTransition] (0,0.17) -- (2.1,1.33) node [midway,above=-0.06cm,sloped] {};

    \draw[stateTransition] (0.7,0.17) -- (0,1.33) node [midway,above=-0.06cm,sloped] {};
    \draw[stateTransition] (0.7,0.17) -- (0.7,1.33) node [midway,above=-0.06cm,sloped] {};
    \draw[stateTransition] (0.7,0.17) -- (2.1,1.33) node [midway,above=-0.06cm,sloped] {};

    \draw[stateTransition] (2.1,0.17) -- (0,1.33) node [midway,above=-0.06cm,sloped] {};
    \draw[stateTransition] (2.1,0.17) -- (0.7,1.33) node [midway,above=-0.06cm,sloped] {};
    \draw[learned,stateTransition] (2.1,0.17) -- (2.1,1.33) node [midway,left=-0.1cm] {$w_{m,n}$};

\end{tikzpicture}
		\end{overlayarea}
		\column{0.6\textwidth}
		\begin{overlayarea}{\textwidth}{\textheight}
			\begin{itemize}\justifying
				\item<1->Okay, so we have,	
					\begin{align*}
						\frac{\partial \ln \mathscr{L}(\theta)}{\partial \theta} &= -\sum_H p(H|V) \frac{\partial E(V,H)}{\partial \theta} \\
						&+ \sum_{V,H} p(V,H) \frac{\partial E(V,H)}{\partial \theta}
					\end{align*}
				\item<2-> Remember that $\theta$ is a collection of all the parameters in our model, i.e.,
				$W_{ij}, b_i, c_j \forall i \in \{1,\ldots,m\}$ and $\forall j \in \{1,\ldots,n\}$ 
				\item<3-> We will follow our usual recipe of computing the partial derivative w.r.t. one weight $w_{ij}$ and then generalize to the gradient w.r.t. the entire weight matrix $W$
			\end{itemize}
		\end{overlayarea}
	\end{columns}
\end{frame}


\begin{frame}
	\begin{columns}
		\column{0.4\textwidth}
		\begin{overlayarea}{\textwidth}{\textheight}
			\centering
\vspace{0.5cm}
\tikzstyle{neuronv}=[circle,minimum size=20pt,inner sep=0pt, thick, fill=orange!30, draw=red!50]
\tikzstyle{neuronh}=[circle,minimum size=20pt,inner sep=0pt, thick, fill=blue!20, draw=blue!60]
\tikzstyle{stateTransition}=[thick]
\tikzstyle{learned}=[text=black]
\begin{tikzpicture}[scale=1.9]
    % \draw ;
    \draw[rounded corners=0.5cm, draw=red!60, thick] (-0.4, -0.25) rectangle (2.5, 0.25) {};
    \draw[rounded corners=0.5cm, draw=red!60, thick] (-0.4, 1.25) rectangle (2.5, 1.75) {};

    \node (v1)[neuronv] at (0, 0) {$v_1$};
    \node (v2)[neuronv] at (0.7, 0) {$v_2$};
    \node (v3)[] at (1.4, 0) {$\cdots$};
    \node (v4)[neuronv] at (2.1, 0) {$v_m$};
    \node[below=0.5cm of v2] (v) {$V \in \{0, 1\}^m$};
    \node[learned,below=0.1cm of v1] (bv1) {$b_1$};
    \node[learned,below=0.1cm of v2] (bv2) {$b_2$};
    % \node[learned,below=0.1cm of v3, scale=0.7] (bv3) {$b_{v_3}$};
    \node[learned,below=0.1cm of v4] (bv4) {$b_m$};

    \node (h1)[neuronh] at (0, 1.5) {$h_1$};
    \node (h2)[neuronh] at (0.7, 1.5) {$h_2$};
    \node (h3)[] at (1.4, 1.5) {$\cdots$};
    \node (h4)[neuronh] at (2.1, 1.5) {$h_n$};
    \node[above=0.5cm of h2] (h) {$H \in \{0, 1\}^n$};
    \node[learned,above=0.1cm of h1] (bv1) {$c_1$};
    \node[learned,above=0.1cm of h2] (bv2) {$c_2$};
    % \node[learned,below=0.1cm of v3, scale=0.7] (bv3) {$b_{v_3}$};
    \node[learned,above=0.1cm of h4] (bv4) {$c_n$};

    \node[learned, scale=0.7] (W) at (2.5, 0.75) {$W \in \mathbb{R}^{m \times n}$};

    \draw[learned,stateTransition] (0,0.17) -- (0,1.33) node [midway,left=-0.1cm] {$w_{1,1}$};
    \draw[stateTransition] (0,0.17) -- (0.7,1.33) node [midway,above=-0.06cm,sloped] {};
    \draw[stateTransition] (0,0.17) -- (2.1,1.33) node [midway,above=-0.06cm,sloped] {};

    \draw[stateTransition] (0.7,0.17) -- (0,1.33) node [midway,above=-0.06cm,sloped] {};
    \draw[stateTransition] (0.7,0.17) -- (0.7,1.33) node [midway,above=-0.06cm,sloped] {};
    \draw[stateTransition] (0.7,0.17) -- (2.1,1.33) node [midway,above=-0.06cm,sloped] {};

    \draw[stateTransition] (2.1,0.17) -- (0,1.33) node [midway,above=-0.06cm,sloped] {};
    \draw[stateTransition] (2.1,0.17) -- (0.7,1.33) node [midway,above=-0.06cm,sloped] {};
    \draw[learned,stateTransition] (2.1,0.17) -- (2.1,1.33) node [midway,left=-0.1cm] {$w_{m,n}$};

\end{tikzpicture}
		\end{overlayarea}
		\column{0.6\textwidth}
		\begin{overlayarea}{\textwidth}{\textheight}
			\begin{align*}
				&\frac{\partial \mathscr{L}(\theta)}{\partial w_{ij}} \\
				\onslide<2->{&= -\sum_H p(H|V) \frac{\partial E(V,H)}{\partial w_{ij}} + \sum_{V,H} p(V,H) \frac{\partial E(V,H)}{\partial w_{ij}}\\}
				\onslide<3->{&= \sum_H p(H|V) h_i v_j - \sum_{V,H} p(V,H) h_i v_j\\}
				\onslide<5->{&= \mathbb{E}_{p(H|V)} [v_i h_j] - \mathbb{E}_{p(V,H)} [v_i h_j]}
				% \onslide<4->{&= \sum_H p(H|V) h_i v_j - \sum_{V} p(V) \sum_{H} p(H|V) h_i v_j}
			\end{align*}
			\begin{itemize}\justifying
				\item<4-> We can write the above as a sum of two expectations
			\end{itemize}
		\end{overlayarea}
	\end{columns}
\end{frame}

\begin{frame}
	\begin{columns}
		\column{0.4\textwidth}
		\begin{overlayarea}{\textwidth}{\textheight}
			\begin{align*}
				\hspace{-0.5cm}\frac{\partial \mathscr{L}(\theta)}{\partial w_{ij}} = \mathbb{E}_{p(H|V)} [v_i h_j] - \mathbb{E}_{p(V,H)} [v_i h_j]
			\end{align*}
		\end{overlayarea}
		\column{0.6\textwidth}
		\begin{overlayarea}{\textwidth}{\textheight}
			\begin{itemize}
				\item<1-> How do we compute these expectations? 
				\item<2-> The first summation can actually be simplified (we will come back and simplify it later) 
				\item<3-> However, the second summation contains an exponential number of terms and hence intractable in practice
				\item<4-> So how do we deal with this ?
			\end{itemize}
		\end{overlayarea}
	\end{columns}
\end{frame}
