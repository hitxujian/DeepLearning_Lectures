\PassOptionsToClass{}{beamer}
\documentclass[serif,aspectratio=169,dvipsnames]{beamer}
\usepackage[utf8]{inputenc}

%\StartShownPreambleCommands
\usepackage{amsmath,esint}
\usepackage{amssymb}
\usepackage[british]{babel}
\usepackage{multicol}
\usetheme{Warsaw}
\usecolortheme{rose}
\usepackage[normalem]{ulem}
%\usetheme{metropolis}
%\usepackage{appendixnumberbeamer}

%\StopShownPreambleCommands
\usepackage{pgfplots}

\usepackage{ mathrsfs }
\usepackage{hyperref}

\usepackage{gensymb}
\usepackage{color}
\usepackage{tkz-euclide}
\usetkzobj{all}
\usepackage{tkz-fct}  
\usetikzlibrary{calc}
\usepackage{ragged2e}
\usepackage[ruled]{algorithm2e}
\usepackage{tikz}
\usepackage{animate}
\usepackage{adjustbox}
\usepackage[labelformat=empty]{caption}
\usepackage{blindtext}
\usepackage{biblatex}

\usepackage{utils/dtklogos}
\usepackage{utils/style}

\usetikzlibrary{matrix,chains,positioning,decorations.pathreplacing,arrows} 
\usetikzlibrary{shapes.geometric}
\usetikzlibrary{intersections}


\makeatletter

\DeclareMathOperator*{\minimize}{minimize}



\addtobeamertemplate{navigation symbols}{}{%
    \usebeamerfont{footline}%
    \usebeamercolor[fg]{footline}%
    \hspace{1em}%
    \insertframenumber/\inserttotalframenumber
}

\titlegraphicii{}
\setbeamertemplate{title page}
{
  \vspace{0.3in}
  \vbox{}
   %{\usebeamercolor[fg]{titlegraphic}\inserttitlegraphic\hfill\inserttitlegraphicii\par}
  \begin{centering}
    \begin{beamercolorbox}[sep=8pt,center]{title}
      \usebeamerfont{title}\inserttitle\par%
      \ifx\insertsubtitle\@empty%
      \else%
        \vskip0.25em%
        {\usebeamerfont{subtitle}\usebeamercolor[fg]{subtitle}\insertsubtitle\par}%
      \fi%     
    \end{beamercolorbox}%
    \vskip1em\par
    \begin{beamercolorbox}[sep=8pt,center]{date}
      \usebeamerfont{date}\insertdate
    \end{beamercolorbox}%\vskip0.5em
    \begin{beamercolorbox}[sep=8pt,center]{author}
      \usebeamerfont{author}\insertauthor
    \end{beamercolorbox}
    \begin{beamercolorbox}[sep=8pt,center]{institute}
      \usebeamerfont{institute}\insertinstitute
    \end{beamercolorbox}
  \end{centering}
  %\vfill
}
\makeatother

\author{Mitesh M. Khapra}
\title{CS7015 (Deep Learning) : Lecture 11}
\subtitle{Convolutional Neural Networks, LeNet, AlexNet, ZF-Net, VGGNet, GoogLeNet and ResNet}
\institute{Department of Computer Science and Engineering\\ Indian Institute of Technology Madras}
\date{}
\titlegraphic{\includegraphics[height=1cm,width=2cm]{images/iitm_logo.png}}

\newcommand{\borderColor}{white}
\newcommand{\forefillColor}{blue!20!black!30!green}
\newcommand{\toprsidefillcolor}{green!40}

\newcommand{\handmadecube}[5]{
\pgfmathsetmacro{\width}{#1}
\pgfmathsetmacro{\height}{#2}
\pgfmathsetmacro{\x}{#4}
\pgfmathsetmacro{\y}{#5}
\pgfmathsetmacro{\ti}{#3}


       \pgfmathsetmacro{\yfactor}{0.9}
            \pgfmathsetmacro{\xfactor}{0.6}


% % square
\draw [-,draw=\borderColor] (\x-\width,\y) -- (\x,\y); % % x cahnge ->
\draw [-,draw=\borderColor] (\x,\y) -- (\x,\y-\height); % % y change   |
\draw [-,draw=\borderColor] (\x,\y-\height) -- (\x-\width,\y-\height); % % x change <-
\draw [-,draw=\borderColor] (\x-\width,\y-\height) -- (\x-\width,\y); % % y change  |

\draw  [draw=\borderColor,fill=\forefillColor] (\x-\width,\y)--(\x,\y) -- (\x,\y-\height) -- (\x-\width,\y-\height) --cycle ;

\draw  [draw=\borderColor,fill=\toprsidefillcolor] (\x,\y) -- (\x+\xfactor*\ti,\y+\yfactor*\ti) -- (\x+\xfactor*\ti,\y+\yfactor*\ti-\height) -- (\x,\y-\height) --cycle ;

\draw  [draw=\borderColor,fill=\toprsidefillcolor] (\x-\width,\y)--(\x,\y) -- (\x+\xfactor*\ti,\y+\yfactor*\ti) -- (\x+\xfactor*\ti-\width,\y+\yfactor*\ti) --cycle ;

\draw [-,draw=\borderColor]   (\x,\y) -- (\x+\xfactor*\ti,\y+\yfactor*\ti);

\draw [-,draw=\borderColor] (\x-\width,\y) -- (\x-\width+\xfactor*\ti,\y+\yfactor*\ti);

\draw [-,draw=\borderColor] (\x-\width+\xfactor*\ti,\y+\yfactor*\ti) -- (\x+\xfactor*\ti,\y+\yfactor*\ti);

\draw [-,draw=\borderColor] (\x+\xfactor*\ti,\y+\yfactor*\ti) -- (\x+\xfactor*\ti,\y+\yfactor*\ti-\height);

\draw [-,draw=\borderColor] (\x,\y-\height) -- (\x+\xfactor*\ti,\y+\yfactor*\ti-\height);

}


\usetikzlibrary{svg.path}
\usetikzlibrary{decorations.pathreplacing,decorations.markings}
\newcounter{pos}
\tikzset{
  initcounter/.code={\setcounter{pos}{0}},
  style between/.style n args={3}{
    postaction={
      initcounter,
      decorate,
      decoration={
        show path construction,
        curveto code={
          \addtocounter{pos}{1}
          \pgfmathtruncatemacro{\min}{#1 - 1}
          \ifthenelse{\thepos < #2 \AND \thepos > \min}{
            \draw[#3]
            (\tikzinputsegmentfirst)
            ..
            controls (\tikzinputsegmentsupporta) and (\tikzinputsegmentsupportb)
            ..
            (\tikzinputsegmentlast);
          }{}
        }
      }
    },
  },
}

\newcommand{\firstrowcolor}{black}
\newcommand{\secondrowcolor}{black}
\newcommand{\thirdrowcolor}{black}
\newcommand{\fourrowcolor}{black}

\begin{document}
\newcommand{\tikzmark}[1]{\tikz[baseline,remember picture] \coordinate (#1) {};}

\def\cuboid#1#2#3#4#5{
	\begin{scope}
		\edef\mycolor{#2}
		\edef\depth{#3}
		\edef\height{#4}
		\edef\width{#5}
		\draw[black,fill=\mycolor] #1 -- ++(-\depth,0,0) -- ++(0,-\height,0) -- ++(\depth,0,0) -- cycle #1 -- ++(0,0,-\width) -- ++(0,-\height,0) -- ++(0,0,\width) -- cycle  #1 -- ++(-\depth,0,0) -- ++(0,0,-\width) -- ++(\depth,0,0) -- cycle;
	\end{scope}
}
\def\cuboidlabelmine#1#2#3#4#5#6#7#8{
	\begin{scope}
		\edef\mycolor{#2}
		\edef\depth{#3}
		\edef\height{#4}
		\edef\width{#5}
		\edef\depthlabel{#6}
		\edef\heightlabel{#7}
		\edef\widthlabel{#8}
		\draw[black,fill=\mycolor] #1 -- ++(-\depth,0,0) -- ++(0,-\height,0) -- ++(\depth,0,0) node[pos=0.5,below] {\tiny \depthlabel} -- cycle #1 -- ++(0,0,-\width) -- ++(0,-\height,0) node[pos=0.6,right,outer sep=-3pt] {\tiny \heightlabel} -- ++(0,0,\width)  node[pos=0.5,below] {\tiny \widthlabel} -- cycle  #1 -- ++(-\depth,0,0) -- ++(0,0,-\width) -- ++(\depth,0,0) -- cycle;
	\end{scope}
}

\def\lenetparamnew#1#2#3#4#5#6#7#8{
	\begin{scope}
		\edef\mycolor{#2}
		\edef\depth{#3}
		\edef\height{#4}
		\edef\width{#5}
		\edef\depthlabel{#8}
		\edef\heightlabel{#7}
		\edef\sep{#6}
		\draw[black,fill=\mycolor] #1 -- ++(-\depth,0,0) -- ++(0,-\height,0) -- ++(\depth,0,0) -- cycle #1 -- ++(0,0,-\width) -- ++(0,-\height,0) node[pos=0.9,above,outer sep=-13 pt] {\tiny \heightlabel} node[pos=0.9,above,outer sep=-20 pt] {\tiny \depthlabel} -- ++(0,0,\width)  -- cycle  #1 -- ++(-\depth,0,0) -- ++(0,0,-\width) -- ++(\depth,0,0) -- cycle;
	\end{scope}
}
\def\lenetparam#1#2#3#4#5#6#7#8{
	\begin{scope}
		\edef\mycolor{#2}
		\edef\depth{#3}
		\edef\height{#4}
		\edef\width{#5}
		\edef\depthlabel{#6}
		\edef\heightlabel{#7}
		\edef\widthlabel{#8}
		\draw[black] #1 -- ++(-\depth,0,0) -- ++(0,-\height,0) -- ++(\depth,0,0) node[pos=0.5,above,outer sep=-28pt] {\tiny \depthlabel} node[pos=0.6,above,outer sep=-35pt] {\tiny \heightlabel} node[pos=0.5,above,outer sep=-42] {\tiny \widthlabel}  -- cycle #1 -- ++(0,0,-\width) -- ++(0,-\height,0)  -- ++(0,0,\width) -- cycle  #1 -- ++(-\depth,0,0) -- ++(0,0,-\width) -- ++(\depth,0,0) -- cycle;
	\end{scope}
}

\def\kernel#1#2#3#4#5#6{
	\begin{scope}
		\edef\mycolor{#2}
		\edef\depth{#3}
		\edef\height{#4}
		\edef\width{#5}
		\draw[black,fill=\mycolor] #1 -- ++(-\depth,0,0) -- ++(0,-\height,0) -- ++(\depth,0,0) -- cycle #1 -- ++(0,0,-\width) -- ++(0,-\height,0) -- ++(0,0,\width) -- cycle  #1 -- ++(-\depth,0,0) -- ++(0,0,-\width) -- ++(\depth,0,0) -- cycle;
		\draw[dashed] #1 -- #6 #1++(0,0,-\width) -- #6 #1++(0,-\height,0) -- #6 #1++(0,-\height,-\width) -- #6;
	\end{scope}
}
\def\kernellabel#1#2#3#4#5#6#7#8#9{
	%#6 is target pixel
	\begin{scope}
		\edef\mycolor{#2}
		\edef\depth{#3}
		\edef\height{#4}
		\edef\width{#5}
		\edef\depthlabel{#7}
		\edef\heightlabel{#8}
		\edef\widthlabel{#9}
		\draw[black,fill=\mycolor] #1 -- ++(-\depth,0,0) -- ++(0,-\height,0) -- ++(\depth,0,0) node[pos=0.5,below] {\tiny \depthlabel} -- cycle #1 -- ++(0,0,-\width) -- ++(0,-\height,0) node[pos=0.5,right] {\tiny \heightlabel} -- ++(0,0,\width)  node[pos=0.5,below] {\tiny \widthlabel} -- cycle  #1 -- ++(-\depth,0,0) -- ++(0,0,-\width) -- ++(\depth,0,0) -- cycle;
		\draw[dashed] #1 -- #6 #1++(0,0,-\width) -- #6 #1++(0,-\height,0) -- #6 #1++(0,-\height,-\width) -- #6;
	\end{scope}
}


\def\mybox#1#2#3#4#5#6{
	\begin{scope}
		\edef\mycolor{#2}
		\edef\depth{#3}
		\edef\height{#4}
		\edef\width{#5}
		\edef\depthlabel{#6}
		\draw[black,fill=\mycolor] #1 -- ++(-\depth,0,0) -- ++(0,-\height,0) -- ++(\depth,0,0) node[pos=0.5,below] {\tiny \depthlabel} -- cycle #1 -- ++(0,0,-\width) -- ++(0,-\height,0) -- ++(0,0,\width) -- cycle  #1 -- ++(-\depth,0,0) -- ++(0,0,-\width) -- ++(\depth,0,0) -- cycle;
	\end{scope}
}

\def\mylabel#1#2#3#4#5#6#7#8{
	\begin{scope}
		\edef\mycolor{#2}
		\edef\depth{#3}
		\edef\height{#4}
		\edef\width{#5}
		\edef\depthlabel{#6}
		\edef\heightlabel{#7}
		\edef\widthlabel{#8}
		\draw[black,fill=\mycolor] #1 -- ++(-\depth,0,0) node[pos=0.5,below] {\tiny \depthlabel} node[pos=0.5,below,outer sep=7pt] {\tiny \widthlabel} -- ++(0,-\height,0) -- ++(\depth,0,0) node[pos=0.5,above] {\tiny \heightlabel} -- cycle #1 -- ++(0,0,-\width) -- ++(0,-\height,0) -- ++(0,0,\width) -- cycle  #1 -- ++(-\depth,0,0) -- ++(0,0,-\width) -- ++(\depth,0,0) -- cycle;
	\end{scope}
}

\def\triangle#1#2#3{
	\begin{scope}
		\edef\xa{#1}
		\edef\ya{#2}
		\edef\mycolor{#3}
		\draw[black,fill=\mycolor] (\xa,\ya) -- ($(\xa,\ya)+(0.2,0)$) -- ($(\xa,\ya)+(0.1,0.2)$) -- (\xa,\ya);
	\end{scope}
}
	
		
		
\maketitle
%\centering

%\begin{frame}
	\begin{block}{Acknowledgements}
		\begin{itemize}\justifying
			%\footnote{xx}
			\item Chapter 7, Deep Learning book  
			\item Ali Ghodsi's Video Lectures on Regularization\footnote{\href{http://www.youtube.com/watch?v=21jL0I6wbns}{\textcolor{blue}{Lecture 2.1}} and \href{http://www.youtube.com/watch?v=_ojGVetxCpQ}{\textcolor{blue}{Lecture 2.2}}}
			\item Dropout:  A Simple Way to Prevent Neural Networks from Overfitting\footnote{\href{http://www.cs.toronto.edu/~hinton/absps/JMLRdropout.pdf}{\textcolor{blue}{Dropout}}}
		\end{itemize}
	\end{block}
\end{frame}


\foreach \n in {1, 2, 3, 4, 5}
   	{\input{modules/Module\n/Lecture11_\n}}
										
%\begin{frame}
	\myheading{Appendix}
\end{frame}

\begin{frame}
	\begin{overlayarea}{\textwidth}{\textheight}
		\begin{itemize}
			\item<1-> To prove: The below two equations are equivalent
				\begin{align*}
					w_t & = (I-\eta Q \Lambda Q^{T} )w_{t-1}+\eta Q \Lambda Q^{T} w^{*}\\
					w_t & = Q[I-(I-\varepsilon \Lambda)^{t}]Q^{T}w^{*} 
				\end{align*}
			\item<2-> Proof by induction:
			\item<3-> Base case: $t$ = 1 and $w_0$=0:
			\item<4-> $w_1$ according to the first equation:
				\begin{align*}
					w_1 & = (I - \eta Q\Lambda Q^T) w_0 + \eta Q\Lambda Q^T w^* \\
					         & = \eta Q\Lambda Q^T w^*
				\end{align*}
			\item<5-> $w_1$ according to the second equation:
                \begin{align*}
					w_1 & = Q(I - (I - \eta \Lambda)^1)Q^T w^*\\
                  			 & = \eta Q\Lambda Q^T w^*
                \end{align*}
        \end{itemize}
	\end{overlayarea}
\end{frame}

\begin{frame}
	\begin{overlayarea}{\textwidth}{\textheight}
		\begin{itemize}
			\item<1-> Induction step: Let the two equations be equivalent for $t^{th}$ step
              \begin{align*}
                \therefore w_t & = (I-\eta Q \Lambda Q^{T} )w_{t-1}+\eta Q \Lambda Q^{T} w^{*}\\
				                    & = Q[I-(I-\varepsilon \Lambda)^{t}]Q^{T}w^{*} 
              \end{align*}
            \item<2-> Proof that this will hold for $(t+1)^{th}$ step
			\begin{align*}
				\onslide<3-> {w_{t+1} & = (I - \eta Q\Lambda Q^T) w_{t} + \eta Q\Lambda Q^T w^*\\}
				\only<4>{& (\textcolor{red}{\text{using }w_t = Q[I-(I-\varepsilon \Lambda)^{t}]Q^{T}w^{*}}) \\}
				\only<5->{& (\text{using }w_t = Q[I-(I-\varepsilon \Lambda)^{t}]Q^{T}w^{*}) \\}
				\only<5>{& = (I - \eta Q\Lambda Q^T) \textcolor{red}{Q(I - (I - \eta \Lambda)^t)Q^T w^*} + \eta Q\Lambda Q^T w^* \\}
				\only<6>{& = \textcolor{red}{(I - \eta Q\Lambda Q^T)} Q(I - (I - \eta \Lambda)^t)Q^T w^* + \eta Q\Lambda Q^T w^* \\} 
				\only<7->{& = (I - \eta Q\Lambda Q^T) Q(I - (I - \eta \Lambda)^t)Q^T w^* + \eta Q\Lambda Q^T w^* \\} 
				\only<6->{&(\text{Opening this bracket})\\}
				\onslide<7-> {& = \textcolor{red}{I}Q(I - (I - \eta \Lambda)^t)Q^T w^* - \textcolor{red}{\eta Q\Lambda Q^T}Q (I - (I - \eta \Lambda)^t)Q^T w^* + \eta Q\Lambda Q^T w^*\\}
				\onslide<8-> {& = Q(I - (I - \eta \Lambda)^t)Q^T w^* - \eta Q\Lambda Q^TQ (I - (I - \eta \Lambda)^t)Q^T w^* + \eta Q\Lambda Q^T w^*\\}
			\end{align*}
        \end{itemize}
	\end{overlayarea}
\end{frame}

\begin{frame}
	\begin{overlayarea}{\textwidth}{\textheight}
        \begin{itemize}
            \item Continuing
              \begin{align*}
                \onslide<1-> {w_{t+1} & = Q(I - (I - \eta \Lambda)^t)Q^T w^* - \eta Q\Lambda Q^TQ (I - (I - \eta \Lambda)^t)Q^T w^* + \eta Q\Lambda Q^T w^*\\}
				\onslide<2-> {& = Q(I - (I - \eta \Lambda)^t)Q^T w^* - \eta Q\Lambda(I - (I - \eta \Lambda)^t)Q^T w^* + \eta Q\Lambda Q^T w^* \textcolor{red}{(\because Q^TQ=I)}\\}
				\only<3> {& = Q(I - (I - \eta \Lambda)^t)\textcolor{red}{Q^T w^*} - \eta Q\Lambda(I - (I - \eta \Lambda)^t)\textcolor{red}{Q^T w^*} + \eta Q\Lambda \textcolor{red}{Q^T w^*} \\}
				\only<3> {& = Q\big[(I - (I - \eta \Lambda)^t) - \eta \Lambda(I - (I - \eta \Lambda)^t) + \eta \Lambda\big] \textcolor{red}{Q^T w^*}\\}
				\only<4-> {& = Q(I - (I - \eta \Lambda)^t)Q^T w^* - \eta Q\Lambda(I - (I - \eta \Lambda)^t)Q^T w^* + \eta Q\Lambda Q^T w^* \\}
				\only<4> {& = Q\big[(I - (I - \eta \Lambda)^t) - \eta \Lambda(I - (\textcolor{red}{I} - \eta \Lambda)^t) + \textcolor{red}{\eta \Lambda}\big] Q^T w^*\\}
				\only<5-> {& = Q\big[(I - (I - \eta \Lambda)^t) - \eta \Lambda(I - (I - \eta \Lambda)^t) + \eta \Lambda\big] Q^T w^*\\}
				\only<4> {& = Q\big[I - (I - \eta \Lambda)^t + \eta \Lambda(I - \eta \Lambda)^t\big] Q^T w^*\\}
				\only<5> {& = Q\big[I - \textcolor{red}{(I - \eta \Lambda)^t} + \eta \Lambda \textcolor{red}{(I - \eta \Lambda)^t}\big] Q^T w^*\\}
				\only<6-> {& = Q\big[I - (I - \eta \Lambda)^t + \eta \Lambda (I - \eta \Lambda)^t\big] Q^T w^*\\}
				\only<5> {& = Q\big[I - \textcolor{red}{(I - \eta \Lambda)^t} (I-\eta \Lambda)\big] Q^T w^*\\}
				\only<6-> {& = Q\big[I - (I - \eta \Lambda)^t (I-\eta \Lambda)\big] Q^T w^*\\}
				\only<6-> {& = Q(I - (I - \eta \Lambda)^{t+1}) Q^T w^*}
			\end{align*}
               \only<7-> {Hence, proved!}
        \end{itemize}
	\end{overlayarea}
\end{frame}



\end{document}